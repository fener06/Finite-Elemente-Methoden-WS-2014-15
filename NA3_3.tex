\section{Die Finite-Elemente-Methode (FEM)}

\begin{Definition}
    \label{def:3.1}
    Sei $\Omega \subset \R^2$ ein beschränktes Polygongebiet, d.h. $\Omega$ ist offen und
    zusammenhängend und $\partial\Omega$ ist ein Polygonzug. Dann heißt
    $\Triangulation = \{K_1, \cdots, K_M\}$ eine
    \emph{zulässige Triangulierung} von $\Omega$, wenn
    \begin{enumerate}[a)]
      \item
	    $K_m = \conv\left\{z^{m,0}, z^{m,1}, z^{m,2}\right\} \subset
        \overline\Omega$ ist ein Dreieck mit $\interior(K_m) \neq \emptyset$,
      \item
	    $\overline\Omega = \bigcup_{m=1}^M K_m$,
      \item
	    $\interior(K_m) \cap \interior(K_k) = \emptyset$ und
        $K_m \cap K_k = \conv\left(\left\{z^{m,0}, z^{m,1}, z^{m,2}\right\} \cap
        \left\{z^{k,0}, z^{k,1}, z^{k,2}\right\}\right)$ für $m \neq k$ ist
        leer oder eine gemeinsame Ecke oder eine gemeinsame Kante.
    \end{enumerate}
\end{Definition}


Die \emph{Gitterweite} ist definiert als $h = \max_{K\in\Triangulation} \diam K$
mit $\diam K = 2 \min\{r > 0: K \subset \overline{B(x, r)} \text{ für } x \in K\}$.

Wir verwenden die folgenden Bezeichnungen:
\begin{enumerate}[a)]
    \item $\Nodes = \bigcup_{m=1}^M \{z^{m,0},z^{m,1},z^{m,2}\}$ (Menge der Knotenpunkte)
    \item $\Faces = \bigcup_{m=1}^M \big\{ \conv\{z^{m,0},z^{m,1}\}, \conv\{z^{m,1},z^{m,2}\}, \conv\{z^{m,2},z^{m,0}\} \big\}$ (Menge der Kanten)
    \item $\FacesInterior = \{f \in \Faces: f \subset \Omega \}$ (Menge der inneren Kanten)
\end{enumerate}



\begin{Definition}
    \label{def;3.2}
    Sei $\Polynom_1(\R^2) = \Span\{1, x_1, x_2\}$ der Raum der linearen Polynome auf $\R^2$.
    Zu einer zulässigen Triangulierung $\Triangulation$ ist
    \begin{eqnarray*}
        \FEMLinear = \{v \in C(\overline\Omega): v|_K \in
        \Polynom_1(\R^2) \text{ für alle } K \in \Triangulation\}
    \end{eqnarray*}
    der \term{Raum der (stückweise) linearen Finiten Elemente}.
\end{Definition}


\begin{Satz}
    \label{satz:3.3}
    Eine Funktion $v \in \FEMLinear$ ist eindeutig durch die Knotenwerte $\{ v(z) : z \in \Nodes \}$ bestimmt.
\end{Satz}


\begin{proof}
    Für $x \in \overline\Omega$ existiert nach $\eqref{def:3.1}$ ein $m$ mit
    $x \in K_m$, sodass
    \begin{eqnarray*}
          x
        = \underbrace{(1 - \lambda_1 - \lambda_2)}_{=: \lambda_0} z^{m,0}
          + \lambda_1 z^{m,1} + \lambda_2 z^{m,2}.
    \end{eqnarray*}
    Da $v|_{K_m}$ linear gilt
    \begin{eqnarray*}
          v(x)
        = (1 - \lambda_1 - \lambda_2) v\left(z^{m,0}\right)
          + \lambda_1 v\left(z^{m,1}\right) + \lambda_2 v\left(z^{m,2}\right).
    \end{eqnarray*}
    Sei $x \in K_k$ mit $K_k \cap K_m \neq \emptyset$ für ein $k \neq m$ und
    $x \not \in \Nodes$, dann liegt $x$ auf einer Kante von $K_m$.

    Ohne Einschränkung sei $\lambda_2 = 0$. Dann ist
    \begin{eqnarray*}
          v(x)
        = (1 - \lambda_1) v\left(z^{m,0}\right)
          + \lambda_1 v\left(z^{m,1}\right)
    \end{eqnarray*}
    und $z^{m,0}, \ z^{m,1}\in K_k$.
    Also gilt $v\in C(\overline\Omega)$.
\end{proof}


\begin{Folgerung}
    \label{folgerung:3.4}
    $\FEMLinear = \Span\{\Phi_z: z \in \Nodes\}$ mit einer
    Knotenbasis $\Phi_z \in \FEMLinear$ mit
    \begin{eqnarray*}
        \Phi_z(y) = \begin{cases}
                  1, \qquad z = y \\
                  0, \qquad z \neq y
                \end{cases} \ \forall y \in \Nodes.
    \end{eqnarray*}
\end{Folgerung}


Für alle $v \in \FEMLinear$ gilt $v = \sum_{z\in\Nodes} v(z)
\Phi_z$.


\begin{Definition}
    \label{def:3.5}
    Sei $v \in L_1(\Omega)$. Dann heißt $w_i \in L_1(\Omega)$
    \emph{schwache Ableitung} von $v$ (nach $x_i$), wenn
    \begin{eqnarray*}
        \int_\Omega v \partial_i \Psi \dx = - \int_\Omega w_i \Psi \dx
        \qquad \forall \Psi \in C_0^\infty(\Omega)
    \end{eqnarray*}
    gilt. Dabei ist $C_0^\infty(\Omega) = \{\Psi \in C^\infty(\Omega): \supp\Psi
    \subset \Omega\}$ mit $\supp\Psi = \overline{\{x \in \Omega: \Psi(x)
    \neq 0\}}$ die Menge der \term{Testfunktionen}
\end{Definition}


\begin{Bezeichnung}
    Wir setzen
    $w_i := \partial_i v, \
    \left(\begin{smallmatrix}
            w_1 \\
            w_2
    \end{smallmatrix}\right) := \nabla v$
\end{Bezeichnung}


\begin{Bemerkung}
    Die schwache Ableitung ist eindeutig, denn für
    \begin{eqnarray*}
          -\int_\Omega w_i \Psi \dx = \int_\Omega v \partial_i \Psi \dx
        = - \int_\Omega \tilde w_i \Psi \dx
    \end{eqnarray*}
    folgt
    \begin{eqnarray*}
        \int_\Omega (w_i - \tilde w_i) \Psi \dx = 0.
    \end{eqnarray*}
    Da $C_o^\infty(\Omega)$ dicht in $L_1(\Omega)$ folgt $w_i = \tilde w_i$
    für fast alle $x \in \Omega$.
\end{Bemerkung}


\begin{Satz}
    \label{satz:3.6}
    Zu $v \in \FEMLinear$ definiere 
    \begin{enumerate}[a)]
        \item $\tilde{w}_i(x) = \partial_i v(x)$ für $x \in \bigcup_{m=1}^M \interior(K_m)$,
        \item $\tilde{w}_i(x) \in \R$ beliebig für $x \in \bigcup_{m=1}^M \partial K_m$.
    \end{enumerate}
    Dabei bezeichnet $\partial_i v$ die klassische Ableitung von $v$ nach $x_i$.
    
    Dann ist das so konstruierte $\tilde{w}_i$ die schwache Ableitung von $v$ in $\Omega$.
\end{Satz}


\begin{Bemerkung}
    \begin{enumerate}[a)]
        \item Da $\bigcup_{m=1}^M \partial K_m$ eine Nullmenge in $\R^2$ ist,
                ist $\tilde{w}_i$ in $L_1(\Omega)$ eindeutig bestimmt.
        \item Es gilt $L_2(\Omega) \subset L_1(\Omega)$ für beschränkte $\Omega \subset \R^d$.
    \end{enumerate}
\end{Bemerkung}


\begin{proof}
    Für $\Psi \in C_0^\infty(\Omega)$ gilt
    \begin{eqnarray*}
            \int_\Omega (\tilde{w}_i \Psi + v \partial_i \Psi) \dx
        &=& \sum_{m=1}^M \int_{K_m} (\tilde{w}_i \Psi + v \partial_i \Psi)\dx \\
        &=& \sum_{m=1}^M \int_{\interior(K_m)} \underbrace{(\partial_i v \Psi
            + v \partial_i \Psi)}_{= \partial_i (v \Psi) = \nabla \cdot (v \Psi
            e^i)}\dx \\
        &\stackrel{\text{Gauß}}{=}&
            \sum_{m=1}^M \int_{\partial K_m} v \Psi e^i \cdot \nu \da \\
        &=& \sum_{E\in\mathcal{E}_h^{\interior}} \int_E v \Psi e^i
            (\nu_E - \nu_E) \da
        =  0.
    \end{eqnarray*}
    Denn es gilt $\Psi|_{\partial \Omega} = 0$ und $\partial K = E_1 \cup E_2 \cup E_3$
    besteht aus drei Kanten. Jede innere Kante
    $E \in \mathcal{E}_h^{\interior}$ grenzt an genau zwei Dreiecke $K_k,K_m \in \Triangulation$,
    d.h. $\nu_E = \nu_{\partial K_k} = - \nu_{\partial K_m}$.
\end{proof}


\begin{Definition}
    \label{def:3.7}
    Sei $\Omega \subset \R^d$ ein Lipschitz-Gebiet. Dann definiere
    \begin{eqnarray*}
        H^1(\Omega) = \{v \in L_2(\Omega): v \text{ besitzt schwache
	Ableitung } \partial_i v \in L_2(\Omega),\ i=1,\dots,d\}.
    \end{eqnarray*}
\end{Definition}


\begin{Bemerkung}
    Nach \eqref{satz:3.6} gilt $\FEMLinear \subset H^1(\Omega)$.
\end{Bemerkung}


\begin{Satz}
    \label{satz:3.8}
    $H^1(\Omega)$ ist ein Hilbertraum mit Skalarprodukt
    $(v, w)_1 = (v, w)_0 + \sum_{i=1}^d (\partial_i v, \partial_i w)_0$
    und Norm $\|v\|_1 = \sqrt{(v, v)_1}$.
\end{Satz}


\begin{proof}
    Wir zeigen nur die Vollständigkeit von $H^1(\Omega)$.

    Sei dazu $(v_n)_{n\in\N}$ eine Cauchy-Folge in $H^1(\Omega)$.
    Dann sind nach Definition $(v_n)_{n\in\N}$, $(\partial_i v_n)_{n\in\N}$ Cauchy-Folgen in
    $L_2(\Omega)$, d.h. es existieren $v^*$, $w_i^* \in L_2(\Omega)$ mit
    \begin{eqnarray*}
        \lim_{n\to\infty} \|v^* - v_n\|_0 = 0, \qquad
        \lim_{n\to\infty} \|w^*_i - \partial_i v_n\|_0 = 0.
    \end{eqnarray*}
    Nun zeige $v^* \in H^1(\Omega)$ mit $\partial_i v^* = w_i^*$.

    Für alle $\Psi \in C_0^\infty(\Omega)$ gilt
    \begin{eqnarray*}
          \int_\Omega v^* \partial_i \Psi \dx
        = \lim_{n\to\infty} \int_\Omega v_n \partial_i \Psi \dx
        = \lim_{n\to\infty} - \int_\Omega \partial_i v_n \Psi \dx
        = - \int_\Omega w_i^* \Psi \dx.
    \end{eqnarray*}
    Damit ist $H^1(\Omega)$ vollständig.
\end{proof}


\textbf{Ziel}:
Betrachte eine Randwertaufgabe (RWA) (2. Ordnung) und verallgemeinere den Lösungsbegriff mit Hilfe der schwachen Ableitung auf $H^1(\Omega)$-Funktionen und nutze diese Verallgemeinerung, um eine approximative Lösung $u_h$ der RWA in $\FEMLinear1$ zu definieren. Dann untersuche, ob $u_h$ in $H^1(\Omega)$ gegen die verallgemeinerte Lösung konvergiert, falls die Gitterweite $h$ gegen $0$ geht.

\begin{Beispiel}
    Als Modellproblem betrachten wir die Poisson-Gleichung in $\Omega$ mit homogenen Dirichlet-Randbedingungen
    \begin{eqnarray*}
        -\Delta u &=& f \qquad \text{in } \Omega, \\
        u &=& 0 \qquad \text{auf } \partial \Omega.
    \end{eqnarray*}
    Wir wenden den Divergenzsatz von Gauß an, um eine schwache Formulierung zu erhalten. Für $u \in C^2(\Omega)$ und $\Psi \in C_0^\infty(\Omega)$ gilt $\nabla \cdot (\nabla u \Psi) = \Delta u \Psi + \nabla u \cdot \nabla \Psi$ und somit
    \begin{eqnarray*}
            \int_\Omega (\Delta u \Psi + \nabla u \cdot \nabla \Psi) \dx
        =   \int_{\partial \Omega} \nabla u \Psi \da = 0,
    \end{eqnarray*}
    da $\Psi|_{\partial \Omega} = 0$. Ist nun $u \in C^2(\Omega)$ eine klassische Lösung des Modellproblems, folgt für alle $\Psi \in C_c^\infty(\Omega)$
    \begin{eqnarray*}
        \int_\Omega f \Psi \dx = - \int_\Omega \Delta u \Psi \dx = \int_\Omega \nabla u \cdot \nabla \Psi \dx.
    \end{eqnarray*}
    Die entscheidende Beobachtung ist, dass die Gleichung auch für $u \in H^1(\Omega)$ eine Bedeutung besitzt. Dies führt uns für die approximative Lösung zur folgenden
\end{Beispiel}


\textbf{Idee}:
Bestimme $u_h \in \FEMLinear$ mit $u_h = 0$ auf $\partial\Omega$ und
\begin{eqnarray*}
    \int_\Omega \nabla u_h \cdot \nabla \Psi_h \dx = \int_\Omega f \Psi_h \dx
    \qquad \text{für alle } \Psi_h \in \FEMLinear \text{ mit } \Psi_h = 0
    \text{ auf } \partial\Omega.
\end{eqnarray*}

Sei $\left\{z^1, \dots, z^N\right\} = \Nodes \cap \Omega$ die Menge
der inneren Kotenpunkte und $\Phi_n = \Phi_{z^n}$ die Kotenbasis.

Dann bestimme $u_h = \sum_{k=1}^N \underline u_k \Phi_k$ mit
\begin{eqnarray*}
    \int_\Omega \nabla u_h \cdot \nabla \Phi_n \dx = \int_\Omega f \Phi_n \dx
    \qquad \text{für alle } n=1,\dots,N.
\end{eqnarray*}

Bestimme dazu $\underline u \in \R^N$ durch $\underline A \underline u =
\underline f$ mit der \term{Steifigkeitsmatrix}
\begin{eqnarray*}
      \underline A
    = \left(\int_\Omega \nabla \Phi_n \cdot \nabla \Phi_k \dx\right)
      _{n,k=1,\dots,N}\in \R^{N,N}
\end{eqnarray*}
und der rechten Seite
\begin{eqnarray*}
    \underline f = \left(\int_\Omega f \Phi_n \dx\right)_{n=1,\dots,N}\in \R^N.
\end{eqnarray*}


\begin{Definition}
    \label{def:3.9}
    Zu $K \subset \R^2$ sei
    $h_K = 2 \inf\{r>0: K \subset B(x,r) \text{ für ein } x \in K\}$ der \term{Umkreisdurchmesser} und
    $\rho_K = 2 \sup\{r>0: B(x,r) \subset K \text{ für ein } x \in K\}$ der \term{Inkreisdurchmesser}.
    Eine Familie von Triangulierungen $(\Triangulation)_{h\in \mathcal{H}}$
    heißt
    \begin{enumerate}[a)]
      \item
        \term{regulär}, wenn $C>0$ existiert mit
        \begin{eqnarray*}
            \frac{h_K}{\rho_K} \le C
            \qquad \text{ für alle } K\in \Triangulation
            \text{ und } h \in \mathcal{H}.
        \end{eqnarray*}
      \item
        \term{uniform}, wenn $c>0$ existiert mit
        \begin{eqnarray*}
            c h \le \rho_K \le h_K \le h = \max_{K\in\Triangulation} h_K
            \qquad \text{ für alle } h \in \mathcal{H}.
        \end{eqnarray*}
    \end{enumerate}
\end{Definition}


\begin{Beispiel}
    Sei $\TriangulationLetter_0 = \TriangulationLetter_{h_0}$ eine Triangulierung mit
    $h_0 = \max_{K\in\TriangulationLetter_0} h_K$ und $\rho_0 = \min_{K\in\TriangulationLetter_0}
    \rho_K$. Definiere dazu rekursiv $\TriangulationLetter_k = \TriangulationLetter_{h_k}$ mit
    $h_k = 2^{-k} h_0$ durch Zerlegung von $K = K_1 \cup K_2 \cup K_3 \cup
    K_4$ mit
    \begin{eqnarray*}
        K_1 &=& \conv\left\{z^0, z^{01}, z^{02}\right\} \\
        K_2 &=& \conv\left\{z^1, z^{01}, z^{12}\right\} \\
        K_3 &=& \conv\left\{z^2, z^{02}, z^{12}\right\} \\
        K_4 &=& \conv\left\{z^{12}, z^{01}, z^{02}\right\}.
    \end{eqnarray*}
    Dann ist die Familie $(\TriangulationLetter_k)_{k\in\N}$ uniform mit $ c =
    \frac{\rho_0}{h_0}$.
\end{Beispiel}


\begin{Lemma}
    \label{lem:3.10}
    Sei $\hat K = \conv\left\{\hat z^0, \hat z^1, \hat z^2\right\}$ mit
    $\hat z^0 = \left(\begin{smallmatrix} 0 \\ 0 \end{smallmatrix}\right), \
     \hat z^1 = \left(\begin{smallmatrix} 1 \\ 0 \end{smallmatrix}\right), \
     \hat z^2 = \left(\begin{smallmatrix} 0 \\ 1 \end{smallmatrix}\right)$
    das Referenzdreieck, und sei $\Triangulation$ eine reguläre
    Triangulierung. Dann gilt für die linear affine Transformation
    \begin{eqnarray*}
        \varphi_K: \hat K \to K = \conv\left\{z^0, z^1, z^2\right\} \
        \text{ mit }
        \ \varphi_K (\hat x) = (1 - \hat x_1 - \hat x_2) z^0 + \hat x_1 z^1
        + \hat x_2 z^2
    \end{eqnarray*}
    und $F_K = \varphi_K^\prime = \left(z^1 - z^0, z^2 - z^0\right)\in
    \R^{2,2}, \ J_K = \det F_K$:
    \begin{enumerate}[a)]
      \item
        $|F_K| \le C h_K,
        \quad
        |J_K| \le C h_K^2$.
      \item
        $\left|F_K^{-1}\right| \le C \rho_K^{-1},
        \quad
        \left|J_K^{-1}\right| \le C \rho_K^{-2}$.
    \end{enumerate}
\end{Lemma}


\begin{Bemerkung}
    Falls $(\Triangulation)_{h\in\mathcal{H}}$ regulär, dann gilt
    $\left|F_K^{-1}\right| \le C h_K^{-1}, \
    \left|J_K^{-1}\right| \le C h_K^{-2}$.
\end{Bemerkung}


\begin{proof}
    Für $F_K = \left(z^1 - z^0, z^2 - z^0\right)\in \R^{2,2}$ gilt
    $\varphi_K(\hat x) = z^0 + F_K \hat x$. \\
    Au\ss{}erdem gilt für zwei Punkte $\hat x, \ \hat y \in \hat K$
    \begin{eqnarray*}
          |F_K(\hat x - \hat y)|
        = |\varphi_K(\hat x) - \varphi_K(\hat y)|
        \le h_K.
    \end{eqnarray*}
    Zu $\hat v \in \R^2$ und $\hat x = (\frac{1}{3}, \frac{1}{3})^T$ gilt
    $\hat y = \hat x + \frac{1}{3 |\hat v|} \hat v \in \hat K$.
    Dann gilt
    \begin{eqnarray*}
          |F_K(\hat x - \hat y)|
        = \frac{1}{3 |\hat v|} |F_K \hat v|
        \le h_K.
    \end{eqnarray*}
    Damit ist
    \begin{eqnarray*}
        |F_K \hat v| \le 3 |\hat v| h_K
        \quad \text{ also } \quad
        |F_K| \le 3 h_K
    \end{eqnarray*}
    und es folgt
    \begin{eqnarray*}
          |J_K|
        = |F_K[1,1] F_K[2,2] - F_K[1,2] F_K[2,1]|
        \le 2 |F_K|^2
        \le 18 h_K^2.
    \end{eqnarray*}
    Da $\interior(K) \neq \emptyset$ ist $|K| = \frac{1}{2} J_K \neq 0$.

    Umgekehrt existiert zu jedem $v \in \R^2$ ein $x \in K$ mit
    $y = x + \frac{\rho_K}{2 |v|} v \in K$.

    Dann gilt für
    $\hat x = \varphi_K^{-1}(x) = F_K^{-1}(x - z^0)$ und
    $\hat y = \varphi_K^{-1}(y) = F_K^{-1}(y - z^0)$
    \begin{eqnarray*}
        \frac{\rho_K}{2 |v|} |F_K^{-1} v| = |F_K^{-1}(x - y)| = |\hat x -
        \hat y| \le \sqrt{2}.
    \end{eqnarray*}
    Damit folgt
    \begin{eqnarray*}
        |F_K^{-1}| \le \frac{2 \sqrt{2}}{\rho_K} |v| \text{ also }
        |F_K^{-1}| \le \frac{2 \sqrt{2}}{\rho_K} \le 2 \sqrt{2} C h_K^{-1}.
    \end{eqnarray*}
    Außerdem folgt dann auch
    \begin{eqnarray*}
        |J_K^{-1}| \le \frac{8}{\rho_K^2} \le 8 C^2 h_K^{-2}
    \end{eqnarray*}
    die letzte Abschätzung.
\end{proof}


\begin{Lemma}
    \label{lem:3.11}
    Sei $u \in H^1(\Omega)$. Dann gilt für alle $K \in \Triangulation$:
    \begin{eqnarray*}
        \hat u = u \circ \varphi_K \in H^1\big(\interior(\hat K)\big)
        \text{ und }
        (\nabla u) \circ \varphi_K = F_K^{-T} \hat \nabla \hat u
        \text{ mit }
        \hat \nabla = \left(\frac{\partial}{\partial \hat x_i}\right)
        _{i=1,\dots,d}
    \end{eqnarray*}
\end{Lemma}


\begin{proof}
    Es gilt mit der Kettenregel
    \begin{eqnarray*}
            \frac{\partial}{\partial \hat x_i} \hat u(\hat x)
        &=& \frac{\partial}{\partial \hat x_i} u(\varphi_K(\hat x)) 
        = u^\prime (\varphi_K(\hat x)) \frac{\partial}{\partial \hat x_i}
            \varphi_K(\hat x) \\
        &=& \sum_{j=1}^2 \partial_j u(\varphi_K(\hat x))
            \left(\frac{\partial}{\partial \hat x_i} \varphi_K(\hat x)_j\right)
        = \sum_{j=1}^2 \partial_j u(x) F_K[j,i],
    \end{eqnarray*}
    d.h.
    \begin{eqnarray*}
        \left(\hat \nabla \hat u(\hat x)\right)^T = (\nabla u(x))^T F_K.
    \end{eqnarray*}
    Transponieren liefert
    \begin{eqnarray*}
        \hat \nabla \hat u(\hat x) = F_K^T \nabla u(x)
    \end{eqnarray*}
    und die gesuchte Darstellung folgt durch Multiplikation mit $F^{-T}$ von links.
\end{proof}


\begin{Anwendung}
    Wir betrachten erneut das Poisson-Problem $-\Delta u = f$.
    
    Eine schwache Lösung $u \in H^1(\Omega)$ erfüllt dann
    \begin{eqnarray*}
        \int_\Omega \nabla u \cdot \nabla \Psi \dx = \int_\Omega f \Psi \dx
            \qquad \text{für alle } \Psi \in C_c^\infty(\Omega).
    \end{eqnarray*}
    Für ein $g \in L^1(K)$ folgt gilt mit dem Transformationssatz
    \begin{eqnarray*}
        \int_{\hat K} g(\varphi_K(\hat x)) \left|\det \varphi_K^\prime(\hat x)
        \right| \,d\hat x
        = \int_{K=\varphi_K\left(\hat K\right)} g(x) \dx.
    \end{eqnarray*}
    Angewendet auf beide Seiten der schwachen Formulieren liefert dies
    \begin{eqnarray*}
            \int_\Omega f(x) \Psi(x) \dx
        &=& \sum_{K\in\Triangulation} \int_K f(x) \Psi(x) \dx \\
        &=& \sum_{K\in\Triangulation} \int_{\hat K} f(\varphi_K(\hat x))
            \Psi(\varphi_K(\hat x)) |J_K(\hat x)| \,d\hat x,
    \end{eqnarray*}
    und
    \begin{eqnarray*}
            \int_\Omega \nabla u(x) \cdot \nabla \Psi(x) \dx
        &=& \sum_{K\in\Triangulation} \int_K \nabla u(x) \cdot \nabla \Psi(x)
            \dx \\
        &=& \sum_{K\in\Triangulation} \int_{\hat K} F_K^{-T} \hat \nabla
            (u \circ \varphi_K)(\hat x) F_K^{-T} \hat \nabla (\Psi \circ
            \varphi_K) (\hat x) |J_K| \,d\hat x.
    \end{eqnarray*}
    Das bedeutet, dass sich die auftretenden Integrale in der schwachen Formulierung auf Integrale über dem Referenzelement $\hat{K}$ zurückführen lassen.
\end{Anwendung}


\begin{Definition}
    \label{def:3.12}
    $H^2(\Omega) := \{v \in H^1(\Omega): \partial_i v \in H^1(\Omega),\ i=1,\dots,d\}$
    ist ein Hilbertraum mit Skalarprodukt \\
    $(v,w)_2 = (v,w)_1 + \sum_{i,j=1}^d (\partial_i \partial_j v,
    \partial_i \partial_j w)_0$
    und Norm
    $\|v\|_2 = \sqrt{(v,v)_2}$.
\end{Definition}


\begin{Lemma}
    \label{lem:3.13}
    Es gilt:
    \begin{enumerate}[a)]
      \item
        $u \in L_2(\Omega): \quad
            \|u\|_{0,K}
        \le C h_K \|u \circ \varphi_K\|_{0,\hat K}$.
      \item
        $u \in H^1(\Omega): \quad
            \|\nabla u\|_{0,K}
        \le C \|\hat \nabla (u \circ \varphi_K)\|_{0,\hat K}$.
      \item
        $u \in H^2(\Omega): \quad
            \|\hat \nabla^2 (u \circ \varphi_K)\|_{0,\hat K}
        \le C h_K \|\nabla^2 u\|_{0,K}$.
    \end{enumerate}
    Dabei ist $\|f\|_{0,K} = \left(\int_K |f|^2 \dx\right)^\frac{1}{2}$.
\end{Lemma}


\begin{proof}
    \begin{enumerate}[a)]
      \item
        Für $u \in L_2(\Omega)$ gilt
        \begin{eqnarray*}
                \|u\|_{0,K}^2
            &=& \int_K |u(x)|^2 \dx \\
            &\stackrel{\text{Trafo}}{=}&
                \int_{\hat K} |(u \circ \varphi_K)(\hat x)|^2 |J_K| \,d\hat x \\
            &\le& |J_K| \|u \circ \varphi_K\|_{0,\hat K}^2 \\
            &\stackrel{\eqref{lem:3.10}}{\le}&
                C h_K^2 \|u \circ \varphi_K\|_{0,\hat K}^2.
        \end{eqnarray*}
      \item
        Für $u \in H^1(\Omega)$ gilt
        \begin{eqnarray*}
                \|\nabla u\|_{0,K}^2
            &=& \int_K |\nabla u(x)|^2 \dx \\
            &\stackrel{\text{Trafo}}{=}&
                \int_{\hat K} |\nabla (u \circ \varphi_K)(\hat x)|^2 |J_K|
                \,d\hat x \\
            &\stackrel{\eqref{lem:3.11}}{=}&
                \int_{\hat K} |F_K^{-T} \hat \nabla (u \circ \varphi_K)
                (\hat x)|^2 |J_K| \,d\hat x \\
            &\stackrel{\eqref{lem:3.10}}{\le}&
                C h_K^{-2} h_K^2 \|\hat \nabla (u \circ \varphi_K)\|_{0,\hat K}
                ^2.
        \end{eqnarray*}
      \item
         Für $u \in H^2(\Omega)$ gilt
        \begin{eqnarray*}
                \frac{\partial}{\partial \hat x_i} \frac{\partial}
                {\partial \hat x_j}(u \circ \varphi_K)(\hat x)
            &=& \frac{\partial}{\partial \hat x_i}
                \left(\sum_{k=1}^2 \frac{\partial}{\partial x_k} u(x) F_K[k,j]
                \right) \\
            &=& \sum_{k=1}^2 \sum_{l=1}^2 \frac{\partial}{\partial x_k}
                \frac{\partial}{\partial x_l} u(x) F_K[k,j] F_K[l,i].
        \end{eqnarray*}
        Dann folgt
        \begin{eqnarray*}
            \hat \nabla^2 (u \circ \varphi_K) = F_K (\nabla^2 u \circ \varphi_K) F_K^T.
        \end{eqnarray*}
        Und damit
        \begin{eqnarray*}
                \|\nabla^2 (u \circ \varphi_K)\|_{0,\hat K}^2
            &=& \int_{\hat K} \left|\hat \nabla^2 (u \circ \varphi_K)(\hat x)
                \right|^2 \,d\hat x \\
            &\stackrel{\text{Trafo}}{=}&
                \int_K \left|F_K(\nabla^2 u(x)) F_K^T\right|^2 |J_K^{-1}| \dx\\
            &\le& |F_K|^4 |J_K^{-1}| \int_K |\nabla^2 u(x)| \dx \\
            &\le& C h_K^2 \|\nabla^2 u\|_{0,K}^2,
        \end{eqnarray*}
        da $|F_K|^4 \le (C h_K)^4$ und $|J_K^{-1}| \le C h_K^{-2}$.   
    \end{enumerate}
\end{proof}


\begin{Ausblick}
    Das Ziel der folgenden Überlegungen ist eine Interpolationsabschätzung, mit deren Hilfe sich der Abstand eines $v \in H^2(\Omega)$ und seiner Knoteninterpolation $I_h v := \sum_{i=1}^3 v(z^i) \Phi_i$ in der $H^1$-Norm linear in $h$ abschätzen lässt, d.h.
    \begin{eqnarray*}
        \|v - I_h v\|_{1,K} \le C h \|v\|_{2,K},
    \end{eqnarray*}
    wobei $K = \conv\{z^1,z^2,z^3\}$ eine Zelle und $\Phi_i : K \longrightarrow \R$, $i=1,2,3$, eine Knotenbasis auf $K$ sind.

    Die wesentlichen Hilfsmittel sind \eqref{lem:3.13} und die noch zu zeigende Abschätzung
    \begin{eqnarray*}
        \|\hat \nabla (\hat v - I_h \hat v)\|_{0,\hat K}
        \le C \|\hat \nabla^2 \hat v\|_{0,\hat K}.
    \end{eqnarray*}

    Eine einfache Rechnung liefert die gewünschte Konvergenzordnung für $\|v - I_h v\|_{0,K}$, sodass nur noch $\|\nabla (v - I_h v) \|_{0,K}$ betrachtet werden muss:
    \begin{eqnarray*}
            \|\nabla (v - I_h v)\|_{0,K}
        \le C \|\hat \nabla (\hat v - I_h \hat v)\|_{0,\hat K}
        \le \tilde C \|\hat \nabla^2 \hat v\|_{0,\hat K}
        \le \hat C h_K \|\nabla^2 v\|_{0,K}.
    \end{eqnarray*}
    Die Projektion $Q: H^1(K) \longrightarrow \Polynom_1(\R^2)$ aus \eqref{satz:3.14} spielt dabei eine Schlüsselrolle.
\end{Ausblick}


\begin{Satz}
    \label{satz:3.14}
    Sei $\Omega$ konvex und für $x_0\in \Omega$ gelte $B\left(x_0,
    \frac{\rho}{2}\right)\subset \Omega\subset B\left(x_0, \frac{h}{2}\right)$.
    Für $v\in H^1(\Omega)$ definiere $Qv \in \Polynom_1(\R^2)$ durch
    \begin{eqnarray*}
            Q v(x)
        &=& \frac{1}{|\Omega|} \int_\Omega \big( v(y) + \nabla v(y) \cdot (x-y) \big) \,dy \\
        &=& \frac{1}{|\Omega|} \int_\Omega \big( v(y) - \nabla v(y) \cdot y \big) \,dy \\
            &&+ \left(\frac{1}{|\Omega|} \int_\Omega \partial_1 v(y) \,dy
            \right) x_1 + \left(\frac{1}{|\Omega|} \int_\Omega \partial_2 v(y) y
            \,dy \right) x_2, \qquad x\in \R^2.
    \end{eqnarray*}    
    Ist $v\in H^2(\Omega)$, so gilt für $Rv := v - Qv$
    \begin{eqnarray*}
        R v(x) = \int_\Omega k(x, z) (x-z)^T \nabla^2 v(z) (x-z) \,dz
    \end{eqnarray*}
    wobei $|k(x, z)| \le \frac{h^2}{2 \pi \rho^2} |x-z|^{-2}$ für $x,z \in \Omega$, $x\ne z$.
\end{Satz}


\begin{proof}
    Für $\varphi\in C^2(\R)$ gilt
    \begin{eqnarray*}
          \varphi(1)
        = \varphi(0) + \varphi^\prime(0) + \int_0^1 t \varphi^{\prime\prime}
          (1-t) \dt.
    \end{eqnarray*}
    Definiere $\varphi(t) = v((1-t) y + tx)$. Dann gilt
    \begin{eqnarray*}
        \varphi^\prime(t) = \nabla v((1-t) y + tx) (x-y), \\
        \varphi^{\prime\prime}(t) = (x-y)^T \nabla^2 v((1-t) y + tx) (x-y).
    \end{eqnarray*}
    Damit folgt
    \begin{eqnarray*}
            v(x)
        &=& \varphi(1) \\
        &=& v(y) + \nabla v(y) (x-y) + \int_0^1 t (x-y)^T \nabla^2
            v((1-t) x + ty) (x-y) \dt.
    \end{eqnarray*}
    und mittels Integration über $|\Omega|$ nach $y$ folgt
    \begin{eqnarray*}
          v(x)
        = Q v(x) + \frac{1}{|\Omega|} \int_\Omega \int_0^1 t (x-y)^T \nabla^2
          v((1-t) x + ty) (x-y) \dt \,dy.
    \end{eqnarray*}
    Somit ist
    \begin{eqnarray*}
            R v(x)
        &=& \frac{1}{|\Omega|} \int_\Omega \int_0^1 t (x-y)^T \nabla^2
            v((1-t) x + ty) (x-y) \dt \,dy \\
        &\stackrel{\text{Trafo}}{=}&
            \frac{1}{|\Omega|} \int_\Omega \int_0^1 \chi(x, z, t) (x-z)^T
            \nabla^2 v(z) (x-z) t^{-3} \dt \,dz
    \end{eqnarray*}
    mit $(z, t) = \theta(y, t) := ((1-t) x +ty, t) \subset \Omega \times
    (0, 1)$
    und
    \begin{eqnarray*}
        \chi(x, z, t) = \begin{cases}
                            1 \qquad (z, t)\in
                            \theta_x (\Omega \times (0,1)) \\
                            0 \qquad \text{sonst}
                        \end{cases},
    \end{eqnarray*}
    denn
    \begin{eqnarray*}
          \det \theta^\prime (y, t)
        = \begin{vmatrix}
              t & 0 & y_1 - x_1 \\
              0 & t & y_2 - x_2 \\
              0 & 0 & 1
          \end{vmatrix}
        = t^2
    \end{eqnarray*}
    und $x - z = x - ((1-t) x + ty) = t (x-y)$. Dann ist
    \begin{eqnarray*}
        k(x, z) = \frac{1}{|\Omega|} \int_0^1 \chi(x, z, t) t^{-3} \dt.
    \end{eqnarray*}
    Es gilt $h > |x-y| = \frac{1}{t} |x-z|$ für alle $y\in \Omega$.
    Somit ist
    \begin{eqnarray*}
        \chi(x, z, t) = 0 \qquad \text{für } t < \frac{|z-x|}{h}
    \end{eqnarray*}
    und
    \begin{eqnarray*}
            |k(x, z)|
        \le \left[-\frac{1}{2 |\Omega|} t^{-2}\right]_{t=\frac{|z-x|}{h}}^{t=1}
        \le \frac{h^2}{2 \pi \rho} |z-x|^{-2}
    \end{eqnarray*}
    liefert die Restgliedabschätzung.
\end{proof}


\begin{Satz}
    \label{satz:3.15}
    Es existiert $C > 0$ abhängig von $h, \ \rho$, so dass für alle $v\in
    H^2(\Omega)$ gilt
    \begin{enumerate}[a)]
      \item
        $\|v\|_\infty \le C \|v\|_2$.
      \item
        $\|v - Qv\|_2 \le C \|\nabla^2 v\|_0$.
    \end{enumerate}
\end{Satz}


\begin{Bemerkung}
    Für $d \le 3$ gilt $H^2(\Omega) \subset C\left(\overline\Omega\right)$.
\end{Bemerkung}


\begin{proof}
    \begin{enumerate}[a)]
      \item
        Es gilt
        \begin{eqnarray*}
                  \|v\|_\infty
            &\le& \|v -Qv\|_\infty + \|Qv\|_\infty \\
            &\le& \|Rv\|_\infty + \frac{1}{|\Omega|} \left(\int_\Omega |v(y)|
                  \,dy + h \int_\Omega |\nabla v(y)| \,dy\right) \\
            &\le& C \int_\Omega |\nabla^2 v(z)| \,dz + \frac{1}{|\Omega|}
                  \left(\int_\Omega |v(y)| \,dy + h \int_\Omega |\nabla v(y)|
                  \,dy\right)\\
            &\le& C \sqrt{|\Omega|} \underbrace{\|\nabla^2 v\|_0}_{\le \|v\|_2}
                  + \frac{1}{|\Omega|}
                  \Bigl(\sqrt{|\Omega|} \underbrace{\|v\|_0}
                  _{\le \|v\|_1 \le \|v\|_2} + h \sqrt{|\Omega|}
                  \underbrace{\|\nabla v\|_0}_{\le\|v\|_1 \le \|v\|_2} \Bigr) \\
            &\le& \tilde C \| v\|_2.
        \end{eqnarray*}
      \item
        Es gilt $\nabla^2 Qv \equiv 0$ und damit
        \begin{eqnarray*}
            \|\nabla^2 (v - Qv)\|_0 = \|\nabla^2 v\|_0.
        \end{eqnarray*}
        Au\ss{}erdem gilt
        \begin{eqnarray*}
                \|v -Qv\|_0^2
            &=& \|Rv\|_0^2 \\
            &=& \int_\Omega \left|\int_\Omega k(x, z) (x - z)^T \nabla^2 v(z)
                (x - z) \,dz \right|^2 \dx \\
            &\le& C \int_\Omega \left|\int_\Omega |\nabla^2 v(z)| \,dz \right|^2
                    \dx \\
            &\le& C |\Omega|^2 \|\nabla^2 v\|_0^2
        \end{eqnarray*}
        und
        \begin{eqnarray*}
                \partial_i (v - Qv)(x)
            &=& \partial_i v(x) - \frac{1}{|\Omega|} \int_\Omega \partial_i v(y)
                \,dy \\
            &=& \frac{1}{|\Omega|} \int_\Omega \int_0^1 \nabla \partial_i
                v((1 - t) y + tx)) (x - y) \dt \,dy \\
            &=& \frac{1}{|\Omega|} \int_\Omega k(x, z) \nabla (\partial_i v)(z)
                (x - z) \,dz.
        \end{eqnarray*}
        Dann folgt
        \begin{eqnarray*}
                |\underbrace{\partial_i (v - Qv)(x)}_{=: g(x)}|
            \le \frac{C}{|\Omega|} \int_\Omega |x - z|^{-1}
                |\underbrace{\nabla \partial_i v(z)}_{=: f(x)}| \,dz.
        \end{eqnarray*}
        Damit folgt
        \begin{eqnarray*}
                \|g\|_0^2
            &=& C \int_\Omega \Bigl|\int_\Omega |x - z|^{-1} |f(z)| \,dz
                \Bigr|^2 \dx \\
            &\stackrel{CSU}{\le}& C \int_\Omega \Bigl(\Bigl|\underbrace{\int
                _\Omega |x - z|^{-1} \,dz}_{\le \int_0^{2 \pi} \int_0^h r^{-1} r
                \,dr d\varphi = 2 \pi h} \Bigr| \Bigl|\int_\Omega |x - z|^{-1}
                |f(z)|^2 \,dz \Bigr| \Bigr) \dx \\
            &\le& 2 \pi h C \int_\Omega \int_\Omega |x - z|^{-1} \dx |f(z)|^2
                \,dz \\
            &\le& (2 \pi h)^2 C \|f\|_0^2
        \end{eqnarray*}
        und somit
        \begin{eqnarray*}
            \|\nabla (v - Qv)\|_0 \le C \|\nabla^2 v\|_2
        \end{eqnarray*}
        die Abschätzung.
    \end{enumerate}
\end{proof}


\begin{Folgerung}
    \label{fol:3.16}
    Sei $\Span\{\Phi_{z^0}, \Phi_{z^1}, \Phi_{z^2}\} = \Polynom_1(\R^2)$ und
    $I_h v = \sum_{k=0}^2 v\left(z^k\right) \Phi_{z^k}$ für
    $v\in C(\overline\Omega)$.
    Dann gilt für $v\in H^2(\Omega)$
    \begin{eqnarray*}
        \|v - I_h v\|_2 \le C \|\nabla^2 v\|_0.
    \end{eqnarray*}
\end{Folgerung}


\begin{proof}
    Es gilt $I_h Qv = Qv$ und
    \begin{eqnarray*}
            \|I_h w\|_2
        \le \|w\|_\infty \sum_{k=0}^2 \|\Phi_{z^k}\|_2
        \stackrel{\eqref{satz:3.15}}{\le}
            \tilde C \|w\|_2.
    \end{eqnarray*}
    Dann folgt
    \begin{eqnarray*}
            \|v - I_hv\|_2
        &=& \|v - Qv + I_h(Qv - v)\|_2 \\
        &\le& \|v - Qv\|_2 + \|I_h(v - Qv)\|_2 \\
        &\stackrel{\eqref{satz:3.15}}{\le}& \hat C \|\nabla^2 v\|_0
            + \tilde C \|v - Qv\|_2 \\
        &\stackrel{\eqref{satz:3.15}}{\le}& C \|\nabla^2 v\|_0.
    \end{eqnarray*}
\end{proof}


\begin{Satz}
    \label{satz:3.17}
    Sei $\Triangulation$ uniforme Triangulierung und $I_h v = \sum_{k=0}^2 v(z^k)
    \Phi_{z^k}$. Dann gilt für $v\in H^2(\Omega)$
    \begin{eqnarray*}
        \|I_h v - v\|_1 \le C h \|v\|_2.
    \end{eqnarray*}
\end{Satz}


\begin{proof}
    Für die Knoteninterpolation $I_h v = \sum_{k=0}^2 v(z^k) \Phi_{z^k}$ gilt
    folgende Abschätzung
    \begin{eqnarray*}
            \|\hat\nabla (\hat v - I_h \hat v)\|_{0,\hat K}
        \le \hat C \|\hat\nabla^2 \hat v\|_{0,\hat K}.
    \end{eqnarray*}
    Damit folgt
    \begin{eqnarray*}
            \|\nabla (v - I_h v)\|_0^2
        &=& \sum_{K\in\Triangulation} \|\nabla (v - I_h v)\|_{0,K}^2 \\
        &\stackrel{\eqref{lem:3.13}}{\le}& C \sum_{\hat K\in\Triangulation}
              \|\hat\nabla (v \circ \varphi_K
              - \hat I_h (v \circ \varphi_K))\|_{0,\hat K} \\
        &\le& C \sum_{\hat K\in\Triangulation} \hat C
              \|\hat \nabla^2 (v \circ \varphi_K)\|_{0,\hat K}^2 \\
        &\stackrel{\eqref{lem:3.13}}{\le}& \tilde C
              \sum_{K\in\Triangulation} h_K^2 \|\nabla^2 v\|_{0,K}^2 \\
        &\le& \tilde C h^2 \|\nabla^2 v\|_0^2 \\
        &\le& \overline C h^2 \|v\|_2^2
    \end{eqnarray*}
    Durch Nachrechnen lässt sich zeigen, dass
    \begin{eqnarray*}
            \|v - I_h v\|_0
        \le C h \|v\|_2^2.
    \end{eqnarray*}
    Insgesamt erhalten wir also den Interpolationsfehler.
\end{proof}

\begin{Bemerkung}
    Sei $(\Triangulation)_{h\in\mathcal{H}}$ eine Familie von Triangulierungen. Falls $\mathcal{H}$ eine Nullfolge enthält, bezeichnen wir $(\Triangulation)_{h\in\mathcal{H}}$ als eine Familie von Triangulierungen mit $h \to 0$.
\end{Bemerkung}


\begin{Folgerung}
    \label{folgerung:3.18}
    Sei $(\Triangulation)_{h\in\mathcal{H}}$ eine Familie uniformer
    Triangulierungen mit $h \to 0$. Dann ist $\bigcup_{h\in \mathcal{H}}
    \FEMLinear$ dicht in $H^1(\Omega)$.
\end{Folgerung}


\begin{proof}
    Zu $\epsilon > 0$ und $v\in H^1(\Omega)$ wähle
    $v_\epsilon\in C^2(\overline\Omega)$ mit
    $\|v - v_\epsilon\|_1 \le \frac{\epsilon}{2}$.
    Dann gilt
    \begin{eqnarray*}
            \|v - I_h v_\epsilon\|_1
        &\le& \|v - v_\epsilon\|_1 + \|v_\epsilon - I_h v_\epsilon\|_1 \\
        &\le& \frac{\epsilon}{2} + C h \|v_\epsilon\|_2.
    \end{eqnarray*}
    Wähle $h$ mit $C h \|v_\epsilon\|_2 \le \frac{\epsilon}{2}$.
\end{proof}


\begin{Definition}
    \label{def:3.19}
    \begin{enumerate}[a)]
      \item
        Zu $\Gamma_3 \subset \partial\Omega$ definiere $V_h = \{v\in
        \FEMLinear: v(x) = 0 \text{ für } x\in \Gamma_3\}$.
      \item
        Sei $(\Triangulation)_{h\in\mathcal{H}}$ eine Familie uniformer
        Triangulierungen, $\bigcup_{h\in\mathcal{H}} \FEMLinear$ dicht in
        $H^1(\Omega)$. Dann sei $V \subset H^1(\Omega)$ der kleinste
        Hilbertraum, der alle $V_h$ enthält.
    \end{enumerate}
\end{Definition}


\begin{Bemerkung}
    \begin{enumerate}[1)]
      \item
        Es existiert $C > 0$ mit
        \begin{eqnarray*}
                \|v_h\|_{0,\Gamma_3}
            \le C \|v_h\|_1 \qquad \forall v_h\in \FEMLinear.
        \end{eqnarray*}
        Da $\bigcup_{h\in\mathcal{H}} \FEMLinear$ dicht in $H^1(\Omega)$,
        ist die Spur-Abbildung
        \begin{eqnarray*}
            \gamma_3: H^1(\Omega) &\to& L_2(\Gamma_3) \\
            v &\mapsto& v|_{\Gamma_3}
        \end{eqnarray*}
        wohldefiniert und stetig, d.h. es existiert ein $C>0$ mit
        \begin{eqnarray*}
            \|\gamma_3(v)\|_{0,\Gamma_3} \le C \|v\|_1.
        \end{eqnarray*}
      \item
        Es gilt unabhängig von $\Triangulation$
        \begin{eqnarray*}
            V = \{v\in H^1(\Omega): \gamma_3(v) = 0\}.
        \end{eqnarray*}
    \end{enumerate}
\end{Bemerkung}


\begin{proof}
    Von 1)
    Für $\Omega = (0, a_1) \times (0, a_2)$ und $\Gamma_3 = (0, a_1) \times
    \{0\}$ gilt
    \begin{eqnarray*}
          |v_h(x_1, x_2)|^2
        = |v_h(x_1, 0)|^2 + \int_0^{x_2} \partial_2 |v_h(x_1, t)|^2 \dt,
    \end{eqnarray*}
    sodass
    \begin{eqnarray*}
            |v_h(x_1, 0)|^2
        &\le& |v_h(x_1, x_2)|^2
              + 2 \int_0^{x_2} |v_h(x_1, t)| |\partial_2 v_h(x_1, t)| \dt \\
        &\le& \frac{1}{a_2} \int_0^{a_2} \left(|v_h(x_1, x_2)|^2
              + 2 \int_0^{a_2} |v_h(x_1, t)| |\nabla v_h(x_1, t)| \dt\right)
              \dx_2 \\
        &\le& \frac{1}{a_2} \int_0^{a_2} |v_h(x_1, t)|^2 \dx_2
              + 2 \int_0^{a_2} |v_h(x_1, t)| |\nabla v_h(x_1, t)| \dt.
    \end{eqnarray*}
    Damit folgt
    \begin{eqnarray*}
            \|v_h\|^2_{0,\Gamma_3}
        &=& \frac{1}{a_2} \int_0^{a_2} \int_0^{a_1} |v_h(x_1, 0)|^2 \dx_1
            \dx_2 \\
        &\le& \frac{1}{a_2} \int_0^{a_2} \int_0^{a_1} |v_h(x_1, x_2)|^2 \dx_1
              \dx_2 \\
              &&+ 2 \int_0^{a_2} \int_0^{a_1} |v_h(x_1, t)| \ |\nabla
              v_h(x_1, t)| \dx_1 \dt \\
        &\stackrel{\text{CSU}}{\le}& \frac{1}{a_2} \|v_h\|_0^2
              + 2 \|v_h\|_0 \|\nabla v_h\|_0 \\
        &\le& C \|v_h\|_1^2
    \end{eqnarray*}
    der Spursatz.
\end{proof}


\begin{Lemma}[Poincar\`e-Friedrichs-Ungleichung]
    \label{lem:3.20}
    Sei $\meas_{d-1}(\Gamma_3) > 0$.  Dann existiert $C > 0$ mit
    \begin{eqnarray*}
        \|v\|_0 \le C (\|\nabla v\|_0 + \|v\|_{0,\Gamma_3})
        \qquad \forall v\in H^1(\Omega).
    \end{eqnarray*}
    Insbesondere gilt
    \begin{eqnarray*}
        \|v\|_0 \le C \|\nabla v\|_0 \qquad \forall v\in V.
    \end{eqnarray*}
\end{Lemma}


\begin{proof}
    Für das Beispiel $\Omega = (0,a_1)\times (0,a_2)$, $\Gamma_3 = (0,a_1)\times\{0\}$ gilt
    \begin{eqnarray*}
            |v(x_1, x_2)|^2
        \le |v(x_1, 0)|^2 + 2 \int_0^{a_2} |v(x_1, t)| |\nabla v(x_1, t)| \dt.
    \end{eqnarray*}
    Damit folgt wegen
    \begin{eqnarray*}
              \|v\|_0^2
        &\le& a_2 \|v\|_{0,\Gamma_3}^2 + a_2 \underbrace{2 (|v|, |\nabla v|)_0}
              _{\le \|v\|_0^2 + \|\nabla v\|_0^2} \\
        &\le& a_2 \left(\|v\|_{0,\Gamma_3}^2 + \|v\|_0^2 + \|\nabla v\|_0^2 \right) \\
        &\le& a_2 \left(\|v\|_{0,\Gamma_3} + \|v\|_0 + \|\nabla v\|_0 \right)^2.
    \end{eqnarray*}
    die Poincar\`e-Friedrichs-Ungleichung. TODO: $a_2 \ge 1$?
\end{proof}


\begin{Bemerkung}
    Sei $\lambda_1 > 0 $ die kleinste Zahl, sodass ein $w_1 \in V\setminus \{0\}$ existiert mit
    \begin{eqnarray*}
        (\nabla w_1, \nabla v)_0 = \lambda_1 (w_1, v)_0 \qquad \forall v\in V,
    \end{eqnarray*}
    d.h. $w_1$ ist Eigenfunktion zum kleinsten Eigenwert $\lambda_1$.
    Dann gilt TODO: Warum?
    \begin{eqnarray*}
        \|v\|_0 \le \frac{1}{\sqrt{\lambda_1}} \|\nabla v\|_0
        \qquad \forall v\in V.
    \end{eqnarray*}
\end{Bemerkung}


\begin{Satz}
    \label{satz:3.21}
    Sei $K\in C^1(\overline\Omega, \R^{d,d})$, $c\in C(\Omega, \R^d)$, $q,f
    \in C(\Omega)$, $g_i\in C(\Gamma_i)$ mit $\Gamma_1 \cup \Gamma_2 \cup
    \Gamma_3 = \partial\Omega$, $\alpha\in C(\Gamma_2)$ und $L$ ein Differentialoperator zweiter Ordnung mit $L u = -\nabla \cdot (K \nabla u) + c \cdot \nabla u + q u$. \\
    Sei $u\in C^2(\Omega) \cap C^1(\overline\Omega)$ klassische Lösung der
    Randwertaufgabe
    \begin{eqnarray*}
        L u &=& f \qquad \ \text{in } \Omega \\
        K \nabla u \cdot \nu &=& g_1 \qquad \text{auf } \Gamma_1 \quad (Neumann-RB) \\
        K \nabla u \cdot \nu + \alpha u &=& g_2 \qquad \text{auf } \Gamma_2 \quad (Robin-RB) \\
        u &=& g_3 \qquad \text{auf } \Gamma_3 \quad (Dirichlet-RB).
    \end{eqnarray*}
    Dann ist $u \ \emph{schwache Lösung}$ von
    \begin{eqnarray*}
        a(u,v) = l(v) \qquad \forall v\in V
    \end{eqnarray*}
    mit
    \begin{eqnarray*}
            a(u,v)
        &=& \int_\Omega (K \nabla u \cdot \nabla v + c \cdot \nabla u v + q u v)
            \dx + \int_{\Gamma_2} \alpha u v \da, \\
            l(v)
        &=& \int_{\Omega} f v \dx + \int_{\Gamma_1} g_1 v \da
            + \int_{\Gamma_2} g_2 v \da.
    \end{eqnarray*}
\end{Satz}


\begin{proof}
    Sei $v\in V \cap C^1(\overline\Omega)$, dann definiere $w = v K \nabla u$
    und es  gilt
    \begin{eqnarray*}
            \nabla \cdot w
        &=& \sum_{i = 1}^d \partial_i \left(v \sum_{j=1}^d K_{ij} \partial_j u
            \right) \\
        &=& \sum_{i,j = 1}^d (K_{ij} \partial_i v \partial_j u + v \partial_i
            (K_{ij} \partial_j u))) \\
        &=& \nabla v \cdot (K \nabla u) + v \nabla \cdot (K \nabla u).
    \end{eqnarray*}
    Mit dem Satz von Gauss
    \begin{eqnarray*}
        \int_\Omega \nabla \cdot w \dx = \int_{\partial\Omega} w \cdot \nu
        \da
    \end{eqnarray*}
    folgt schlie\ss{}lich
    \begin{eqnarray*}
            \int_\Omega (K \nabla u \cdot \nabla v + c \cdot \nabla u v + q u v
            - f v) \dx
        &=& \int_{\Gamma_1 \cup \Gamma_2} (K \nabla u) \cdot \nu v \da \\
        &=& \int_{\Gamma_1} g_1 v \da + \int_{\Gamma_2} (g_2 - \alpha u) v
            \da,
    \end{eqnarray*}
    da $v = 0$ auf $\Gamma_3$ für alle $v\in V$.
\end{proof}


\begin{Definition}
    \label{def:3.22}
    Sei $K\in L_\infty(\Omega, \R^{d,d}), \ c\in L_\infty(\Omega, \R^d), \ q\in
    L_\infty(\Omega), \ \alpha\in L_\infty(\Gamma_2), \ f\in L_2(\Omega), \
    g_1\in L_2(\Gamma_1)$ und $g_2\in L_2(\Gamma_2)$.
    Dann hei\ss{}t $u\in H^1(\Omega)$ mit $\gamma_3(v) = g_3$ und
    \begin{eqnarray*}
        a(u,v) = l(v) \qquad \forall v\in V
    \end{eqnarray*}
    \emph{schwache Lösung} der Randwertaufgabe $\eqref{satz:3.21}$, und
    $u_h\in \FEMLinear$ mit $\gamma_3(u_h) = I_h g_3$ und
    \begin{eqnarray*}
        a(u_h,v_h) = l(v_h) \qquad \forall v_h\in V_h
    \end{eqnarray*}
    hei\ss{}t $\emph{Galerkin-Approximation}$ von $u$.
\end{Definition}


\begin{Lemma}
    \label{lem:3.23}
    Unter der Voraussetzung $\eqref{def:3.22}$ sind die Bilineraform und die
    Linearform
    \begin{eqnarray*}
        a(\cdot,:)&:& H^1(\Omega) \times H^1(\Omega) \to \R, \\
        l(\cdot)&:& H^1(\Omega) \to \R
    \end{eqnarray*}
    stetig in $H^1(\Omega)$.
\end{Lemma}


\begin{proof}
    Mit dem Spursatz folgt
    \begin{eqnarray*}
              |a(u,v)|
        &\le& \|K\|_\infty \|\nabla u\|_0 \|\nabla v\|_0 + \|\nabla c\|_\infty
              \|\nabla u\|_0 \|v\|_0 + \|q\|_\infty \|u\|_0 \|v\|_0 \\
              &&+ \|\alpha\|_{\infty,\Gamma_2} \|u\|_{0,\Gamma_2}
              \|v\|_{0,\Gamma_2}
              \\
        &\le& \tilde{C}(\|K\|_\infty + \|c\|_\infty + \|q\|_\infty + \|\alpha\|_\infty)
              \|u\|_1 \|v\|_1 \\
        &\le& C \|u\|_1 \|v\|_1
    \end{eqnarray*}
    und
    \begin{eqnarray*}
              |l(v)|
        &\le& \|f\|_0 \|v\|_0 + \|g_1\|_{0,\Gamma_1} \|v\|_{0,\Gamma_1} +
              \|g_2\|_{0,\Gamma_2} \|v\|_{0,\Gamma_2} \\
        &\le& C \|v\|_1.
    \end{eqnarray*}
    Damit erhalten wir die Stetigkeit der Bilinearform und Linearform in
    $H^1(\Omega)$.
\end{proof}


\begin{Satz}
    \label{satz:3.24}
    Es gelte
    \begin{enumerate}[a)]
      \item
        $K$ positiv definit, d.h. $z^T K(x) z > k_0 |z|^2$ für alle $z\in
        \R^d$und fast alle $x\in \Omega$ mit $k_0 > 0$.
      \item
        $\nabla \cdot c \in L_\infty(\Omega)$ und $q - \frac{1}{2} \nabla \cdot
        c \ge 0$ in $\Omega$.
      \item
        $\nu \cdot c \ge 0$ auf $\Gamma_1$.
      \item
        $\alpha + \frac{1}{2} \nu \cdot c \ge 0$ auf $\Gamma_2$.
      \item
        und eine der weiteren Bedingungen sei erfüllt:
        \begin{enumerate}[i)]
          \item
            $\meas_{d-1}(\Gamma_3) > 0$.
          \item
            es existiert $\Omega^\prime \subset \Omega$ mit $\meas_{d}
            (\Omega^\prime) > 0$ und $q_0 > 0$ mit $q - \frac{1}{2} \nabla \cdot
            c \ge q_0 > 0$ in $\Omega^\prime$.
          \item
            es existiert $\Gamma^\prime \subset \Gamma_1$ mit $\meas_{d-1}
            (\Gamma^\prime) > 0$ und $c_0 > 0$ mit $\nu \cdot c \ge c_0 > 0$ auf
            $\Gamma^\prime$.
          \item
            es existiert $\Gamma^\prime \subset \Gamma_2$ mit $\meas_{d-1}
            (\Gamma^\prime) > 0$ und $c_0 >0$ mit $\alpha + \frac{1}{2} \nu
            \cdot c \ge c_0 > 0$ auf $\Gamma^\prime$.
        \end{enumerate}
    \end{enumerate}
    Dann ist $a(\cdot, :)$ elliptisch, d.h. es exisitiert $\alpha_0 > 0$ mit
    \begin{eqnarray*}
        a(v, v) \ge \alpha_0 \|v\|_1^2 \qquad \forall v\in V.
    \end{eqnarray*}
\end{Satz}


\begin{proof}
    Es gelten für $v \in V$
    \begin{eqnarray*}
        \nabla(v^2) = 2 v \nabla v
    \end{eqnarray*}
    und
    \begin{eqnarray*}
        \nabla \cdot (c v^2) = (\nabla \cdot c) v^2 + 2 (c \cdot \nabla v) v.
    \end{eqnarray*}
    Dann folgt mit dem Divergenzsatz von Gauß
    \begin{eqnarray*}
            \int_\Omega (c \cdot \nabla v) v \dx
        &=& \frac{1}{2} \int_\Omega \nabla \cdot (c v^2) \dx
            - \frac{1}{2} \int_\Omega (\nabla \cdot c) v^2 \dx \\
        &=& \frac{1}{2} \int_\Omega v^2 c \cdot \nu \dx
            - \frac{1}{2} \int_{\Gamma_1} (\nabla \cdot c) v^2f \dx.
    \end{eqnarray*}
    und somit
    \begin{eqnarray*}
            a(v, v)
        &=& \int_\Omega \left((K \nabla v) \cdot \nabla v
            + (c \cdot \nabla v) \cdot v + q v^2\right) \dx
            + \int_{\Gamma_2} \alpha v^2 \da \\
        &\ge& \int_\Omega \left(k_0 |\nabla v|^2 + \left(q - \frac{1}{2} \nabla
              \cdot c\right) v^2\right) \dx + \frac{1}{2} \int_{\Gamma_1} v^2 c
              \cdot \nu \da \\
              &&+ \int_{\Gamma_2} (\alpha + \frac{1}{2} c \cdot \nu) v^2 \da \\
        &\ge& k_0 \|\nabla v\|_0^2.
    \end{eqnarray*}
    \begin{enumerate}[i)]
      \item
        Es gilt
        \begin{eqnarray*}
              \|v\|_1^2        
            = \|v\|_0^2 + \|\nabla v\|_0^2
            \stackrel{\eqref{lem:3.20}}{\le} (1 + C_{PF}^2) \|\nabla v\|_0^2. 
        \end{eqnarray*}
        Somit folgt
        \begin{eqnarray*}
                a(v, v)
            \ge k_0 \|\nabla v\|_0^2
            \ge \frac{k_0}{1 + C_{PF}^2} \|v\|_1^2.
        \end{eqnarray*}
      \item
        Eine Variante von $\eqref{lem:3.20}$ ist
        \begin{eqnarray*}
            \|v\|_0 \le C_{PF} (\|\nabla v\|_0 + \|v\|_{0,\Omega^\prime}).
        \end{eqnarray*}
        Dann gilt
        \begin{eqnarray*}
                a(v, v)
            \ge \min\{k_0, q_0\} (\|\nabla v\|_0^2 + \|v\|_{0,\Omega^\prime}^2)
            \ge \frac{1}{2} \frac{\min\{k_0, q_0\}}{1 + C_{PF}^2} \|v\|_1^2.
        \end{eqnarray*}
      \item
        Mit $\eqref{lem:3.20}$ gilt
        \begin{eqnarray*}
                a(v, v)
            \ge k_0 \|\nabla v\|_0^2 + c_0 \|v\|_{0,\Gamma^\prime}^2
            \ge \frac{\min\{k_0, c_0\}}{1 + C_{PF}^2} \|v\|_1^2.
        \end{eqnarray*}
      \item
        Analog.
    \end{enumerate}
\end{proof}


$\textbf{Stabilität}$

Sei $a(\cdot, :)$ elliptisch und für $u_h\in \FEMLinear$ gelte
$a(u_h, v_h) = l(v_h)$. Dann gilt
\begin{eqnarray*}
    \|u_h\|_1^2 \le \frac{1}{\alpha} a(u_h, u_h) = \frac{1}{\alpha} l(u_h)
    \le \frac{C_l}{\alpha} \|u_h\|_1.
\end{eqnarray*}
Also
\begin{eqnarray*}
    \|u_h\|_1 \le \frac{C_l}{\alpha},
\end{eqnarray*}
d.h. die diskrete Lösung $u_h$ bleibt durch eine von $h$ unabhängige Konstante beschränkt.


\begin{Satz}
    \label{satz:3.25}
    Sei $a(\cdot, :)$ elliptisch und sei $u_D\in H^1(\Omega)$ mit
    $\gamma_3(u_D) = g_3$. Dann existiert genau ein $u_h\in V_h$ mit
    \begin{eqnarray*}
        a(u_h, v_h) = l(v_h) - a(u_D, v_h) =: l_D(v_h)
        \qquad \forall v_h\in V_h.
    \end{eqnarray*}
\end{Satz}


\begin{proof}
    Es gilt $V_h = \Span\{\Phi_z: z\in\Nodes \setminus \Gamma_3\}$.
    Der Ansatz
    \begin{eqnarray*}
        u_h = \sum_{z\in\Nodes \setminus \Gamma_3} \underline u_z \Phi_z
    \end{eqnarray*}
    liefert
    \begin{eqnarray*}
        a(u_h, \Phi_z) = l_D(\Phi_z) \ \forall z\in \Nodes \setminus
        \Gamma_3
        \quad \text{genau dann, wenn} \quad
        %\qquad\Leftrightarrow \qquad
        \underline A \underline u = \underline b
    \end{eqnarray*}
    mit
    \begin{eqnarray*}
        \underline A = (a(\Phi_{z^\prime}, \Phi_z))_{z,z^\prime\in \Nodes
        \setminus \Gamma_3}
        \quad \text{und} \quad \underline b = (l_D(\Phi_z))
        _{z\in \Nodes \setminus \Gamma_3}.
    \end{eqnarray*}
    Annahme: $\underline A$ sei singulär.
    
    Dann existiert ein $\underline v\in \R^{|\Nodes \setminus \Gamma_3|}
    \setminus \{0\}$ mit $\underline A \underline v = 0$. Dann gilt
    \begin{eqnarray*}
        0 = a(v_h, v_h) \ge \alpha_0 \|v_h\|_1^2
        \qquad \text{für } \qquad
        v_h = \sum_{z\in\Nodes \setminus \Gamma_3} \underline v_z \Phi_z.
    \end{eqnarray*}
    Damit ist $\|v_h\|_1 = 0$ und somit $\underline v = 0$, was ein Widerspruch
    zur Voraussetzung ist.
    Also ist $\underline u = \underline A^{-1} \underline b$ eindeutige
    Lösung.
\end{proof}


\begin{Satz}
    \label{satz:3.26}
    Sei $a(\cdot, :)$ elliptisch und symmetrisch und beschränkt. Definiere
    \begin{eqnarray*}
        F(v) = \frac{1}{2} a(v + u_D, v + u_D) - l(v) \qquad \text{für } v \in V.
    \end{eqnarray*}
    \begin{enumerate}[a)]
      \item
        Für die Lösung $u_h\in V_h$ in $\eqref{satz:3.25}$ gilt
        \begin{eqnarray*}
            F(u_h) \le F(v_h) \qquad \forall v_h\in V_h.
        \end{eqnarray*}
      \item
        Sei $(V_h)_{h\in\mathcal{H}}$ dicht in $V$ mit $h \to 0$. Dann ist die
        Lösung $(u_h)_{h\in\mathcal{H}}$ eine Minimalfolge von $F(\cdot)$,
        d.h.
        \begin{eqnarray*}
            \liminf_{h\to0} F(u_h) = \inf_{v\in V} F(v) > - \infty.
        \end{eqnarray*}
      \item
        Es existiert ein eindeutiges $u\in V$ mit $F(u) \le F(v)$ für alle
        $v\in V$. Es gilt $u_h \to u$ für $h \to 0$.
      \item
        Und $u\in V$ ist eindeutig durch
        \begin{eqnarray*}
            a(u, v) = l(v) - a(u_D, v) \qquad \forall v\in V
        \end{eqnarray*}
        charakterisiert.
    \end{enumerate}
\end{Satz}


\begin{proof}
    Da $a(\cdot, :)$ symmetrisch ist, gilt für $v \in V$
    \begin{eqnarray*}
          F(v)
        = \frac{1}{2} a(v, v) - \underbrace{(l(v) - a(u_D, v))}_{=: l_D(v)}
          + \frac{1}{2} a(u_D, u_D).
    \end{eqnarray*}
    \begin{enumerate}[a)]
      \item
        Definiere
        \begin{eqnarray*}
              \underline F(\underline v)
            = F\left(\sum_{z\in\Nodes \setminus \Gamma_3} \underline v_z
              \Phi_z\right).
        \end{eqnarray*}
        Dann gilt
        \begin{eqnarray*}
              \underline F(\underline v)
            = \frac{1}{2} \underline v^T \underline A \underline v - \underline
              v^T \underline b + \frac{1}{2} a(u_D, u_D)
        \end{eqnarray*}
        und es folgt
        \begin{eqnarray*}
                \nabla \underline F(\underline v)
            &=& \underline A \underline v - \underline b, \\
                \nabla^2 \underline F(\underline v)
            &=& \underline A .
        \end{eqnarray*}
        Da $a(\cdot, :)$ elliptisch ist $\underline A$ positiv definit und
        somit ist $\underline u = \underline A^{-1} \underline b$ einzige
        kritische Stelle und $\underline u$ ist eindeutiges Minimum von
        $\underline F$. Also ist $u_h$ eindeutiges Minimum von $F$ in $V_h$.
      \item
        Zeige: $\inf_{v\in V} F(v) > - \infty$.
    
        Mit der Young'schen Ungleichung
        \begin{eqnarray*}
            2xy = 2 \sqrt{s} x \frac{1}{\sqrt{s}} y \le s x^2 + \frac{1}{s} y^2
            \ \ \text{genau dann, wenn} \ \
            \left(\sqrt{s} x - \frac{1}{\sqrt{s}} y\right)^2 \ge 0
        \end{eqnarray*}
        gilt für
        \begin{eqnarray*}
                F(v)
            &=& \frac{1}{2} a(v, v) - l_D(v) + \frac{1}{2} a(u_D, u_D) \\
            &\ge& \frac{\alpha_0}{2} \|v\|_1^2 - \underbrace{(C_l + C_a \|u_D\|
                _1)}_{=: \tilde C} \|v\|_1 + \frac{1}{2} a(u_D, u_D) \\
            &\ge& \frac{\alpha_0}{2} \|v\|_1^2 - \frac{1}{2} \left(s \tilde C^2
                + \frac{1}{s} \|v\|_1^2\right) + \frac{1}{2} a(u_D, u_D)
                \qquad \left(\frac{1}{s} = \alpha_0\right) \\
            &=& - \frac{1}{2 \alpha_0} \tilde C^2 + \frac{1}{2} a(u_D, u_D)
            > - \infty.
        \end{eqnarray*}
        Nun wähle eine Minimalfolge $(v^n)_{n\in \N}$ mit
        $\lim_{n\to\infty} F(v^n) = \inf_{v\in V} F(v)$.
        Da $(V_h)_{h\in \mathcal{H}}$ dicht in $V$ und $F$ stetig gilt o.E.
        $v^n\in \bigcup_{h\in\mathcal{H}} V_h$ und somit ist $(u_h)
        _{h\in\mathcal{H}}$ Minimalfolge. TODO: das sollte umformuliert werden!
      \item
        Zeige jede Minimalfolge ist Cauchy-Folge in $V$.

        Behauptung: $F$ ist gleichmä\ss{}ig konvex. Insbesondere gilt dann
        \begin{eqnarray*}
                F\left(\frac{1}{2} \left(v^1 + v^2\right)\right)
            \le \frac{1}{2} (F(v^1) + F(v^2)) - \frac{\alpha_0}{8}
                \| v^1 - v^2\|_1^2 \qquad \forall v^1, \ v^2 \in V.
        \end{eqnarray*}
        Setze dazu $w = \frac{1}{2} \left(v^1 + v^2\right)\in V$.
        Da $F$ quadratisch ist gilt
        \begin{eqnarray*}
              F\left(v^i\right)
            = F(w) + \left(a\left(w, v^i - w\right)
              - l_D\left(v^i - w\right)\right)
              + \frac{1}{2} a\left(v^i - w, v^i - w\right).
        \end{eqnarray*}
        Dann folgt
        \begin{eqnarray*}
                F\left(v^1\right) + F\left(v^2\right) - 2 F(w)
            &=& \frac{1}{4} a\left(v^1 - v^2, v^1 - v^2\right) \\
            &\ge& \frac{\alpha_0}{4} \|v^1 - v^2\|_1^2.
        \end{eqnarray*}
        Zu $\delta > 0$ wähle $n_0$ mit $F\left(v^n\right) \le \inf_{v \in V} F(v)
        + \delta$. Damit folgt
        \begin{eqnarray*}
                  \|v^n - v^m\|_1^2
            &\le& \frac{8}{\alpha_0} \left(\frac{1}{2} F\left(v^m\right)
                  + \frac{1}{2} F\left(v^n\right)
                  - F\left(\frac{1}{2}\left(v^m - v^n\right)\right)\right) \\
            &\le& \frac{8}{\alpha_0} (\inf F + \delta - \inf F) \\
            &=& \frac{8}{\alpha_0} \delta
            \qquad \forall n, \ m \ge n_0.
        \end{eqnarray*}
        Somit ist $(v^n)_{n\in \N}$ Cauchy-Folge.
      \item
        Sei $u$ Lösung von c).
        Wähle $v\in V \setminus \{0\}, t\in \R$ dann ist
        \begin{eqnarray*}
                F(u)
            \le F(u + tv)
            =   F(u) + (a(u, tv) - l_D(tv)) + \frac{1}{2} a(tv, tv),
        \end{eqnarray*}
        sodass
        \begin{eqnarray*}
            -t(a(u, v) - l_D(v)) \le \frac{C_a}{2} t^2 \|v\|_1^2.
        \end{eqnarray*}
        Annahme: Es existiert ein $v\in V$ mit $a(u, v) \neq l_D(v)$.

        Dann ist $v \neq 0$ und für obige Ungleichung mit
        $t = \frac{l_D(v) - a(u, v)}{C_a \|v\|_1^2}$
        gilt $1 \le \frac{1}{2}$, was ein Widerspruch ist.

        Sei umgekehrt $u$ Lösung von d). Dann folgt
        \begin{eqnarray*}
              F(u + v)
            = F(u) + (\underbrace{a(u, v) - l_D(v)}_{= 0}) +
              \frac{1}{2} \underbrace{a(v, v)}_{\ge 0}
            \ge F(u) 
            \qquad \forall v\in V.
        \end{eqnarray*}
        Somit ist die Lösung $u$ eindeutig charakterisiert.
    \end{enumerate}
\end{proof}


\begin{Satz}
    \label{satz:3.27}
    Sei $a(\cdot, :)$ elliptisch und beschränkt. Dann existiert eine
    eindeutige Lösung $u\in V$ mit
    \begin{eqnarray*}
        a(u, v) = l_D(v) \qquad \forall v\in V.
    \end{eqnarray*}
\end{Satz}


\begin{proof}
    Definiere dazu einen Operator $A: V \to V$ mit
    \begin{eqnarray*}
        (Av, w)_1 = a(v, w) \qquad \forall v, \ w\in V.
    \end{eqnarray*}
    Setze $\tilde a(v, w) := (v, w)_1$ für $v, w\in V$. Dann ist $\tilde a$
    symmetrisch und es gelten $(v, w)_1 \le \|v\|_1 \|w\|_1$ und
    $(v, v)_1 \ge \|v\|_1^2$. Zu $v\in V$ definiere $l_v(w) = a(v, w)$.
    Dann gilt
    \begin{eqnarray*}
        |l_v(w)| \le C(v) \|w\|_1.
    \end{eqnarray*}
    Mit \eqref{satz:3.26} existiert genau ein $u_v\in V$ mit
    \begin{eqnarray*}
        (u_v, w)_1 = \tilde a(u_v, w) = l_v(w)
        \qquad \forall w\in V.
    \end{eqnarray*}
    Definiere nun $Av := u_v\in V$.

    Dann gilt für $\vartheta\in \left(0, \frac{\alpha_0}{2 C_a^2}\right)$
    \begin{eqnarray*}
            \|v - \vartheta Av\|_1^2
        &=& \|v\|_1^2 - 2 \vartheta (v, Av)_1 + \vartheta^2 \|Av\|_1^2 \\
        &=& \|v\|_1^2 - 2\vartheta a(v, v) + \vartheta^2 a(v, Av) \\
        &\le& \|v\|_1^2 - 2 \vartheta \alpha_0 \|v\|_1^2
              + \vartheta^2 C_a^2 \|v\|_1^2 \\
        &\le& \underbrace{(1 - 2\vartheta \alpha_0 + \vartheta^2 C_a^2)}
              _{=: \theta < 1} \|v\|_1^2.
    \end{eqnarray*}
    Damit gilt nun
    \begin{eqnarray*}
        \|id - \vartheta A\|_1 < \theta < 1.
    \end{eqnarray*}
    Die Neumann'sche Reihe liefert uns
    \begin{eqnarray*}
            \left\|\sum_{m=0}^N (id - \vartheta A)^m v\right\|_1
        \le \sum_{m=0}^N \theta^m \|v\|_1
        \le \frac{1}{1 - \theta} \|v\|_1.
    \end{eqnarray*}
    Also ist $u := \vartheta \sum_{m\ge0} (id - \vartheta A)^m v\in V$
    wohldefiniert und es gilt
    \begin{eqnarray*}
        \underbrace{(id - (id - \vartheta A))}_{= \vartheta A}
        \left(\sum_{m\ge0} (id - \vartheta A)^m v\right) = v,
    \end{eqnarray*}
    da nach der Neumann'schen Reihe
    \begin{eqnarray*}
          (id - (id - \vartheta A))
        = \left(\sum_{m\ge 0} (id - \vartheta A)^m\right)^{-1}
    \end{eqnarray*}
    gilt. Damit folgt $Au = v$.

    Zu $l_D(\cdot)$ definiere $v\in V$ mit
    \begin{eqnarray*}
        (v, w)_1 = l_D(w) \qquad \forall w\in V.
    \end{eqnarray*}
    Zu $v\in V$ definieren wir nun wie oben
    \begin{eqnarray*}
        u := \vartheta \sum_{m\ge0} (id - \vartheta A)^m v
    \end{eqnarray*}
    und es folgt $Au = v$.
    
    Somit folgt
    \begin{eqnarray*}
        a(u, v) = (Au, w)_1 = (v, w)_1 = l_D(w)
        \qquad \forall w\in V.
    \end{eqnarray*}
    Angenommen wir haben zwei Lösungen $u, \ \tilde u$ von
    \begin{eqnarray*}
        a(u, v) &=& l_D(v)
        \qquad \forall v\in V \\
        a(\tilde u, v) &=& l_D(v)
        \qquad \forall v\in V.
    \end{eqnarray*}
    Durch Subtraktion beider Gleichungen folgt dann
    \begin{eqnarray*}
        a(u - \tilde u, v) = 0
        \qquad \forall v\in V.
    \end{eqnarray*}
    Wähle nun $v = u - \tilde u$.
    Dann gilt
    \begin{eqnarray*}
        0 = a(u - \tilde u, u - \tilde u) \ge \alpha_0 \|u - \tilde u\|_1.
    \end{eqnarray*}
    und es folgt die Eindeutigkeit $u = \tilde u$.
\end{proof}


\begin{Satz}[Cea's Lemma]
    \label{satz:3.28}
    Sei $a(\cdot, :)$ elliptisch und beschränkt. Dann gilt für die
    Lösung $u\in V$ aus $\eqref{satz:3.27}$ und $u_h\in V_h$ aus
    $\eqref{satz:3.25}$
    \begin{eqnarray*}
        \|u - u_h\|_1 \le \frac{C_a}{\alpha_0} \inf_{v_h\in V_h} \|u - v_h\|_1.
    \end{eqnarray*}
\end{Satz}


\begin{proof}
    Für alle $v_h\in V_h$ gilt $a(u, v_h) = l_D(v_h) = a(u_h, v_h)$.

    Durch Subtraktion beider Seiten folgt die \emph{Galerkin-Orthogonalität}
    $a(u - u_h, v_h) = 0$ und damit
    \begin{eqnarray*}
              \alpha_0 \|u - u_h\|_1^2
        &\le& a(u - u_h, u - u_h) \\
        &=& a(u - u_h, u) \\
        &=& a(u - u_h, u - v_h) \\
        &\le& C_a \|u - u_h\|_1 \|u - v_h\|_1,
    \end{eqnarray*}
    was zu zeigen war.
\end{proof}


\begin{Folgerung}
    \label{folgerung:3.29}
    Für die Lösung $u\in V$ aus $\eqref{satz:3.27}$ gelte zusätzlich
    $u\in H^2(\Omega)$. Dann existiert ein $C > 0$ (unabhängig von h) mit
    \begin{eqnarray*}
        \|u - u_h\|_1 \le C h \|u\|_2.
    \end{eqnarray*}
\end{Folgerung}


\begin{Bemerkung}
    In $C$ geht die Gitterregularität ein.
\end{Bemerkung}


\begin{proof}
    Wähle dazu in \eqref{satz:3.28} $v_h = I_h u \in V_h$.
    Dann folgt
    \begin{eqnarray*}
            \|u - u_h\|_1
        \stackrel{\eqref{satz:3.28}}{\le}
            \frac{C_a}{\alpha_0} \|u - I_hu\|_1
        \stackrel{\eqref{satz:3.17}}{\le}
            \frac{C_a}{\alpha_0} C_I h \|u\|_2.
    \end{eqnarray*}
    die Abschätzung.
\end{proof}


\begin{Definition}
    \label{def:3.30}
    Sei $f\in L_2(\Omega), \ a(\cdot, :)$ elliptisch und beschränkt,
    $u_f\in V$ sei Lösung von
    \begin{eqnarray*}
        a(u_f, v) = (f, v)_0 \qquad \forall v\in V.
    \end{eqnarray*}
    Dann hei\ss{}t dieses Problem \emph{$H^2$-regulär}, wenn zusätzlich
    $u_f \in H^2(\Omega)$ gilt und eine Konstante $C > 0$
    (abhängig von $a(\cdot, :)$ und $\Omega$) existiert mit
    \begin{eqnarray*}
        \|u_f\|_2 \le C \|f\|_0.
    \end{eqnarray*}
%     Das bedeutet, dass der Lösungsoperator $T: L_2(\Omega) \longrightarrow H^2(\Omega)$, $f\mapsto u_f$ stetig ist. 
\end{Definition}


\begin{Beispiel}
    \begin{enumerate}[A)]
        \item
          Falls die Koeffizienten $K(x), \ c(x) , \ q(x)$ glatt und
          $\partial\Omega$ glatt (genug), dann ist $u\in H^s(\Omega)$ für
          $s \ge 2$.
        \item
          Betrachte den Laplace-Operator $Lu = -\Delta u$ und
          $a(u, v) = \int_{\Omega} \nabla u \cdot \nabla v \dx$ mit
          Dirichlet-RB (d.h. $\Gamma_3 = \partial\Omega$).

          Falls $\Omega$ konvex gilt
          \begin{eqnarray*}
              \|\nabla^2 u\|_0 \le \|\Delta u\|_0.
          \end{eqnarray*}
          Dann ist $u\in H^2(\Omega)$ mit
          \begin{eqnarray*}
              \alpha_0 \|u\|_1^2 \le a(u, u) = (f, u)_0 \le \|f\|_0 \|u\|_0
              \le \|f\|_0 \|u\|_1
          \end{eqnarray*}
          und es folgt
          \begin{eqnarray*}
                \|u\|_2^2
              = \|u\|_1^2 + \|\nabla^2 u\|_0^2
              \le \|u\|_1^2 + \|\Delta u\|_0^2
              \le \left(\left(\frac{1}{\alpha}\right)^2 + 1\right) \|f\|_0^2.
          \end{eqnarray*}
        \item
          Betrachte im Lipschitz-Gebiet
          $\Omega = (-1, 1)^2 \setminus \big( [0, 1] \times (-1, 0] \big)$
          mit einspringender Ecke
          \begin{eqnarray*}
              u(x) = \Im z^\frac{2}{3}, \qquad z = x_1 + i x_2 \in \C.
          \end{eqnarray*}
          Dann ist $u$ nicht $H^2$-regulär, denn es gilt
          \begin{eqnarray*}
                  \|\nabla^2 u\|_0^2
              \ge \int_0^{\frac{3}{2}\pi} \int_0^{R(\varphi)}
                  \left|\partial_r^2 u\right|^2 r \,dr \,d\varphi = \infty,
          \end{eqnarray*}
          da
          \begin{eqnarray*}
                \int_0^{R(\varphi)} \left|\partial_r^2 u\right|^2 r \,dr
              = \int_0^{R(\varphi)} r^{-\frac{5}{3}} \,dr
              = \left[-\frac{2}{3} r^{-\frac{2}{3}}\right|_{r=0}^{r=R(\varphi)}
              = \infty.
          \end{eqnarray*}
          Somit ist $\|u\|_2 = \infty$.
    \end{enumerate}
\end{Beispiel}


\begin{Satz}[Aubin/Nitsche]
    \label{satz:3.31}
    Sei $a(\cdot, :)$ elliptisch und beschränkt. Für die Lösung $u\in
    V$ aus \eqref{satz:3.27} gelte $u\in H^2(\Omega)$. Das \emph{adjungierte
    Problem}:
    \begin{eqnarray*}
        \text{finde } w\in V: \qquad a(v, w) = (f, v)_0 \qquad \forall v\in V.
    \end{eqnarray*}
    sei $H^2$-regulär, d.h. für alle $f\in L_2(\Omega)$ sei $w\in
    H^2(\Omega)$ und (unabhängig von f) existiert eine Konstante $C > 0$ mit
    $\|w\|_2 \le C \|f\|_0$.
    Dann gilt
    \begin{eqnarray*}
        \|u - u_h\|_0 \le \tilde C h^2 \|u\|_2
    \end{eqnarray*}
    mit $\tilde C > 0$ nur abhängig von $\Omega$ und der Gitterregularität.
\end{Satz}


\begin{proof}
    Sei
    \begin{eqnarray*}
        u\in V: \quad \qquad a(u, v) &=& l_D(v) \qquad \qquad \ \forall v\in V
        \\
        u_h\in V_h: \qquad a(u_h, v_h) &=& l_D(v_h) \qquad \qquad \forall
                    v_h\in V_h \\
        w\in V: \ \ \ \qquad a(v, w) &=& (u - u_h, v)_0 \qquad \forall v\in V.
    \end{eqnarray*}
    Es gilt
    \begin{eqnarray*}
        \|w\|_2 \le C \|u - u_h\|_0.
    \end{eqnarray*}
    Die Galerkin-Orthogonalität liefert
    \begin{eqnarray*}
            \|u - u_h\|_0^2
        &=& a(u - u_h, w) \\
        &=& a(u - u_h, w - I_h w) \\
        &\le& C_a \|u - u_h\|_1 \|w - I_h w\|_1 \\
        &\stackrel{\substack{\eqref{folgerung:3.29}}{\eqref{satz:3.17}}}{\le}&
              C_a C h \|u\|_2 \tilde C h \|w\|_2 \\
        &\le& \hat C h^2 \|u\|_2 \|u - u_h\|_0
    \end{eqnarray*}
    die Fehlerabschätzung.
\end{proof}