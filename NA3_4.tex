\section{Praktische Aspekte der Finite-Elemente-Methode}

Für realistische Anwendungen wird folgendes benötigt:
\begin{itemize}
    \item
      verschiedene Elementtypen in 2 und 3 Dimensionen
    \item
      nicht nur Polygongebiete
    \item
      verschiedene Diskretisierungen
    \item
      approximative Integralauswertungen
\end{itemize}

Im Folgenden sei $\Omega \subset \R^d$, $(d = 2,3)$ ein Lipschitz-Gebiet.

\begin{Definition}
    \label{def:4.1}
    Eine Triangulierung $\Triangulation$ von $\Omega$ ist eine Zerlegung in endlich viele
    Zellen $K\subset \overline\Omega$ mit
    \begin{enumerate}[i)]
        \item
          $K\in \Triangulation$ abgeschlossen
        \item
          $\meas_d (K) > 0$
        \item
          $\overline\Omega = \bigcup_{K\in \Triangulation} K$
        \item
          $\interior(K) \cap\interior(K^\prime) = \emptyset$ für $K, \
          K^\prime \in \Triangulation$ und $K \neq K^\prime$
    \end{enumerate}
\end{Definition}


\begin{Bemerkung}
    ES gilt $\meas_d(\Omega) = \sum_{K\in \Triangulation} \meas_d(K)$ und wir setzen
    $h = \max\{\diam(K): K\in \Triangulation\}$.
\end{Bemerkung}


\begin{Beispiel}
    \begin{enumerate}[A)]
        \item
          hängende Knoten (nicht zulässig).
        \item
          Gemischte Triangulierungen.
        \item
          Gekrümmter Rand, z.B isoparametrische Elemente.
        \item
          Voronoi.
    \end{enumerate}
\end{Beispiel}


\begin{Definition}
    \label{def:4.2}
    Für eine Finite-Elemente-Triangulierung gilt zusätzlich:
    
    Zu jedem $K\in \Triangulation$ existiert
    \begin{enumerate}[a)]
        \item
          eine Referenzzelle $\hat K$, wobei $\hat K$ ein Dreieck/Viereck
          (in 2-d) \\
          oder Tetraeder/Pyramide/Prisma/Quader/Hexaeder (in 3-d).
        \item
          eine invertierbare, orientierungserhaltende Abbildung $\varphi_K:
          \hat K \to K$, d.h. $F_K(\hat{x}) = \varphi_K'(\hat{x})$ regulär und
          $J_K(\hat{x}) = \det F_K(\hat{x}) > 0$ für alle $\hat x \in \hat K$.
    \end{enumerate}
    Eine Triangulierung heißt \emph{affin}, wenn
    $\varphi_K(\hat x) = \varphi_K(0) + F_K \hat x$ für alle $K\in
    \Triangulation$. Zu $\hat K$ seien $\hat{\mathcal{V}}, \ \hat{\mathcal{E}}, \
    \hat{\mathcal{F}}$ die Ecken/Kanten/Seitenflächen. \\
    Zu $K$ definiere $\mathcal{V}_K = \varphi_K(\hat{\mathcal{V}}), \
    \mathcal{E}_K = \varphi_K(\hat{\mathcal{E}}), \ \mathcal{F}_K =
    \varphi_K(\hat{\mathcal{F}})$.
    Eine Triangulierung heißt \emph{zulässig}, wenn
    $K \cap K^\prime \in \mathcal{V}_K \cap \mathcal{V}_{K^\prime}$ oder
    $\mathcal{E}_K \cap \mathcal{E}_{K^\prime}$ oder
    $\mathcal{F}_K \cap \mathcal{F}_{K^\prime}$ für $K, \ K^\prime \in
    \Triangulation$ und $K \neq K^\prime$.
\end{Definition}

Im Folgenden sei $\hat{X}$ ein Funktionenraum auf der Referenzzelle $\hat{K}$, z.B. $\hat{X} = C(\hat{K})$.

\begin{Definition}
    \label{def:4.3}
    \begin{enumerate}[a)]
        \item
          Ein \emph{Finites-Element} ist ein Tripel
          $(\hat K, \hat V, \hat \Sigma)$
          mit
          \begin{enumerate}[i)]
              \item
                Referenzzelle $\hat K$,
              \item
                Ansatzraum $\hat V = \Span\{\hat{\Phi}_1, \dots, \hat{\Phi}_r\}$
                mit $\hat{\Phi}_i: \hat K \to \R$.
              \item
                Freiheitsgrade $\hat\Sigma = \{\hat{\Phi}_1^\prime, \dots,
                \hat{\Phi}_r^\prime\}$ mit $\hat \Phi_m'(\Phi_m) = 1$ 
                und $\hat \Phi_m'(\Phi_n) = 0$ für $n\neq m$
                mit $\hat{\Phi}_i: \hat X \to \R$ und $\hat V \subset \hat X$.
          \end{enumerate}
        \item
          Der zugehörige \emph{Finite-Elemente-Raum} ist
          \begin{eqnarray*}
              X_h &=& \{v\in X: v \circ \varphi_K \in \hat V\},
          \end{eqnarray*}
          wobei $X$ auf verschiedene Arten gewählt werden kann:
          \begin{eqnarray*}
              X &=& C(\overline\Omega) \qquad \ \ \ C^0-\text{Element} \\
              X &=& C^1(\overline\Omega) \qquad \ C^1-\text{Element} \\
              X &=& H^1(\Omega) \qquad H^1-\text{konformes Element} \\
              X &=& H(\curl,\Omega) = \{u\in L_2(\Omega, \R^2): \curl u \in
                    L_2(\Omega, \R^2)\}.
          \end{eqnarray*}
    \end{enumerate}
\end{Definition}


\subsection{Simpliziale Elemente}


Zu dem Element $K = \conv\{z^0, \dots, z^d\}$ mit
\begin{eqnarray*}
    \varphi_K(e^i) &=& z^i \\
    \varphi_K(\hat x) &=& z^0 + F_K \hat x
    \qquad \text{wobei } F_K = (z^1 - z^0, \dots,  z^d - z^0),
\end{eqnarray*}
ist die Referenzzelle $\hat K = \conv\{0 = e^0, e^1, \dots, e^d\}$ ein Dreieck
oder Tetraeder mit $e^i =$ i-ter Einheitsvektor. 

Zur Vereinfachung der Darstellung verwenden wir Baryzentrische Koordinaten
\begin{eqnarray*}
    \lambda_i: K \to \R \qquad (i = 0, \dots, d)
    \qquad \text{mit } x = \sum_{i=0}^d \lambda_i(x) z^i.
\end{eqnarray*}
Dies ist genau dann der Fall, wenn
\begin{eqnarray*}
    \begin{pmatrix}
        z^0 & \cdots & z^d \\
        1 & \cdots & 1
    \end{pmatrix} \lambda (x)
    =
    \begin{pmatrix}
        x \\
        1
    \end{pmatrix} \in \R^{d+1}
\end{eqnarray*}
gilt und $\lambda(x)$ ist eindeutig bestimmt, falls
\begin{eqnarray*}
      \det \begin{pmatrix}
                z^0 & \cdots & z^d \\
                1 & \cdots & 1
           \end{pmatrix}
    = \det \begin{pmatrix}
                0 & z^1 - z^0 & \cdots & z^d - z^0 \\
                1 & 0 & \cdots & 0
           \end{pmatrix}
    = J_K
    = \meas_d K > 0.
\end{eqnarray*}
Die Cramer'sche Regel liefert uns also
\begin{eqnarray*}
        \lambda_i(x)
    &=& J_K^{-1} \det\begin{pmatrix}
                        z^0 & \cdots & z^{i-1} & x & z^{i+1} & \cdots & z^d \\
                        1 & \cdots & 1 & 1 & 1 & \cdots & 1
                   \end{pmatrix} \\
    &=& J_K^{-1} \meas_d\big(\conv\{z^0, \dots, z^{i-1},x,z^{i+1},\dots, z^d\}\big).
\end{eqnarray*}


\subsubsection{Lineare Elemente in einem Simplex}


Für lineare Elemente sei $\hat K = \{0, e^1, \dots, e^d\}$ ,
$\hat V = \Polynom_1 = \Span\{1, x_1, \dots, x_d\}$
mit der Basis
\begin{eqnarray*}
    \hat \Phi_0 (\hat x) &=& 1 - \sum_{i=1}^d \hat x_i \\
    \hat \Phi_i(\hat x) &=& \hat x_i
    \qquad (i = 1, \dots, d)
\end{eqnarray*}
und $\hat \Sigma = \{\hat{\Phi}_i^\prime: i = 0, \dots, d\}$ mit
$\hat{\Phi}_i^\prime(\hat v) = \hat v(e^i)$.

Dann ist die Basis in $V_K$:
\begin{eqnarray*}
    \hat \Phi_i \circ \varphi_K^{-1} (x) = \lambda_i (x)
    \qquad (i = 0, \dots, d)
\end{eqnarray*}
Für die Freiheitsgrade $\Sigma_K$ gilt:
\begin{eqnarray*}
    \Phi_i^\prime (v) = v(z^i) \qquad (i = 0, \dots, d)
\end{eqnarray*}


\subsubsection{Quadratische Elemente in einem Simplex}


Für quadratische Elemente sei $\hat V = \Polynom_2$, dann
ist $V_K = \Polynom_2 = \Span\{\Phi_i, \Phi_{ij}\}$ mit der Basis
\begin{eqnarray*}
    \Phi_i(x) &=& \lambda_i(x) (2 \lambda_i(x) - 1), \\
    \Phi_{ij}(x) &=& 4 \lambda_i(x) \lambda_j(x), \qquad (i < j)
\end{eqnarray*}
und Freiheitsgrade $\Sigma_K = \{\Phi_i^\prime, \Phi_{ij}^\prime\}$ mit
\begin{eqnarray*}
    \Phi_j^\prime(v) &=& v(z^i), \\
    \Phi_{ij}^\prime(v) &=& v\left(\frac{1}{2}(z^i + z^j)\right).
\end{eqnarray*}
Eine Alternative ist die hierarchische Darstellung:
\begin{eqnarray*}
        v^{(1)}
    &=& \sum_{i=0}^d v(z^i) \lambda_i(x) \in \Polynom_1, \\
        v^{(2)}
    &=& v^{(1)} + \sum_{i<j} (v - v^{(1)}) \left(\frac{1}{2}(z^i + z^j)\right)
        \Phi_{ij}(x).
\end{eqnarray*}


\subsubsection{Kubische Elemente in einem Simplex}


Für kubische Elemente sei $V_K = \Polynom_3$ und
$\Sigma_K = \{\Phi_i^\prime, \Phi_{ij}^\prime, \Phi_{ijk}^\prime\}$ mit
\begin{eqnarray*}
        \Phi_i^\prime(v)
    &=& v(z^i) \qquad (i = 0, \dots, d) \\
        \Phi_{ij}^\prime(v)
    &=& v\left(\frac{2}{3} z^i + \frac{1}{3} z^j\right)
        \qquad (0 \le i < j \le d) \\
        \Phi_{ijk}^\prime(v) &=& v\left(\frac{1}{3}(z^i + z^j + z^k)\right)
        \qquad (0 \le i < j < k \le d).
\end{eqnarray*}
Die Anzahl der Freiheitsgrade ist $\#\Sigma_K = \frac{(d + 1)(d + 2)}{2}$.

Die Basis in $V_K$ ist:
\begin{eqnarray*}
        \Phi_i(x)
    &=& \frac{1}{2} \lambda_i(x) (3 \lambda_i(x) - 1)(3 \lambda_i(x) - 2)
        \qquad (i  = 1, \dots, d) \\
        \Phi_{ij}(x)
    &=& \frac{9}{2} \lambda_i(x) \lambda_j(x) (3 \lambda_i(x) -1)
        \qquad (i, j = 0, \dots, d) \\
        \Phi_{ijk}(x)
    &=& 27 \lambda_i(x) \lambda_j(x) \lambda_k(x)
        \qquad (0 \le i < j <k \le d).
\end{eqnarray*}


\begin{Definition}
    \label{def:4.4}
    \begin{enumerate}[a)]
        \item
          Ein Finites Element $(K, V_K, \Sigma_K)$ hei\ss{}t
          \emph{Lagrange-Element}, wenn $X = C(\overline\Omega)$ und wenn
          $\Sigma_K = \{\Phi_i^\prime\}$ nur aus Punktauswertungen
          $\Phi_i^\prime(v) = v(y^i)$ besteht ($y^i$ hei\ss{}en
          \emph{Knotenpunkte}). $I_K: C(K) \to V_K, \ v \mapsto  I_K v(x) =
          \sum_{i=1}^M v(y^i) \Phi_i$ hei\ss{}t \emph{Lagrange-Interpolation}.
        \item
          Eine Familie von Lagrange-Elementen hei\ss{}t \emph{äquivalent},
          wenn ein gemeinsames Referenz-Element $(\hat K, \hat V, \hat\Sigma)$
          existiert mit
          $K = \varphi_K(\hat K), \ V_K = \{v\in C(K): v \circ \varphi_K\in \hat
          V\}$ und $\Sigma_K = \{\Phi_i^\prime\}$ mit $\Phi_i^\prime(v) =
          v(\varphi_K(\hat y^i))$.
          Sie hei\ss{}en \emph{affin äquivalent}, wenn $\varphi_K$ linear
          affin ist.
    \end{enumerate}
\end{Definition}


\begin{Bemerkung}
    Die Punkte $\{y^i\}$ hei\ss{}en \emph{unisolvent}, wenn die
    Lagrange-Interpolationsaufgabe:
    \begin{eqnarray*}
        \text{finde } v_h\in V_K \text{ mit } v_h(y^i) = v(y^i)
    \end{eqnarray*}
    eindeutig lösbar ist.
\end{Bemerkung}


\begin{Satz}
    \label{satz:4.5}
    Die Einbettung von $H^{k+1}(\Omega) \subset H^k(\Omega)$ ist kompakt, d.h.
    jede beschränkte Teilfolge \\
    $(v^n)_{n\in\N}$ in $H^{k+1}(\Omega)$ besitzt eine konvergente Teilfolge
    $(v^{n_j})_{j\in\N}$ in $H^k(\Omega)$.
\end{Satz}


\begin{proof}
    Ohne Beweis.
\end{proof}


\begin{Satz}[Bramble-Hilbert-Lemma]
    \label{satz:4.6}
    Sei $I_K: C(K) \to H^{k+1}(K)$ eine Lagrange-Interpolation mit $I_K(p) = p$
    für $p\in \Polynom_k$ ($k \ge 1$).
    Dann existiert eine Konstante $C > 0$ mit
    \begin{eqnarray*}
        \|I_K v - v\|_{k+1} \le C |I_K v - v|_{k+1}
        \qquad \forall v\in H^{k+1}(K).
    \end{eqnarray*}
    Dabei ist $|v|_k = \left(\sum_{|\alpha| = k} \|\partial^\alpha v\|_0^2
    \right)^\frac{1}{2}$ Seminorm in $H^k(\Omega)$ mit Multiindex
    $\alpha\in \N_0^d, \ |\alpha| = \alpha_1 + \dots + \alpha_d$ und
    $\partial^\alpha v = \partial_1^\alpha \cdots \partial_d^\alpha v$.
\end{Satz}


\begin{Bemerkung}
    Wenn $I_K(C(K)) = \Polynom_k$, dann ist $|I_K v|_{k+1} = 0$. 
\end{Bemerkung}


\begin{Definition}
    Definiere
    $H^m(\Omega) = \{v\in H^{m-1}(\Omega): \partial_i v\in H^{m-1}(\Omega)\}$
    mit Skalarprodukt \\
    $(v, w)_m = (v, w)_0 + \sum_{i=1}^d (\partial_i v, \partial_iw)_0$ und Norm
    $\|v\|_m = \sqrt{(v, v)_m}$.
\end{Definition}


\begin{proof}
    Definiere $|\|v|\| = \sum_{i=1}^3 |v(y^i)| + |v|_{k+1}$.

    Behauptung: $\|v\|_{k+1} \le C |\|v|\|$
    (d.h. $\|I_K v - v\|_{k+1} \le C |I_K v - v|_{k+1}$).

    Annahme: Es existiert $v^n\in H^{k+1}(K)$ mit $|\|v^n|\| \le \frac{1}{n}$
    und $\|v^n\|_{k+1} = 1$.

    Dann existiert eine Teilfolge $(w^j = v ^{n_j})_{j\in\N}$ und
    $w^*\in H^k(K)$ mit $\|w^j - w^*\|_k \to 0$ für $ j \to \infty$.
    Also ist $w^j$ Cauchy-Folge in $H^k(K)$ und
    $|w^j|_{k+1} \le |\|w^j|\| \le \frac{1}{n_j}$.
    Damit gilt folgende Abschätzung
    \begin{eqnarray*}
        \|w^j - w^i\|_{k+1} \le \|w^j - w^i\|_k + \frac{1}{n_j} +\frac{1}{n_i},
    \end{eqnarray*}
    Diese liefert uns, dass $(w^j)_j$ Cauchy-Folge in $H^{k+1}(K)$ ist.
    Somit gilt $w^j \to w^*\in H^{k+1}(\Omega)$ mit $\|w^*\|_{k+1} = 1$, also
    $w^* \neq 0$, und $|w^*|_{k+1} = 0$, da $|\|w^*|\| = 0$.
    Dann ist $w^*\in \Polynom_k$.

    Au\ss{}erdem ist $w^*\in C(K)$ mit $w^*(y^j) = 0$.
    Da $y^j$ unisolvent ist die Lagrange-Interpolationsaufgabe eindeutig
    lösbar und somit $w^* = 0$.
    Was ein Wiederspruch ist.
\end{proof}


\subsubsection{Tensor-Produkt-Element}



Sei $\hat K = [0, 1]^d$ und $\hat V = \Polynom_k(\R)^d = \Polynom_k^d = \TensorElem_k =
\Span\{p: p(x) = \Pi_{i=1}^d p_i(x_i) \text{ mit } p_i\in \Polynom_k\}$.
Dann ist $\dim \hat V = (k + 1)^d$.
Weiter seien die Freiheitsgrade $\hat \Sigma = \{\hat{\Phi}_\alpha^\prime:
 \hat{\Phi}_\alpha^\prime(v) = v(\hat{y}^\alpha), \ \alpha\in \N_0^d, \ \alpha_i
\le k\}$ mit $\hat{y}^\alpha = \frac{1}{k} (\alpha_1, \dots, \alpha_d)$.

Die Basisfunktionen sind Produkte von Lagrange-Basispolynomen in $\R$.

Zu $k \ge 1$ definiere $t_i = \frac{i}{k} \ (i = 0, \dots, k)$ und
\begin{eqnarray*}
    L_i^k(t) = \Pi_{j=0,j\neq i} \frac{t - t_j}{t_i - t_j}
    \qquad \text{mit }
    L_i^k(t_i) = \begin{cases}
                      1, \qquad i = j \\
                      0, \qquad i\neq j
                 \end{cases}.
\end{eqnarray*}
Die Basisfunktionen sind definiert durch
$\hat{\Phi}_\alpha(x) = \Pi_{i=1}^d L_{\alpha_i}^k(x_i)$
und $\hat{\Phi}_\alpha\in \Polynom_k^d \subset \Polynom_{dk}(\R^d)$,
wobei
$\hat{\Phi}_\alpha(\hat{y}^\beta) = 0$ für $\beta \neq 0$.


\begin{Beispiel}[Bilineare Elemente]
    Für $d = 2$ und $k = 1$ seien die Basisifunktionen
    \begin{eqnarray*}
            \hat \Phi_{00}(\hat x)
        &=& (1 - \hat x_1) (1 - \hat x_2) \\
            \hat\Phi_{10}(\hat x)
        &=& \hat x_1 (1 - \hat x_2) \\
            \hat \Phi_{01}(\hat x)
        &=& (1 - \hat x_1) \hat x_2 \\
            \hat \Phi_{11}(\hat x)
        &=& \hat x_1 \hat x_2
    \end{eqnarray*}
\end{Beispiel}


\subsubsection{Isoparametrische Elemente}



\begin{Definition}
    \label{def:4.7}
    Ein Finites Element hei\ss{}t \emph{isoparametrisch}, wenn
    $\varphi_K\in \hat V^d$ gilt.
\end{Definition}


\begin{Beispiel}[Isoparametrische bilineare Elemente]
    Sei $\hat K = \conv\{\left(\begin{smallmatrix}
                                    0 \\ 0
                               \end{smallmatrix}\right)
                         \left(\begin{smallmatrix}
                                    1 \\ 0
                               \end{smallmatrix}\right)
                         \left(\begin{smallmatrix}
                                    0 \\ 1
                               \end{smallmatrix}\right)
                         \left(\begin{smallmatrix}    
                                    1 \\ 1
                               \end{smallmatrix}\right)\}$
    transformiert auf $K = \{z^{00}, \ z^{10}, \ z^{01}, \ z^{11}\}$.

    Definiere
    \begin{eqnarray*}
            \varphi_K(\hat x)
        &=& (1 - \hat x_1)(1 - \hat x_2) z^{00} + \hat x_1 (1 - \hat x_2) z^{10}
            \\
            &&+ (1 - \hat x_1) \hat x_2 z^{01} + \hat x_1 \hat x_2 z^{11} \in K
            \qquad \text{für } \hat x \in \hat K.
    \end{eqnarray*}
    Dann ist
    \begin{eqnarray*}
            F_K(\hat x)
        &=& ((1 - \hat x_2)(z^{10} - z^{00}) + \hat x_2 (z^{11} - z^{01}), \\
            &&(1 - \hat x_1)(z^{01} - z^{00}) + \hat x_1 (z^{11} - z^{10})
        \in \R^{2,2}
    \end{eqnarray*}
    und es gilt
    \begin{enumerate}[1)]
        \item 
          $\varphi_K$ linear affin genau dann wenn
          $z^{11} = z^{10} + (z^{01} - z^{00})$.
        \item
          $K$ konvex und es gilt
          \begin{eqnarray*}
              \det (z^{00} | z^{10} | z^{01}) &>& 0 \\
              \det (z^{10} | z^{11} | z^{01}) &>& 0.
          \end{eqnarray*}
          Somit ist $F_K(\hat x)$ invertierbar und $\varphi_K$
          orientierungserhaltend (d.h. $\det F_K(\hat x) = J_K(\hat x) >0$).
        \item
          Im Allgemeinen sind
          $V_K = \{v\in C(K): v \circ \varphi_K\in \hat V\} \not\subset
          \Polynom_{dk}(\R^3)$!
        \item
          Warnung:
          Einige Vierecke konvergieren gegen ein Dreieck. Also $F_K$ fast
          singulär und deshalb keine Finite-Elemente-Konvergenz!
    \end{enumerate}
\end{Beispiel}


\begin{Lemma}
    \label{lem:4.8}
    Für isoparametrische Lagrange-Elemente gilt
    $\varphi_K(\hat x) = \sum_{j=1}^M \hat \Phi_j(\hat x) y^j$ mit
    $\Sigma_K = \{\Phi_j^\prime: \Phi_j^\prime(v) = v(y^j)\}$.
\end{Lemma}


\begin{Beispiel}
    Isoparametrische quadratische Dreiecke ($d = 2$).
\end{Beispiel}


\begin{Bemerkung}
    Wenn $\partial K$ einen glatten Rand $\partial\Omega$ approximiert, dann
    gilt:
    \begin{eqnarray*}
        \left|z^{ij} - \frac{1}{2} (z^i + z^j)\right| \le C h^2
    \end{eqnarray*}
    und für die Lagrange-Interpolation gilt:
    \begin{eqnarray*}
            \|v - I_h v\|_{m, \Omega \cap \Omega_h}
        \le C h^{3 - m} \|v\|_{m, \Omega}
        \qquad \text{für } v\in H^3(\Omega)
    \end{eqnarray*}
    mit $\Omega_h = \bigcup_{K\in\Triangulation} K$.
\end{Bemerkung}


\paragraph{Babushka-Paradoxon:}



Für festes $h$, immer grö\ss{}eres $k$ und immer feinere
Polygon-Approximation von $\Omega$ gibt es keine Finite-Elemente-Konvergenz!



\subsection{Der Konsistenzfehler und Kubatur}



\begin{Definition}
    \label{def:4.9}
    Eine Kubatur $Q_\varXi$ wird durch endlich viele Punkte $\varXi \subset \hat
    K$ und Gewichte $\omega_\xi$ ($\xi\in \varXi$) bestimmt mit
    $Q_\varXi(v) = \sum_{\xi\in \varXi} \omega_\xi v(\xi)$.
    Der Kubaturfehler sei $E_\varXi(v) = \int_{\hat K} v \,d\hat x -
    Q_\varXi(v)$. Die Genauigkeit der Kubatur ist das grö\ss{}te $k\in\N_0$ mit
    $E_\varXi(p) = 0$ für alle $p\in \Polynom_k(\R^d)$.
\end{Definition}


\begin{Beispiel}
    Für $d = 1, \ \hat K = [0, 1]$ und der Gau\ss{}-Quadratur gilt für:

    k=1:
    \begin{eqnarray*}
          \int_0^1 p(t) \,dt
        = p\left(\frac{1}{2}\right),
        \qquad p\in \Polynom_1(\R)
    \end{eqnarray*}
    k=3:
    \begin{eqnarray*}
          \int_0^1 p(t) \,dt
        = \frac{1}{2} (p(\alpha) + p(1 - \alpha)),
        \qquad p\in \Polynom_3(\R)
        \text{ mit } \alpha = \frac{1}{6} (3 - \sqrt{3})
    \end{eqnarray*}
    k=5:
    \begin{eqnarray*}
          \int_0^1 p(t) \,dt
        = \frac{1}{18} \left(5 p(\beta) + 8 p\left(\frac{1}{2}\right)
          + 5 p( 1 - \beta)\right), \\
        \qquad p\in \Polynom_5(\R)
        \text{ mit } \beta = \frac{1}{2} \left(1 - \sqrt{\frac{3}{5}}\right)
    \end{eqnarray*}
    Für $d = 2$, Referenzzelle $\hat K = [0, 1]^2$ und
    $\int_0^1 p(t) \,dt = \sum_{\xi\in \varXi} \omega_\xi p(\xi)$
    folgt
    \begin{eqnarray*}
          \int_0^1 \int_0^1 p(\hat x) \,d \hat x
        = \sum_{\xi_1\in \varXi} \sum_{\xi_2\in \varXi} \omega_{\xi_1}
          \omega_{\xi_2} p(\xi_1, \xi_2)
        \qquad p\in \Polynom_k^d.
    \end{eqnarray*}
    Analog für $d = 3$ und $\hat K = [0, 1]^3$.
    
    Für $d = 2$ und $\hat K = \conv\{0, e^1, e^2\}$ gilt für

    k=1:
    \begin{eqnarray*}
            \int_{\hat K} p(\hat x) \,d\hat x
          = \frac{1}{2} p\left(\frac{1}{3}, \frac{1}{3}\right),
          \qquad p\in \Polynom_1(\R^2)
    \end{eqnarray*}
    k=2:
    \begin{eqnarray*}
            \int_{\hat K} p(\hat x) \,d\hat x
          = \frac{1}{6} \left(p\left(\frac{1}{6}, \frac{1}{6}\right) +
            p\left(\frac{2}{3}, \frac{1}{6}\right) +
            p\left(\frac{1}{6}, \frac{2}{3}\right)\right),
          \qquad p\in \Polynom_2(\R^2).
    \end{eqnarray*}
    Für $d = 3$ und $\hat K = \conv\{0, e^1, e^2, e^3\}$ gilt für

    k=1:
    \begin{eqnarray*}
          \int_{\hat K} p(\hat x) \,d\hat x
        = \frac{1}{6} p\left(\frac{1}{4}, \frac{1}{4}, \frac{1}{4}\right),
        \qquad p\in \Polynom_1(\R^3)
    \end{eqnarray*}
    k=2:
    \begin{eqnarray*}
          \int_{\hat K} p(\hat x) \,d\hat x
        = \frac{1}{24} (p(\alpha, \alpha, \alpha) + p(\alpha, \alpha, \beta)
          + p(\alpha, \beta, \alpha) + p(\beta, \alpha, \alpha)), \\
        \qquad p\in \Polynom_2(\R^3)
        \text{ mit } \alpha = \frac{1}{20} (5 - \sqrt{5}), \beta = \frac{1}{20}
        (5 - 3 \sqrt{5}).
    \end{eqnarray*}
    Für $d = 2$, der Referenzzelle $\hat K = \conv\{0, e^1, e^2\}$ und
    $\int_0^1 (1 - t) p(t) = \sum_{\xi\in \varXi} \tilde \omega_\xi p(\xi)$
    folgt
    \begin{eqnarray*}
          \int_{\hat K} p(\hat x) \,d\hat x
        = \int_0^1 \int_0^{1 - \hat x_1} p(\hat x) \,d\hat x_2 \,d\hat x_1
        = \sum_{\xi\in \varXi} \sum_{\nu\in \tilde \varXi} \omega_\xi \omega_\nu
          p(\xi, (1 - \xi) \nu).
    \end{eqnarray*}
\end{Beispiel}


\subsection{Assemblierung}



Sei $\int_{\hat K} \hat v(\hat x) \,d\hat x = \sum_{\xi\in \varXi} \omega_\xi
\hat v(\xi) + \hat E(v)$.
Dann wird
\begin{eqnarray*}
        a(u, v)
    &=& \int_\Omega K \nabla u \cdot \nabla v \dx \\
        l(v)
    &=& \int_\Omega f v \dx
\end{eqnarray*}
durch
\begin{eqnarray*}
        a_h(u_h, v_h)
    &=& \sum_{K\in\Triangulation} \sum_{\xi\in \varXi} \omega_\xi J_K(\xi)
        K(\varphi_K(\xi)) F_K^{-T}(\xi) \hat \nabla (u_h \circ \varphi_K)(\xi)
        F_K^{-T}(\xi) \hat\nabla (v_h \circ \varphi_K)(\xi) \\
        l_h(v_h)
    &=& \sum_{K\in\Triangulation} \sum_{\xi\in \varXi} f(\varphi_k(\xi))
        v(\varphi_K(\xi)) J_K(\xi)
\end{eqnarray*}
approximiert.

Der folgende Satz ist eine Verallgemeinerung des Lemmas von Cea \eqref{satz:3.28}, die neben dem
Interpolationsfehler auch die durch die Approximation von $a$ und $l$ entstehenden Konsistenzfehler berücksichtigt.
\begin{Lemma}[1. von Strang]
    \label{lem:4.10}
    Sei $u\in V$ mit $a(u,v) = \ell(v)$ für alle $v\in V$, und sei $u_h\in V_h$
    mit $a_h(u_h,v_h) = \ell_h(v_h)$ für alle $v_h\in V_h$.
    Wenn $a(u,v) \le C_a \| u\|_1\|v\|_1 $ und $a_h(v_h,v_h) \ge \alpha_0 
    \|v_h\|_1^2$ für alle $h\in (0,h_0)$ gilt, dann existiert $C>0$ mit
    \begin{eqnarray*}
              \|u - u_h\|_1
        &\le& C \inf_{v_h\in V_h}\Big(\| u - v_h\|_1
              + \sup_{\|w_h\|_1=1} |a(v_h, w_h) - a_h(v_h, w_h)| \Big)\\
              &&+ C\sup_{\|\tilde w_h\|_1 = 1} |l(\tilde w_h) - l_h(\tilde w_h)|.
    \end{eqnarray*}
\end{Lemma}


\begin{proof}
    Für alle $v_h\in V_h$ und $v_h \neq u_h$ gilt
    \begin{eqnarray*}
              \alpha_0 \|u_h - v_h\|_1^2
        &\le& a_h(u_h - v_h, u_h - v_h) \\
        &=& a(u - v_h, u_h - v_h) + a(v_h, u_h - v_h) - a(u, u_h - v_h) \\
          &&+ a_h(u_h, u_h - v_h) - a_h(v_h, u_h - v_h) \\
        &=& a(u - v_h, u_h - v_h) + a(v_h, u_h - v_h) - a_h(v_h, u_h - v_h) \\
          &&+ l_h(u_h - v_h) - l(u_h - v_h) \\
        &\le& C_a \|u - v_h\|_1 \|u_h - v_h\|_1 \\
          &&+ \frac{|a(v_h, u_h - v_h) - a_h(v_h, u_h - v_h)|}{\|u_h - v_h\|_1}
              \|u_h - v_h\|_1 \\
          &&+ \frac{|l_h(u_h - v_h) - l(u_h - v_h)|}{\|u_h - v_h\|_1}
              \|u_h - v_h\|_1.
    \end{eqnarray*}
    Dann ist
    \begin{eqnarray*}
              \alpha_0 \|u_h - v_h\|_1
        &\le& C_a \|u - v_h\|_1 +
              \sup_{\|w_h\|_1 = 1} |a_h(v_h, w_h) - a(v_h, w_h)| \\
          &&+ \sup_{\|\tilde w_h\|_1 = 1} |l_h(\tilde w_h) - l(\tilde w_h)|
    \end{eqnarray*}
    und es folgt mit
    \begin{eqnarray*}
              \|u - u_h\|_1
        &\le& \|u - v_h\|_1 + \|u_h - v_h\|_1 \\
        &\le& \left(1 + \frac{C_a}{\alpha_0}\right) \|u - v_h\|_1
              + \frac{1}{\alpha_0} \Biggl(\sup_{\|w_h\|_1 = 1}
              |a_h(v_h, w_h) - a(v_h, w_h)| \\
          &&+ \sup_{\|\tilde w_h\|_1 = 1} |l_h(\tilde w_h) - l(\tilde w_h)|
              \Biggr)
    \end{eqnarray*}
    die Fehlerabschätzung.
\end{proof}


\begin{Satz}[Transformationsformel] TODO: korrigieren
    \label{satz:4.11}
    Für $|\alpha| =m$, $v\in H^m(\Omega)$ und 
    $\hat v = v \circ \varphi_K$ gilt
        \begin{eqnarray*}
                \min_{\hat x\in\hat K} |F_K(\hat x)^{-1}|^m J_K(\hat x)^{1/2}
                \|\partial^\alpha v\|_{0,K}
            \le \|\hat \partial^\alpha \hat v\|_{0,\hat K}
            \le \max_{\hat x\in\hat K} |F_K(\hat x)|^m J_K(\hat x)^{-1/2}
                \|\partial^\alpha v\|_{0,K}.
        \end{eqnarray*}
\end{Satz}


\begin{proof}
    Ohne Beweis!
\end{proof}


\begin{Definition}
    \label{def:4.12}
    Eine Familie $(\mathcal T_h)_h$ von Triangulierungen heißt 
    \begin{enumerate}[a)]
      \item
        regulär, wenn $\min |F_K(\hat x)^{-1}| \max |F_K(\hat x)|\leq C$
        unabhängig von $K\in \mathcal T_h$ und $h$.
      \item 
        uniform, wenn $C\geq c>0$ existiert mit
        $ch\leq (\min |F_K(\hat x)^{-1}|)^{-1}\leq \max |F_K(\hat x)|\leq Ch$.
    \end{enumerate}
\end{Definition}


\begin{Lemma}
    \label{lem:4.13}
    Sei $\hat E(p) = 0$ für $p\in \mathbb P_{2k-2}$. Dann gilt für alle affine
    Elemente $K\in \Triangulation, p,q\in \mathbb P_{k-1}$ und $a\in C^k(K)$
    \begin{eqnarray*}
        E_K(apq) \le C h^k \|a\|_{k,\infty,K}
        \| p \|_{k-1} \| q \|_{0},
    \end{eqnarray*}
    mit $E_K(v) = J_K \, \hat E (v \circ \varphi_K)$ und
    $\|a\|_{k,\infty,K} = \sup_{\alpha_i \le k} \|\partial^\alpha a\|_\infty$.
    Die Konstante $C$ hängt von $\hat K$ und von der Gitterregularität ab.
\end{Lemma}


\begin{proof}
    Für $k = 1$:
    Sei ohne Einschränkung $\#\varXi = 1$ und $x_K = \varphi_K(\xi)$.
    Dann gilt
    \begin{eqnarray*}
            a(x) - a(x_K)
        &=& \int_0^1 \frac{d}{dt} a(x_K + t(x - x_K)) \,dt \\
        &=& \int_0^1 \nabla a(x_k + t(x - x_K)) (x - x_K) \,dt \\
        &\le& \|a\|_{1,\infty,K} h.
    \end{eqnarray*}
    Für den Fehler gilt dann
    \begin{eqnarray*}
            E_K(apq)
        &=& \int_K apq \dx + J_K \meas_d(\hat K) a(x_K) pq \\
        &=& \int_K pq (a(x) - a(x_K)) \dx \\
        &\le& h \|a\|_{1,\infty,K} \|p\|_{0,K} \|q\|_{0,K} \\
    \end{eqnarray*}
    Für $k > 1$:
    Sei $\hat v\in H^k(\interior(\hat K))$ und
    $\hat q\in \Polynom_{k-1}(\hat K)$.
    Dann gilt
    \begin{eqnarray*}
            |\hat E(\hat v \hat q)|
        &=& \left|\int_{\hat K} \hat v \hat q \,d\hat x - \sum_{\xi\in \varXi}
            \omega_\xi \hat v(\xi) \hat q(\xi)\right| \\
        &\le& C \|\hat v\|_{\infty,\hat K} \|\hat q\|_{\infty,\hat K} \\
        &\le& \tilde C \|\hat v\|_{k,\hat K} \|\hat q\|_{0,\hat K},
    \end{eqnarray*}
    denn für $k > 1$ gilt
    \begin{eqnarray*}
            \|\hat v\|_{\infty,\hat K}
        \le C \|\hat v\|_{2,\hat K}
        \le C \|\hat v\|_{k,\hat K}
    \end{eqnarray*}
    und da $\Polynom_{k-1}$ endlich-dimensional ist, existieren nach dem
    Normäquivalenzsatz Konstanten $C_0, \ C_1$  mit
    \begin{eqnarray*}
            C_0 \|\hat q\|_{0,\hat K}
        \le \|\hat q\|_{\infty,\hat K}
        \le C_1 \|\hat q\|_{0,\hat K}
        \qquad \forall \hat q\in \Polynom_{k-1}.
    \end{eqnarray*}
    Für festes $\hat q$ definiere $G(\hat v) := \hat E(\hat v \hat q)$.

    Damit folgt
    $|G(\hat v)| \le C(\hat q) \|\hat v\|_{k,\hat K}$ und
    $G(\hat p) = 0$ für $\hat p\in \Polynom_{k-1}$.
    Bramble-Hilbert liefert
    \begin{eqnarray*}
            |G(\hat v)|
        &=& |G(\hat v - I_{k-1} \hat v)| \\
        &\le& C(\hat q) \|\hat v - I_{k-1} \hat v\|_{k,\hat K} \\
        &\le& \tilde C C(\hat q) |\hat v|_{k,\hat K}
    \end{eqnarray*}
    für eine Lagrange-Interpolation $I_{k-1}: C(\hat K) \to \Polynom_{k-1}$.
    Sei nun $\hat v = \hat a \hat p$ mit $\hat a = a \circ \varphi_K$.
    Dann ist
    \begin{eqnarray*}
            |\hat E(\hat v \hat q)|
        \le C \|\hat q\|_{0,\hat K} \sum_{l=0}^k |\hat a|_{l,\infty,\hat K}
              |\hat p|_{k-l,\hat K}
    \end{eqnarray*}
    und es folgt
    \begin{eqnarray*}
          |E_K(apq)|
        = |J_K| |\hat E((apq) \circ \varphi_K)|
    \end{eqnarray*}
    Nach \eqref{satz:4.11} und \eqref{def:4.12} gilt:
    \begin{eqnarray*}
              |\hat p|_{k-l,\hat K}
        &\le& C h^{k-l} J_K^{-\frac{1}{2}} |p|_{k-l,K} \\
              \|\hat q\|_{0,\hat K}
        &\le& C J_K^{-\frac{1}{2}} |q|_{0,K} \\
              |\hat a|_{l,\infty,\hat K}
        &\le& C h^l |a|_{l,\infty,K}.
    \end{eqnarray*}
    Wir erhalten damit den Quadraturfehler.
\end{proof}


\begin{Satz}
    \label{satz:4.14}
    Sei $I_{\hat K}(p) = p$ für $p\in \Polynom_k$. Dann gilt für uniforme
    Triangulierungen und $m = 0, \dots, k$ 
    \begin{eqnarray*}
            \|I_K v - v\|_m
        \le C h^{k+1-m} |v|_{k+1} 
        \quad \text{ für alle } v\in H^{k+1}(\Omega).
    \end{eqnarray*}
\end{Satz}


\begin{proof}
    Ohne Beweis!
\end{proof}


\begin{Satz}
    \label{satz:4.15}
    Sei $(\Triangulation)_h$ eine reguläre Familie von affin äquivalenten
    Lagrange-Elementen in $\R^d$ ($d = 2, 3$) mit $V_K \subset \Polynom_k(\R^d)$
    und $a_h(\cdot, :)$ sei durch eine Quadratur mit $\tilde E(p) = 0$ für
    $p\in \Polynom_{2k - 2}$ definiert.
    Es seien $u\in V$, $u_h\in V_h$ wie in \eqref{lem:4.10} und es gelte
    $u\in H^{k+1}(\Omega)$. Dann  gilt
    \begin{eqnarray*}
        \|u - u_h\|_1 \le C h^k |u|_{k+1},
    \end{eqnarray*}
    wobei $C > 0$ von der Gitterregularität und von
    $\|K_{ij}\|_{k,\infty,\Omega}$ und $\|f\|_{k,\infty,\Omega}$ abhängt.
\end{Satz}


\begin{proof}
    Es gilt
    \begin{eqnarray*}
              |a(v_h, w_h) - a_h(v_h, w_h)|
        &\le& \sum_{K\in\Triangulation} E_K(K \nabla v_h \cdot \nabla w_h) \\
        &\stackrel{\eqref{lem:4.13}}{\le}&
              \max_{i,j} C_{a,k} h^k \|K_{ij}\|_{k,\infty,\Omega}
              \sum_{K\in\Triangulation} \|\nabla v_h\|_{k-1,K}
              \|\nabla w_h\|_{0,K} \\
        &\le& C_{a,k} h^k \max_{i,j} \|K_{ij}\|_{k,\infty,\Omega}
              \left(\sum_{K\in\Triangulation} \|v_h\|_{k,K}^2
              \right)^\frac{1}{2}
              \left(\sum_{K\in\Triangulation} \|w_h\|_{1,K}^2\right)^\frac{1}{2}
              \\
        &\le& C_{a,k} h^k \max_{i,j} \|K_{ij}\|_{k,\infty,\Omega}
              \|v_h\|_{k,\Omega} \|w_h\|_{1,\Omega}.
    \end{eqnarray*}
    Analog folgt $|l(v_h) - l_h(v_h)| \le C_l h^k \|f\|_{k,\infty,\Omega} \|v_h\|_k$
    und für $k = 1$ gilt insbesondere
    \begin{eqnarray*}
            |a(v_h, w_h) - a_h(v_h, w_h)|
        \le C_{a,1} h \|v_h\|_{1,\Omega} \|w_h\|_{1,\Omega}
    \end{eqnarray*}
    und es folgt
    \begin{eqnarray*}
            |a_h(v_h, v_h)|
        &=& |a_h(v_h, v_h) + a(v_h, v_h) - a(v_h, v_h)| \\
        &\ge& (\alpha_0 - C_{a,1} h) \|v_h\|_1^2,
    \end{eqnarray*}
    dass $a_h(\cdot, :)$ elliptisch ist für $h < \frac{\alpha_0}{C_1}$.
    Mit \eqref{lem:4.10} und $v_h = I_h u$ folgt
    \begin{eqnarray*}
              \|u - u_h\|_1
        &\le& C (\|u - I_h u\|_1 + \tilde C h^k \|I_h u\|_k) \\
        &\stackrel{\eqref{satz:4.14}}{\le}&
              C h^k |u|_{k+1} + \tilde C h^k (\|u\|_k + \|u - I_h u\|_k) \\
        &\stackrel{\eqref{satz:4.6}}{\le}&
              \hat C h^k |u|_{k+1}
    \end{eqnarray*}
    die Abschätzung.
\end{proof}


\subsection{Gittergenerierung und a-posteriori-Fehlerschätzer}



Numerische Simulation erfordert:

\begin{itemize}
    \item
      Preprocessing:
      \begin{itemize}
          \item
            Geometrie-Erzeugung
          \item
            Gitter- (Netz-)generierung
      \end{itemize}
    \item
      Rechenkern:
      \begin{itemize}
          \item
            Assemblierung
          \item
            Lösen
      \end{itemize}
    \item
      Postprocessing:
      \begin{itemize}
          \item
            Visualisierung
          \item
            Fehlerschätzung
      \end{itemize}
\end{itemize}


\subsubsection{Techniken der Gittergenerierung}


\begin{enumerate}[A)]
    \item
      Overlay-Methoden
      \begin{itemize}
          \item
            Auflösen der Geometrie durch Quad-/Oct-tree
            \begin{itemize}
                \item
                  Start: Bounding-Box $B \supset \Omega$
                \item
                  Rekursive Zerlegung der Box $B$ falls $\partial\Omega \neq
                  \emptyset$
            \end{itemize}
          \item
            Strukturiertes Gitter im Inneren
          \item
            Randanpassung an $\partial\Omega$
      \end{itemize}
    \item
      Delaunay Triangulierung
      \begin{itemize}
          \item
            Unterteilung von $\Omega$ in konvexe Teilgebiete $\Omega_j$
          \item
            Verteilung von Punkten $\{a^j\}$ auf $\partial\Omega_j$ und dann in
            $\Omega_j$
          \item
            Konstruktion eines Voronoi-Diagramms:
            
            Wenn das duale Gitter
            $\omega_j = \{x\in \Omega: |x - a^j| \le |x - a^n| \forall n\in \N\}$
            \emph{regulär} ist, d.h. in jedem Schnittpunkt treffen sich genau
            3 Kanten, dann definiere $\Triangulation$ mit:
            \begin{eqnarray*}
                (a^j, a^n) \text{ ist Kante}
                \quad \Leftrightarrow \quad
                \meas_{d-1}(\omega_j \cap \omega_n) > 0
            \end{eqnarray*}
      \end{itemize}
      \begin{Bemerkung}
          In $2-D$ wird damit zu gegebenen Punkten die Triangulierung mit
          kleinsten Winkel bestimmt!
      \end{Bemerkung}
    \item
      Advancing Front
      \begin{itemize}
          \item
            Bilde eine \emph{Front} aus gegebenen Randabschnitten
          \item
            Füge am Rand Elemente ein und bilde damit eine neue Front
      \end{itemize}
      Wenn die Front zusammenwächst muss die Kolission von eingefügten
      Elementen vermieden werden
\end{enumerate}


\subsubsection*{Synopsis}


\begin{tabular}{c|c|c}
    A) Overlay & B) Delaunay & C) Advancing Front \\
    \hline
    $O(N)$ & $O(N \log(N))$ & $O(N^2)$ \\
    schnell & langsam & sehr langsam \\
    schlechte Randanpassung & optimale Randanpassung & beste Randanpassung
\end{tabular}

TODO: ``optimal'' vs. ``beste''?


\subsubsection{Gitteradaption}


Nach der Lösung des Finite-Elemente-Problems kann das Gitter angepasst werden:
\begin{itemize}
  \item
    Neu vernetzen mit Gitterfunktion
  \item
    Verfeinern:
    \begin{itemize}
      \item
        Einfügen von neuen Elementen bei denen der Fehler gro\ss{} geschätzt
        wird
    \end{itemize}
  \item
    Entfeinern:
    \begin{itemize}
      \item
        Frühere Verfeinerung rückgängig machen bei denen Fehler klein
        geschätzt wird
    \end{itemize}
\end{itemize}


\begin{enumerate}[1)]
  \item
    Bisektion:
    \begin{itemize}
      \item
        Markiere Elemente
      \item
        Markiere dann längste Kante zur Verfeinerung
    \end{itemize}
  \item
    Vollständiger Regelsatz:
    \begin{itemize}
      \item 
        Markiere alle Kanten
        \begin{enumerate}[a)]
          \item
            reguläre Verfeinerung
          \item
            irreguläre Verfeinerung
        \end{enumerate}
    \end{itemize}
\end{enumerate}


Irreguläre Elemente dürfen nicht weiter verfeinert werden, sondern müssen
vorher durch reguläre Elemente ersetzt werden.


\paragraph{Alternative:}

Verwende nur reguläre Verfeinerung und \emph{hanging nodes}.
Für hanging nodes gilt, falls $z^{ij}$ zwischen $z^i$ und $z^j$ liegt und
$v_h\in V_h = \{v\in C(\Omega): v \circ \Phi_K \in \hat V\}$ dann ist die
Nebenbedienung $v(z^{ij}) = \frac{1}{2} (v(z^i) + v(z^j))$ an den hanging nodes.


\subsubsection{A-posteriori-Fehlerschätzer}

Im Gegensatz zu a-priori Fehlerabschätzungen wie in \eqref{satz:4.15}, mit deren Hilfe man
ohne Kenntnis der exakten Lösung $u$ und der Approximation $u_h$
Konvergenzgeschwindigkeiten beweisen kann, dienen a-posteriori Fehlerschätzer dazu, \emph{nach}
der Berechnung einer Approximation $u_h$ die Abweichung von der tatsächlichen Lösung $u$ zu schätzen.

Dabei müssen die vom Schätzer verwendeten Größen bei bekannter Approximation $u_h$ explizit berechenbar sein.
Eine wichtige Anwendung der Fehlerschätzer ist die adaptive Verfeinerung des verwendeten Gitters.
Daher gibt ein guter Fehlerschätzer nicht nur Informationen über den Fehler im gesamten Rechengebiet
$\Omega$, sondern auch lokal in jeder Zelle $K \in \Triangulation$.

\begin{Beispiel}[Der Z-Z-Schätzer nach Zienkiewicz und Zhu]
    Betrachte $-\Delta u = f$ in $H_0^1(\Omega)$, d.h. $u = 0$ auf
    $\partial\Omega$.
    Zur Finite-Element-Lösung $u_h$ berechne eine Mittelung von $\nabla u_h$ in der Umgebung
    $\Delta(z) := \{K\in \Triangulation: z\in K\}$ jedes Knotenpunktes $z \in \Nodes$ durch
    \begin{eqnarray*}
        q(z) = \frac{1}{\#\Delta(z)} \sum_{z\in K} \nabla u_h|_K(z) \in \R^d.
    \end{eqnarray*}
    Damit lässt sich eine Approximation des Gradienten in $V_h^d$ mit
    \begin{eqnarray*}
        q_h = \sum_{z\in K} q(z) \Phi_z \in V_h^d
    \end{eqnarray*}
    berechnen und es folgt mit \eqref{lem:3.20}
    \begin{eqnarray*}
              \|u - u_h\|_1
        &\le& C \|\nabla u - \nabla u_h\|_0 \\
        &=& C \sup_{v\in H_0^1(\Omega)}
            \frac{(\nabla u - \nabla u_h, \nabla v)_0}{\|\nabla v\|_0} \\
        &\le& C \left(\sup_{v\in H_0^1(\Omega)}
              \frac{(\nabla u - q_h, \nabla v)_0}{\|\nabla v\|_0} +
              \frac{(q_h - \nabla u_h, \nabla v)_0}{\|\nabla v\|_0}\right) \\
        &\le& C \left(\sup_{v\in H_0^1(\Omega)}
              \frac{(f + \nabla \cdot q_h, v)_0}{\|\nabla v\|_0} +
              \frac{\|q_h - \nabla u_h\|_0 \|\nabla v\|_0}{\|\nabla v\|_0}
              \right) \\
        &\le& C \Bigg(\|f + \nabla \cdot q_h\|_0 \underbrace{\sup_{v\in H_0^1(\Omega)}
              \frac{\|v\|_0}{\|\nabla v\|_0}}_{\le C_{PF}} +
              \|q_h - \nabla u_h\|_0\Bigg).
    \end{eqnarray*}
    Der Divergenzsatz von Gauß liefert
    \begin{eqnarray*}
            (\nabla u - q_h, \nabla v)_0
        &=& \int_\Omega (\nabla u - q_h) \cdot \nabla v \dx \\
        &=& -\int_\Omega \nabla \cdot (\nabla u - q_h) v \dx
            \qquad (v|_{\partial\Omega} = 0) \\
        &=& (-\Delta u +\nabla \cdot q_h, v)_0.
    \end{eqnarray*}
    Nun definieren wir einen lokalen Fehlerschätzer für jede Zelle $K \in \Triangulation$ durch
    \begin{eqnarray*}
        \eta_K := \|f + \nabla \cdot q_h\|_{0,K} + \|q_h - \nabla u_h\|_{0,K}
    \end{eqnarray*}
    und erhalten so den globalen Fehlerschätzer $\eta = \left(\sum_{K\in\Triangulation} \eta_K^2\right)^\frac{1}{2}$. 
    Für diesen gilt nach der obigen Rechnung
    \begin{eqnarray*}
        \|u - u_h\|_1 \le C \eta.
    \end{eqnarray*}
\end{Beispiel}



\subsubsection{Die Cl\'ement-Interpolation}

\begin{Satz}
    \label{satz:4.16}
    Sei $X_h = \Span\{\Phi_j: j=1,\dots,N\}$ ein Lagrange-Finite-Elemente-Raum.
    Wir definieren durch
    \begin{eqnarray*}
        \Pi_h v = \sum_{j=1}^N P_jv\, \Phi_j,
        \qquad
        P_jv = \frac{1}{\meas_d(\omega_j)} \int_{\omega_j} v \dx,
        \qquad
        \omega_j = \supp \Phi_j
    \end{eqnarray*}
    die Cl\'ement-Interpolation $\Pi_h\colon L_1(\Omega) \longrightarrow X_h$.
    
    Sei $\omega_K = \bigcup_{\omega_j\supset K} \omega_j$.
    Dann gilt für $v\in H^1(\Omega)$:
    \begin{enumerate}[a)]
      \item
        $\|v - \Pi_h v\|_{0,K} \le C h_K \|\nabla v\|_{0,\omega_K}$
      \item
        $\|v -\Pi_h v\|_{0,F} \le C h_F^{1/2} 
        \|\nabla v\|_{0,\omega_K}$ für jede Seitenfläche $F \subset \partial K$.
\end{enumerate}
\end{Satz}


\begin{Bemerkung}
    Für reguläre Gitter gilt
    $\#\{K: K \subset \omega_j\} \le C$.
\end{Bemerkung}
      

\begin{proof}
    \begin{enumerate}[a)]
      \item
        Sei $v\in C^1(\overline\omega_j)$ dann gilt
        \begin{eqnarray*}
                \meas(\omega_j)(v - P_jv)(x)
            &=& \int_{\omega_j} (v(x) - v(y)) \,dy
                \qquad \forall x \\
            &=& \int_{\omega_j} \int_0^1 \left(-\frac{d}{dt} v(x + t(y - x))
                \right) \,dt \,dy \\
            &=& \int_{\omega_j} \int_0^1 \nabla v(x + t(y - x)) (x - y) \,dt
                \,dy \\
            &\stackrel{\text{Trafo}}{=}&
                \int_{\omega_j} \int_0^1 \nabla v(z) (x - z) t^{-1-d}
                \kappa_j(x,z,t) \,dt \,dz
        \end{eqnarray*}
        mit $z = x + t(y - x), \ z - x = t(y - x), \ \,dz = t^{-d} \dx$ und
        \begin{eqnarray*}
            \kappa_j(x,z,t) = \begin{cases}
                                  1 \qquad x + t(y - x)\in \omega_j \\
                                  0 \qquad \text{sonst}
                              \end{cases}.
        \end{eqnarray*}
        Es gilt 
        \begin{eqnarray*}
            2h \ge |y - x| = \frac{1}{t} |z - x| \qquad \forall x,y\in \omega_j
        \end{eqnarray*}
        und damit
        \begin{eqnarray*}
            \kappa_j(x,z,t) = 0 \qquad \text{für } t < \frac{1}{2h} |z - x|,
        \end{eqnarray*}
        sodass
        \begin{eqnarray*}
                k_j(x,z)
            &:=& \int_0^1 \kappa_j(x,z,t) t^{-1-d} \,dt \\
            &\le& -\frac{1}{d} t^{-d} \Bigg|_{t = \frac{|z - x|}{2h}}^1 \\
            &\le& \frac{1}{d} \frac{(2h)^d}{|z - x|^d}
                  \qquad \text{für } z \neq x
        \end{eqnarray*}
        Damit gilt folgende Abschätzung
        \begin{eqnarray*}
                |v(x) - P_jv|
            \le \underbrace{C \frac{h^d}{\meas\omega_j}}_{\le \tilde C}
                \int_{\omega_j} |\nabla v(z)| |x - z|^{1-d} \,dz.
        \end{eqnarray*}
        Diese lässt sich weiter abschätzen durch
        \begin{eqnarray*}
                \int_{\omega_j} |\nabla v(z)| |x - z|^{\frac{1-d}{2}}
                |x - z|^{\frac{1-d}{2}} \,dz
            \stackrel{\text{CSU}}{\le}
                \left(\int_{\omega_j} |\nabla v(z)|^2 |z - x|^{1-d} \,dz
                \right)^\frac{1}{2} \\
                \Biggl(\underbrace{\int_{\omega_j} |z - x|^{1-d}\,dz}
                _{\le \int_{S^{d-1}} \int_0^{2h} r^{1-d} r^{d-1} \,dr \da \le
                C h}\Biggr)^\frac{1}{2}
        \end{eqnarray*}
        Damit ergibt sich also
        \begin{eqnarray*}
                  \|v - P_jv\|_{0,\omega_j}^2
            &\le& Ch \int_{\omega_j} \int_{\omega_i} |\nabla v(z)| |x - z|^{1-d}
                  \,dz \dx \\
            &\le& Ch \|\nabla v\|_{0,\omega_j}^2 \sup_{x\in\omega_j}
                  \int_{\omega_j} |x - z|^{1-d} \,dz \\
            &\le& \tilde C h^2 \|\nabla v\|_{0,\omega_j}^2.
        \end{eqnarray*}
        Insgesamt folgt
        \begin{eqnarray*}
                \|v - P_jv\|_{0,\omega_j}
            \le Ch \|\nabla v\|_{0,\omega_j}.
        \end{eqnarray*}
        Mit der Zerlegung der $1 = \sum_{j=1}^N \Phi_j$ folgt
        \begin{eqnarray*}
                  \|v - \Pi_hv\|_{0,K}
            &\le& \left\|\sum_{j=1}^N (v - P_jv) \Phi_j\right\|_{0,K} \\
            &\le& C \sup_{\omega_j \supset K} \|v - P_jv\|_{0,\omega_j}
                  \sup_{x\in\omega_j} |\Phi_j(x)| \\
            &\le& \tilde C h \|\nabla v\|_{0,\omega_K}.
        \end{eqnarray*}
      \item
        Sei $\hat F \subset \hat K$ und $F = \varphi_K(\hat F)$ dann gilt
        \begin{eqnarray*}
              \int_F v \da
            = \int_{\hat F} v \circ \varphi_K \underbrace{|F_K^{-1} \hat n|}
              _{\le C h^{-1}} \underbrace{J_K}_{\le C h^d} \,d\hat a
        \end{eqnarray*}
        und somit
        \begin{eqnarray*}
                  \|v\|_{0,F}^2
            &\le& C h^{d-1} \|\hat v\|_{0,\hat F}^2 \\
            &\le& C h^{d-1} \tilde C_{\text{PF}}(\|\hat v\|_{0,\hat K}^2 +
                  \|\nabla \hat v\|_{0,\hat K}^2) \\
            &\le& C h^{d-1} \tilde C_{\text{PF}} (C h^{-d} \|v\|_{0,K}^2 + C
                  h^{2-d} \|\nabla v\|_{0,K}^2).
        \end{eqnarray*}
        Mit $\|v - P_jv\|_{0,\omega_j} \le Ch \|\nabla v\|_{0,\omega_j}$ aus
        Beweisteil a) folgt
        \begin{eqnarray*}
            \|\nabla (v - \Pi_h v)\|_{0,K}
            &\le& \left\|\sum_{j=1}^K |\nabla v| |\Phi_j| +
                  |v - P_j v| |\nabla \Phi_j|\right\|_{0,K} \\
            &\le& C \|\nabla v\|_{0,K} + C \|v - P_j v\|_{0,K} h^{-1} \\
            &\le& \tilde C \|\nabla v\|_{0,\omega_K}
        \end{eqnarray*}
        und wir erhalten
        \begin{eqnarray*}
                  \|v - \Pi_h v\|_{0,F}^2
            &\le& C h^{-1} \|v - \Pi_h v\|_{0,K}^2 +
                  C h \|\nabla (v - \Pi_h v)\|_{0,F}^2 \\
            &\stackrel{\eqref{satz:4.16}}{\le}&
                  C h \|\nabla v\|_{0,\omega_K}^2,
        \end{eqnarray*}
        was zu zeigen war.
    \end{enumerate}    
\end{proof}


\subsubsection{Residualer Fehlerschätzer}



Im Folgenden betrachten wir $Lu = -\Delta u + qu$ und
$a(u, v) = \int_\Omega (\nabla u \cdot \nabla v + q u v) \dx$. 

\begin{Lemma}
    \label{lem:4.17}
    Sei $Lu = -\Delta u + q u$. Dann gilt für die schwache Lösung $u\in  V$ 
    von $Lu = f$ und $u_h\in V_h$
    \begin{eqnarray*}
            a(u - u_h, v)
        \le C \left(\sum_{K\in \mathcal T_h} h_K^2 \|r_K(u_h)\|_{0,K}^2 +
            \sum_{F\in \mathcal F_h} h_F\| [\nu_F \cdot \nabla u_h]_F
            \|_{0,F}^2\right)^{1/2} \| v \|_1
    \end{eqnarray*}
    mit $r_K(u_h) = (L u_h - f)|_K$ Element-Residuum und Sprung von $\nabla u_h$
    an der Seitenfläche $F = K \cap K^\prime$, d.h.
    $[\nu_F \cdot w]_F (x) = \lim\limits_{t\to 0} \nu_F \cdot \big(w(x + t\nu_F)
    - w(x - t\nu_F)\big)$ und der Menge aller inneren Seitenflächen
    $\mathcal{F}_h$.
\end{Lemma}


\begin{proof}
    Es gilt
    \begin{eqnarray*}
            a(u - u_h, v)
        &=& a(u, v) - a(u_h, v) \\
        &=& l(v) - a(u_h, v) \\
        &=& \sum_{K\in\Triangulation} \int_K (fv - \nabla u_h \cdot \nabla v -
            q u_h v) \dx \\
        &\stackrel{\text{Gauß}}{=}&
            \sum_{K\in\Triangulation} \left(\int_K (f + \Delta u_h - q u_h) v
            \dx - \int_{\partial K} \nabla u_h \cdot \nu_{\partial K} v \da
            \right)\\
        &=& \sum_{K\in\Triangulation} -\int_K r_K(u_h) v \dx -
            \sum_{F\in\mathcal{F}_h} \int_F [\nu_F \cdot \nabla u_h]_F v \da \\
        &\le& \sum_{K\in\Triangulation} \|r_K(u_h)\|_{0,K} \|v\|_{0,K} +
              \sum_{F\in\mathcal{F}_h} \|[\nu_F \cdot \nabla u_h]_F\|_{0,F}
              \|v\|_{0,F}.
    \end{eqnarray*}
    Mit der Galerkin-Orthogonalität $a(u - u_h, \Pi_h v) = 0$ folgt
    \begin{eqnarray*}
            a(u - u_h, v)
        &=& a(u - u_h, v - \Pi_h v) \\
        &\le& \sum_{K\in\Triangulation} h_K \|r_K(u_h)\|_{0,K} \|v -
              \Pi_h v\|_{0,K} h_K^{-1} \\
            &&+ \sum_{F\in\mathcal{F}_h} h_F^\frac{1}{2}
              \|[\nu_F \cdot \nabla u_h]_F\|_{0,F} \|v - \Pi_h v\|_{0,F}
              h_F^{-\frac{1}{2}} \\
        &\stackrel{\text{CSU}}{\le}&
              \left(\sum_{K\in\Triangulation} \|r_K(u_h)\|_{0,K}^2 h_K^2\right)
              ^\frac{1}{2} \left(\sum_{K\in\Triangulation} h_K^{-2} \|v - 
              \Pi_h v\|_{0,K}^2\right)^\frac{1}{2} \\
            &&+ \left(\sum_{F\in\mathcal{F}_h} \|[\nu_F \cdot \nabla u_h]_F\|
              _{0,F}^2 h_F\right)^\frac{1}{2} \left(\sum_{F\in\mathcal{F}_h}
              h_F^{-1} \|v - \Pi_h v \|_{0,F}^2\right)^\frac{1}{2} \\
        &\le& C \left(\left(\sum_{K\in\Triangulation} \|r_K(u_h)\|_{0,K} h_K^2
              \right)^\frac{1}{2} + \left(\sum_{F\in\mathcal{F}_h} \|[\nu_F
              \cdot \nabla u_h]_F\|_{0,F}^2 h_F\right)^\frac{1}{2}\right)
              \| v\|_1
    \end{eqnarray*}
    die Behauptung.
\end{proof}


\begin{Folgerung}
    \label{folgerung:4.18}
    Der Fehlerschätzer $\eta = \left(\sum_{K\in\Triangulation} \eta_K^2\right)
    ^\frac{1}{2}$ mit
    \begin{eqnarray*}
          \eta_K
        = \left(\sum_{K\in\Triangulation} h_K^2 \|r_K(u_h)\|_{0,K}^2
          + \sum_{F\in\mathcal{F}_h} h_F
          \|[\nu_F \cdot \nabla u_h]_F\|_{0,F}^2\right)^\frac{1}{2}
    \end{eqnarray*}
    ist zuverlässig, d.h. $\|u - u_h\|_1 \le C \eta$.
\end{Folgerung}


\begin{proof}
    Es gilt
    \begin{eqnarray*}
            \|u - u_h\|_1^2
        \le \frac{1}{\alpha_0} a(u - u_h, u - u_h)
        \stackrel{\eqref{lem:4.17}}{\le} \frac{1}{\alpha_0} C \eta \|u - u_h\|_1
        ,
    \end{eqnarray*}
    also ist der Fehlerschätzer zuverlässig.
\end{proof}
