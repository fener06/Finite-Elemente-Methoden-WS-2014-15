\section{FE-Diskretisierung der Stokes-Gleichungen}


\begin{Definition}
    \label{def:5.1}
    Sei $\Omega\subset \R^d, \ f\in C(\Omega, \R^d), \ g\in C(\partial\Omega, \R^d)$
    . Gesucht ist eine klassische Lösung  $(u, p)\in \big( C^2(\Omega, \R^d) \cap
    C(\overline\Omega, \R^d) \big) \times C^1(\Omega, \R)$ mit
    \begin{eqnarray*}
        -\Delta u + \nabla p &=& f \qquad \text{in } \Omega \\
        \nabla \cdot u &=& 0 \qquad \text{in } \Omega \\
        u &=& g \qquad \text{auf } \partial\Omega
    \end{eqnarray*}
\end{Definition}


\begin{Bemerkung}
    \begin{enumerate}[1)]
      \item
        Die Stokes-Gleichungen beschreiben die Strömung eines (sehr) viskosen
        Fluids, z.B. Mikroströmungen (etwa Wasser durch Sand):

        Diese werden durch das Darcy-Gesetz beschrieben:
        $u \approx \nabla q -b$

        Mit $\nabla \cdot u = 0$ gilt $\Delta q = \nabla \cdot b$, wobei
        $\nabla \cdot u = 0$ die Volumenerhaltung beschreibt.

        Insbesondere muss genauso viel aus $\Omega$ herausfließen wie hineinfließt:
        \begin{eqnarray*}
              \int_{\partial\Omega} u \cdot \nu \,da
            = \int_\Omega \nabla \cdot u \,dx
        \end{eqnarray*}
        mit der Nebenbedingung folgt die Kompatibilitätsbedingung:
        \begin{eqnarray*}
            \int_{\partial\Omega} g \cdot \nu \,da = 0.
        \end{eqnarray*}
      \item
        Der Druck $p$ ist nur bis auf eine Konstante bestimmt.
    \end{enumerate}
\end{Bemerkung}


Annahme: Es existiert $u_D$ mit $\nabla \cdot u_D = 0$ in $\Omega$ und $u_D = g$ auf
$\partial\Omega$.

Für $u = u_D + u_0$ ergibt sich für die Stokes-Gleichung
\begin{eqnarray*}
    -\Delta u_0 + \nabla p &=& f + \Delta u_D
    \qquad \text{in } \Omega \\
    \nabla \cdot u_0 &=& 0
    \qquad \qquad \ \ \ \text{in } \Omega \\
    u_0 &=& 0
    \qquad \qquad \ \ \ \text{auf } \partial\Omega
\end{eqnarray*}
Setzte wieder $u := u_0, \ f:= f + u_D$.

Im Folgenden sei ohne Einschränkung $g \equiv 0$.


\begin{Lemma}
    \label{lem:5.2}
    Sei $X = H^1_0(\Omega, \R^d)$ und definiere
    $V = \{v\in X: \nabla \cdot v \equiv 0\}$. Sei $(u,p)$ eine klassische
    Lösung von \eqref{def:5.1} mit $g\equiv 0$. Dann gilt
    \begin{eqnarray*}
        a(u, v) = l(v)
        \qquad \text{für alle } v\in V
    \end{eqnarray*}
    mit $a(v,w) = \int_\Omega \nabla v : \nabla w\, dx$ und  
    $l(v) = \int_\Omega f \cdot v\, dx$ mit
    $\nabla v : \nabla w = \sum_{i,j = 1}^d \partial_i v_j \partial_i w_j$ und
    $f \cdot v = \sum_{i = 1}^d f_i v_i$.
\end{Lemma}

%    Definiere
%    $b(v,q) = \int_\Omega \nabla \cdot v q\, dx$. Zu jedem
%    $f\in L_2(\Omega, \Real^d)$ existiert eine eindeutige schwache Lösung
%    $u\in V$.

\begin{proof}
    Es gilt
    \begin{eqnarray*}
            l(v)
        &=& \int_\Omega f \cdot v \,dx \\
        &=& \int_\Omega (-\Delta u + \nabla p) \cdot v \,dx \\
        &=& \int_\Omega \sum_{i = 1}^d (-\Delta u_i v_i + \partial_i p v_i)
            \,dx \\
        &=& \int_\Omega \sum_{i,j = 1}^d (\partial_j u_i
            \partial_j v_i - p \partial_i v_i) \,dx \\
        &=& \int_\Omega \nabla u : \nabla v \,dx
            - \int_\Omega p \nabla \cdot v \,dx \\
        &=& a(u , v),
    \end{eqnarray*}
    da $\nabla \cdot v = 0$ für alle $v\in V$.
\end{proof}


\begin{Definition}
    \label{def:5.3}
    Sei $X_h \subset H^1_0(\Omega,\Real^d)$ und $Q_h\subset L_2(\Omega)/\Real$
    Finite-Element-Räume. Definiere
    \begin{eqnarray*}
        V_h = \{ v_h\in X_h: \int_\Omega (\nabla \cdot v_h) q_h \,dx = 0
        \text{ für alle } q_h\in Q_h\}
    \end{eqnarray*}
    mit $b(v, q) = \int_\Omega (\nabla \cdot v) q \,dx$. 
    Dann hei\ss{}t $u_h\in V_h$ mit
    \begin{eqnarray*}
        a(u_h, v_h) = l(v_h)\qquad
        \text{ für alle } v_h\in V_h.
    \end{eqnarray*}
    Finite-Element-Lösung der Stokes-Gleichung.
\end{Definition}


\begin{Bemerkung}
    \begin{enumerate}[1)]
      \item 
        Falls $a(v ,v) \ge \alpha_0 \|v\|_1^2$ für alle $v\in H_0^1(\Omega,
        \R^d)$, dann ist $u_h\in V_h$ in \eqref{def:5.3} eindeutig bestimmt!
      \item
        Im Allgemeinen gilt $v_h\in X_h$ und $\nabla \cdot v_h  \equiv 0$,
        sodass $v_h \equiv 0$ ist.
      \item
        Im Allgemeinen ist $V_h \not \subset V$ nicht konform! Also keine
        Galerkin-Orthogonalität in $V_h$ möglich (aber später in $V_h
        \times Q_h$).
    \end{enumerate}
\end{Bemerkung}


\begin{Lemma}[2. von Strang]
    \label{lem:5.4}
    Sei $a_h(v, w) = a(v, w)$ für $v,w\in V$ und sei $\|\cdot\|_h$ eine Norm auf
    $V+V_h$. Sei $u\in V$ mit $a(u, v) = l(v)$ für alle $v\in V$, und sei
    $u_h\in V_h$ mit $a_h(u_h, v_h) = l_h(v_h)$ für alle $v_h\in V_h$. Wenn
    $|a_h(v, w) | \le C_a \|v\|_h \|w\|_h$ für $v,w\in V+V_h$ und 
    $a_h(v_h, v_h) \ge \alpha_0 \|v_h\|_h^2$ für $v_h\in V_h$ gilt, dann
    existiert $C>0$ mit
    \begin{eqnarray*}
              \|u - u_h\|_h
          \le C \inf_{v_h\in V_h} \Big(\|u - v_h\|_h
              + \sup_{\|w_h\|_h = 1} |a_h(u, w_h)- l_h(w_h)|\Big).
    \end{eqnarray*}
\end{Lemma}


\begin{proof}
    Für $v_h\in V_h \setminus\{u_h\}$ gilt
    \begin{eqnarray*}
              \alpha_0 \|u_h - v_h\|_h^2
        &\le& a_h(u_h - v_h, u_h - v_h) \\
        &=& a_h(u - v_h, u_h - v_h) + a_h(u_h, u_h - v_h) - a_h(u, u_h - v_h) \\
        &=& a_h(u - v_h, u_h - v_h) + \frac{l_h(u_h - v_h) - a_h(u, u_h - v_h)}
            {\|u_h - v_h\|_h} \|u_h - v_h\|_h \\
        &\le& (C_a \|u - v_h\|_h + \sup_{\|w_h\|_h = 1} |l_h(w_h) -
              a_h(u, w_h)|) \|u_h - v_h\|_h
    \end{eqnarray*}
    und es folgt
    \begin{eqnarray*}
              \|u - u_h\|_h
        &\le& \|u - v_h\|_h + \|v_h - u_h\|_h \\
        &\le& \left(1 + \frac{C_a}{\alpha_0}\right)\|u - v_h\|_h +
              \frac{1}{\alpha_0} \sup_{\|w_h\|_h = 1} |l_h(w_h) - a_h(u, w_h)|,
    \end{eqnarray*}
    die Fehlerabschätzung.
\end{proof}


\begin{Bemerkung}
    Anwendung für $a_h(\cdot, \cdot) \equiv a(\cdot, \cdot)$ und $\|\cdot\|_h
    \equiv \|\cdot\|_1$.
\end{Bemerkung}


\begin{Bemerkung}
    Wahl von $X_h$ und $Q_h$ ist ein Kompromiss, sodass der Approximationsfehler
    und der Konsistenzfehler die gleiche Ordnung haben.
\end{Bemerkung}


\begin{Bezeichnung}
    \begin{enumerate}[1)]
      \item  
        $v_h\in X_h$ mit $b(v_h, q_h) = 0$ für alle $q_h\in Q_h$ hei\ss{}t
        \emph{diskret divergenzfrei}.
      \item
        $q_h\in Q_h$ mit $q_h \not\equiv 1$ und $b(v_h, q_h) = 0$ für alle
        $v_h\in X_h$ hei\ss{}t \emph{spurious pressure mode}.
    \end{enumerate}
\end{Bezeichnung}


\subsection{Sattelpunkt-Formulierung}



Für die Lösung $u_h\in V_h$ gilt:
\begin{eqnarray*}
    a(u_h, v_h) = l(v_h)
    \qquad \text{für alle } v_h\in V_h.
\end{eqnarray*}
Die Lösung $u_h\in V_h$ erfüllt:
\begin{eqnarray*}
    F(u_h) = \frac{1}{2}a(u_h, u_h) - l(u_h) \le F(v_h)
    \qquad \text{für alle } v_h\in V_h.
\end{eqnarray*}
Die Lösung $u_h\in X_h$ minimiert also $F(u_h)$ unter der Nebenbedingung
$b(u_h, q_h) = 0$ für alle $q_h\in Q_h$.


\begin{Lemma}
    \label{lem:5.5}
    Wenn $(u_h,p_h)\in X_h \times Q_h$ ein Sattelpunkt von 
    dem Lagrangefunktional
    \begin{eqnarray*}
        L(v_h,q_h) = F(v_h) + b(v_h,q_h)
    \end{eqnarray*}
    ist, d.h.
    \begin{eqnarray*}
        L(u_h,q_h) \le L(u_h,p_h) \le L(v_h,p_h)
        \qquad
        \text{ für alle } (v_h,q_h) \in X_h \times Q_h \,,
    \end{eqnarray*}
    dann ist $u_h\in V_h$ und $u_h$ löst \eqref{def:5.3}.
\end{Lemma}


\begin{proof}
    Die Sattelpunktbedingung liefert uns
    \begin{eqnarray*}
        0 \le L(u_h, p_h) - L(u_h, q_h) = b(u_h, p_h - q_h).
    \end{eqnarray*}
    Setze nun $q_h = p_h \pm \tilde q_h$, dann ist
    \begin{eqnarray*}
        b(u_h, -\tilde q_h) \le 0 \le b(u_h, \tilde q_h).
    \end{eqnarray*}
    Also ist $b(u_h, \tilde q_h) = 0$ für alle $\tilde q_h\in Q_h$ und somit
    $u_h\in V_h$ und es folgt
    \begin{eqnarray*}
        F(u_h) = L(u_h, p_h) \le L(v_h, p_h) = F(v_h)
        \qquad \text{für alle } v_h\in V_h.
    \end{eqnarray*}
    Somit löst $u_h\in V_h$ \eqref{def:5.3}.
\end{proof}

Wählen wir uns nun eine Basis für die Finite-Element-Räume des
Geschwindigkeitsfeldes und Druckes, sodass
$X_h = \Span\{\Phi_n: n = 1, \dots, N\}$ und
$Q_h = \Span\{\Psi_m: m = 1, \dots, M\}$, 
dann ergibt sich für das Lagrangefunktional mit den Lösungen
$u_h = \sum_{n=1}^N \underline u_n \Phi_n$ und
$p_h = \sum_{m=1}^M \underline p_m \Psi_m$ folgende Darstellung
\begin{eqnarray*}
      L(u_h, p_h)
    = \frac{1}{2} \underline u^T \underline A \underline u - \underline u^T
      \underline b + \underline u^T \underline B^T \underline p,
\end{eqnarray*}
wobei
\begin{eqnarray*}
    \underline A &=& (a(\Phi_n, \Phi_k))_{n,k=1,\dots,N}\in \R^{N,N}, \\
    \underline B &=& (b(\Psi_k, \Phi_n))_{k=1,\dots,M;n=1,\dots,N}\in \R^{M,N}
    \qquad \text{und} \\
    \underline b &=& (l(\Phi_n))_{n=1,\dots,N}\in \R^N.
\end{eqnarray*}
Also ist $(u_h, p_h)$ Sattelpunkt, wenn $(\underline u, \underline p)$ kritische
Stelle von $L(\cdot, \cdot)$. ist, d.h.
\begin{eqnarray*}
    \underline A \underline u + \underline B^T \underline p &=& \underline b \\
    \underline B \underline u &=& 0.
\end{eqnarray*}


\begin{Lemma}
    \label{lem:5.6}
    Wenn $\underline B \underline A^{-1} \underline B^T\in \R^{M,M}$ regulär
    ist, dann existiert ein $p_h\in Q_h$, sodass $(u_h, p_h)$ das
    Sattelpunktproblem löst.
\end{Lemma}


\begin{proof}
    Nach der Sattelpunktbedingung gilt
    \begin{eqnarray*}
        \underline u = \underline A^{-1} (\underline b - \underline B^T
        \underline p).
    \end{eqnarray*}
    Ebenfalls liefert uns diese
    \begin{eqnarray*}
          0
        = \underline B \underline u
        = \underline B \underline A^{-1} (\underline b - \underline B^T
          \underline p),
    \end{eqnarray*}
    sodass
    \begin{eqnarray*}
          \underline p
        = (\underline B \underline A^{-1} \underline B^T)^{-1} \underline B
          \underline A^{-1} \underline b.
    \end{eqnarray*}
    Das wiederum bedeutet, dass ein $p_h\in Q_h$ existiert, falls
    $\underline B \underline A^{-1} \underline B^T$ regulär ist.
\end{proof}


\begin{Bemerkung}
    Es gilt $\underline B \underline A^{-1} \underline B^T$ ist regulär genau
    dann, wenn $\underline p^T (\underline B \underline A^{-1} \underline B^T)
    \underline p > 0$ für alle $\underline p \neq 0$, d.h. da
    $\underline A$ symmetrisch ist gilt
    $\sup_{|\underline v| = 1} \underline v \underline B^T \underline p > 0$ mit
    $\underline v = \underline A^{-1} \underline B^T \underline p$.
\end{Bemerkung}

Es stellt sich die Frage, ob diese Bedingung genügt!

\begin{Beispiel}[Das $Q_1$-$P_0$-Element auf Vierecken]
    Sei $X_h = \{v\in H_0^1(\Omega, \R^2): v|_K\in \mathbb Q_1^2 =
    \Polynom_1 \otimes \Polynom_1\}$ der Raum der bilinearen Elemente für das
    Geschwindigkeitsfeld und
    $Q_h = \{q_h\in L_2(\Omega): q_h|_K\in \Polynom_0\} / \R$ der Raum
    der Konstanten Elemente für den Druck auf jedem Element.
    Weiter sei $\Omega = (0, 1)^2, \ h = \frac{1}{J}, \ z^{ij} = (t_i, t_j), \
    t_i = \frac{i}{J} = i h, \ K_{ij} = \conv\{z^{ij}, z^{i+1,j}, z^{i,j+1},
    z^{i+1,j+1}\}$.

    Dann gilt der \emph{checker board mode}
    \begin{eqnarray*}
        q_h|_{K_{ij}} = \begin{cases}
                            a \qquad \text{für } i + j \text{ gerade} \\
                            b \qquad \text{für } i + j \text{ ungerade}
                        \end{cases}.
    \end{eqnarray*}
    Zur Vereinfachung setze $u = \left(\begin{smallmatrix}
                                    U \\
                                    V
                                \end{smallmatrix}\right)$
    und wir erhalten
    \begin{eqnarray*}
            b(u, q_h)
        &=& \sum_{K\in \Triangulation} \int_K \nabla \cdot u \, q_h \,dx \\
        &=& \sum_{i,j} h^2 q_{ij} \frac{1}{h} (U_{i+1,j+1} + U_{i+1,j} -
            U_{i,j+1} - U_{ij} \\
            &&+ V_{i+1,j-1} + V_{i,j+1} - V_{i+1,j} - V_{ij})\\
        &=& h \sum_{i,j} U_{ij} (q_{ij} + q_{i,j+1} - q_{i-1,j} - q_{i-1,j-1} \\
            &&+ V_{ij} (q_{ij} + q_{i-1,j} - q_{i,j-1} - q_{i-1,j-1}) \\
        &=& 0.
    \end{eqnarray*}
    Für alle $v_h$ gilt somit $b(v_h, q_h) = 0$. Damit ist $\underline B
    \underline A^{-1} \underline B^T$ singulär und das Sattelpunktproblem
    nicht eindeutig lösbar.
\end{Beispiel}


\begin{Bemerkung}
    \begin{enumerate}[1)]
      \item
        Es existiert eine Lösung, aber die Lösung ist nicht eindeutig im
        Druck!
      \item
        Für das $Q_1$-$P_0$-Element konvergiert $u_h \to u$ und
        $\int_{\omega_K} p_h \to p$ mit $\omega_K \supset K$ Elementpatch.
    \end{enumerate}
\end{Bemerkung}


\begin{Definition}
    \label{def:5.7}
    $(X_h,Q_h)$ heißt \emph{inf-sup stabil}, wenn für alle $q_h\in Q_h$ gilt: 
    \begin{eqnarray*}
            \sup_{v_h\in X_h, \|v_h\|_1 = 1} b(v_h,q_h) 
        \ge \beta_0 \sup_{v\in X, \|v\|_1 = 1} b(v,q_h).
    \end{eqnarray*}
\end{Definition}


\begin{Bemerkung}
    Die inf-sup Stabilität hei\ss{}t auch \emph{LBB-Bedingung}.
\end{Bemerkung}


\begin{Bemerkung}
    $V \subset X = H_0^1(\Omega, \R^d)$ ist abgeschlossen in $X$.
    Also ist $X / V$ ein Hilbertraum bzgl. der Norm
    $\|v + V\|_{X / V} = \inf_{w\in v + V} \|w\|_1$.
    Der dazugehörige Dualraum $Q = (X / V)^\prime$ ist ebenfalls ein
    Hilbertraum bzgl. der Norm
    $\|q\|_Q = \sup_{\|v + V\|_{X / V} = 1} \langle q, v + V \rangle
    _{Q \times X / V}$.

    Für $q_h\in Q_h \subset L_2(\Omega) / \R$ definiere einen Operator
    \begin{eqnarray*}
        E_h: Q_h &\to& Q \qquad \text{mit} \\
        q_h &\mapsto& (v \mapsto b(v, q_h)).
    \end{eqnarray*}
    Wegen
    \begin{eqnarray*}
          b(v, 1)
        = \int_\Omega \nabla \cdot v \,dx
        = \int_{\partial\Omega} v \cdot \nu \,da
        = 0
    \end{eqnarray*}
    ist $E_h$ wohldefiniert in $L_2(\Omega) / \R$ und weil
    \begin{eqnarray*}
        b(v, q_h) = 0 \qquad \text{für } v\in V
    \end{eqnarray*}
    ist $v \mapsto b(v, q_h)$ in $(X / V)^\prime$. Daher gilt
    \begin{eqnarray*}
          \|E_h q_h\|_Q
        = \sup_{\|v\|_1 = 1} b(v, q_h)
    \end{eqnarray*}
    und somit ist die Stabilitätskonstante
    \begin{eqnarray*}
        \beta_0 = \inf_{\|E_h q_h\|_Q = 1} \sup_{\|v_h\|_1 = 1} b(v_h, q_h).
    \end{eqnarray*}
\end{Bemerkung}


\begin{Satz}[Fortin-Kriterium]
    \label{satz:5.8}
    Sei $F_h: X \to X_h$ ein linearer Operator
    (\emph{Fortin-Operator}) mit
    \begin{enumerate}[a)]
      \item
        $\|F_h v\|_1 \le C \|v\|_1$
      \item
        $b(v, q_h) = b(F_h v, q_h)$ für alle $v\in X$ und alle $q_h\in Q_h$
    \end{enumerate}
    Dann ist $(X_h, Q_h)$ inf-sup stabil mit $\beta_0 = \frac{1}{C}$.
\end{Satz}


\begin{proof}
    Es gilt
    \begin{eqnarray*}
            \sup_{v\in X, \|v\|_1 = 1} b(v, q_h)
        &=& \frac{b(F_h v, q_h)}{\|v\|_1} \\
        &\le& C \frac{b(F_h v, q_h)}{\|F_h v\|_1} \\
        &=& C \frac{b(v_h, q_h)}{\|v_h\|_1} \\
        &=& C \sup_{v_h\in X, \|v_h\|_1 = 1} b(v_h, q_h).
    \end{eqnarray*}
    $(X_h, Q_h)$ ist folglich inf-sup stabil mit $\beta_0 = \frac{1}{C}$.
\end{proof}


\begin{Satz}[Stabilität]
    \label{satz:5.9}
    Sei $(u_h, p_h)\in X_h \times Q_h$ die Lösung von
    \begin{eqnarray*}
        a(u_h, v_h) + b(v_h, p_h) = (f, v_h)_0 &=& l(v_h)
        \qquad \forall v_h\in X_h \\
        b(u_h, q_h) &=& 0
        \qquad \ \quad \ \forall q_h\in Q_h.
    \end{eqnarray*}
    Dann gilt unter der Voraussetzung \eqref{def:5.7}:
    \begin{enumerate}[a)]
      \item
        $\|u_h\|_1 \le \frac{1}{\alpha_0} \|l\|_{X^\prime}$ mit
        $\|l\|_{X^\prime} = \sup_{\|v\|_1 = 1} l(v)$
      \item
        $\|A^{-1} B_h^\prime p_h\|_1 \le \frac{2}{\alpha_0 \beta_0}
        \|l\|_{X^\prime}$
    \end{enumerate}
    Dabei ist
    \begin{eqnarray*}
        A: X &\to& X^\prime
        \qquad \ \text{definiert durch }
        \langle Av, w \rangle_{X^\prime \times X} = a(v, w) \\
        B_h: X &\to& Q_h^\prime
        \qquad \text{definiert durch }
        \langle B_h v, q_h \rangle_{Q_h^\prime \times Q_h} = b(v, q_h) \\
        B_h^\prime: Q_h &\to& X^\prime
        \qquad \ \text{definiert durch }
        \langle B_h^\prime q_h, v \rangle_{X^\prime \times X} = b(v, q_h)
    \end{eqnarray*}
\end{Satz}


\begin{proof}
    \begin{enumerate}[a)]
      \item
        Es gilt
        \begin{eqnarray*}
                \alpha_0 \|u_h\|_1^2
            \le a(u_h, u_h)
            = l(u_h)
            \le \|l\|_{X^\prime} \|u_h\|_1
        \end{eqnarray*}
        und damit
        \begin{eqnarray*}
            \|u_h\|_1 \le \frac{\|l\|_{X^\prime}}{\alpha_0}.
        \end{eqnarray*}
      \item
        Es gilt
        \begin{eqnarray*}
                  \alpha_0 \|A^{-1} B_h^\prime p_h\|_1^2
            &\le& a(A^{-1} B_h^\prime p_h, A^{-1} B_h^\prime p_h) \\
            &=& \langle A (A^{-1} B_h^\prime) p_h, A^{-1} B_h^\prime p_h \rangle
                _{X^\prime \times X} \\
            &=& \langle B_h^\prime p_h, A^{-1} B_h^\prime p_h \rangle
                _{X^\prime \times X} \\
            &=& b(A^{-1} B_h^\prime p_h, p_h).
        \end{eqnarray*}
        Für $B_h^\prime p_h \neq 0$ folgt
        \begin{eqnarray*}
                  \|A^{-1} B_h^\prime p_h\|_1
            &\le& \frac{1}{\alpha_0} \frac{b(A^{-1} B_h^\prime p_h, p_h)}
                  {\|A^{-1} B_h^\prime p_h\|_1} \\
            &\le& \frac{1}{\alpha_0} \sup_{v \neq 0} \frac{b(v, p_h)}{\|v\|_1}\\
            &\stackrel{\eqref{def:5.7}}{\le}&
                  \frac{1}{\alpha_0 \beta_0} \sup_{v_h \neq 0}
                  \frac{b(v_h, p_h)}{\|v_h\|_1} \\
            &\le& \frac{1}{\alpha_0 \beta_0} \sup_{\|v_h\|_1 = 1} (l(v_h) -
                  a(u_h, v_h)) \\
            &\le& \frac{1}{\alpha_0 \beta_0} (\|l\|_{X^\prime} +
                  \alpha_0 \|u_h\|_1) \\
            &\stackrel{\text{a)}}{\le}&
                  \frac{2}{\alpha_0 \beta_0} \|l\|_{X^\prime}
        \end{eqnarray*}
        die Stabilität.
    \end{enumerate}
\end{proof}


Im Folgenden sei $Q = L_2(\Omega)/\R$.

\textbf{Ziel:}
Zeige, dass der Druck $p$ in $Q$ liegt, d.h. zu $u\in V$ mit
\begin{eqnarray*}
    a(u, v) = l(v) \qquad \text{für alle } v\in V
\end{eqnarray*}
existiert $p\in L_2(\Omega)$ mit
\begin{eqnarray*}
    b(v, p) = l(v) - a(u, v) \qquad \text{für alle } v\in X.
\end{eqnarray*}

\textbf{Beobachtung:}
\begin{eqnarray*}
    l(v) - a(u, v) = 0 \qquad \text{für alle } v\in V.
\end{eqnarray*}
Um die inf-sup Stabilität auch in unendlich-dimensionalen Räumen zu
gewährleisten  betrachten wir den Komplementärraum
\begin{eqnarray*}
    V^\bot := \{v\in X: a(v, w) = 0 \ \text{für alle } w\in V\}.
\end{eqnarray*}


\begin{Bemerkung}
    Der Komplementärraum $V^\bot$ ist abgeschlossen und ein Teil-Hilbertraum
    von $X$, sodass $X = V + V^\bot$ und $V \cap V^\bot = \{0\}$.
\end{Bemerkung}


\begin{Satz}
    \label{satz:5.10}
    Zu $X = H_0^1(\Omega, \R^d), \ Q = L_2(\Omega) / \R$ und
    $V = \{v\in X: b(v, q) = 0 \text{ für alle } q\in Q\}$,
    definiere $E: Q \to V^\bot$ mit $a(Eq, v) = b(v, q)$ für alle
    $v\in V^\bot$.
    Dann gilt:
    \begin{enumerate}[a)]
      \item
        $E(Q)$ ist dicht in $V^\bot = \{v\in X: a(v, w) = 0 \text{ für alle }
        w\in X\}$.
      \item
        Wenn $\beta > 0$ existiert mit
        $\sup_{\|v\|_1 = 1} b(v, q) \ge \beta \|q\|_0$,
        dann gilt $E(Q) = V^\bot$.
    \end{enumerate}
\end{Satz}


\begin{proof}
    \begin{enumerate}[a)]
      \item
        Für $v\in E(Q)^\bot$ ist
        $a(v, w) = 0$ für $w = Eq\in E(Q)$
        und es folgt für $q\in Q$
        $a(v, Eq) = b(v, q)$,
        sodass $v\in V$.
        Dann ist $E(Q)^\bot \subset V$, d.h.
        $V^\bot \subset (E(Q)^\bot)^\bot = \overline{E(Q)}$.
        Andererseits ist $E(Q) \subset V^\bot$ und $V^\bot$ abgeschlossen,
        sodass $\overline{E(Q)} \subset V^\bot$. \\
        Insgesamt ergibt sich somit $\overline{E(Q)} = V^\bot$.
      \item
        Es gilt
        \begin{eqnarray*}
                \beta \|q\|_0
            &\le& \sup_{\|v\|_1 = 1} b(v, q)
            = \sup_{\|v\|_1 = 1} a(Eq, v) \\
            &\le& C_a \sup_{\|v\|_1 = 1} \|Eq\|_1 \|v\|_1
            = C_a \|Eq\|_1.
        \end{eqnarray*}
        Damit ergibt sich
        \begin{eqnarray*}
            \|q\|_0 \le \frac{C_a}{\beta} \|Eq\|_1.
        \end{eqnarray*}
        Wähle nun $v^\bot\in V^\bot$.
        Dann existiert eine Folge $(q^n)_{n\in \N}$ mit
        $\lim_{n\to \infty} Eq^n = v^\bot$ und es gilt
        \begin{eqnarray*}
                \|q^n - q^m\|_0
            \le \frac{C_a}{\beta} \|Eq^n - Eq^m\|_1.
        \end{eqnarray*}
        Also ist $q^n$ Cauchyfolge in $Q$.
        Somit existiert ein $q\in L_2(\Omega)$ mit $\lim_{n\to \infty} q^n = q$.
        Da $E$ stetig ist folgt $E(Q) = V^\bot$.
    \end{enumerate}
\end{proof}


\begin{Satz}
    \label{satz:5.11}
    Zu $q\in L_2(\Omega)$ mit $\int_\Omega q \,dx = 0$ existiert ein $v\in H_0^1
    (\Omega, \R^d) \ (d \ge 2)$ mit $\nabla \cdot v = q$ und es existiert eine
    Konstante $C > 0$ (unabhängig von $q$) mit
    \begin{eqnarray*}
        \|v\|_1 \le C \|q\|_0.
    \end{eqnarray*}
\end{Satz}


\begin{proof}
    Ohne Beweis!
\end{proof}


Also gilt $B: X \to Q = L_2(\Omega) / \R, \ v \mapsto \nabla \cdot v$ ist
surjektiv mit $V = \Kern B$.

Dadurch erhalten wir
\begin{eqnarray*}
    B: X/V \simeq V^\bot &\to& Q \qquad \ \quad \ \text{Isomorphismus}, \\
    B^\prime: Q &\to& (V^\bot)^\prime \qquad \text{Isomorphismus}, \\
    E = A^{-1} B: Q &\to& X \qquad \ \quad \ \text{Isomorphismus}.
\end{eqnarray*}
Insbesondere ist
\begin{eqnarray*}
    c \|q\|_0 \le \|Eq\|_1 \le C \|q\|_0
\end{eqnarray*}
und es folgt
\begin{eqnarray*}
          c \|q\|_0
    &\le& \|Eq\|_1 \\
    &\le& \frac{1}{\alpha} a(Eq, Eq) \frac{1}{\|Eq\|_1} \\
    &=& \frac{1}{\alpha} \frac{b(Eq, q)}{\|Eq\|_1} \\
    &\le& \frac{1}{\alpha} \sup_{v \neq 0} \frac{b(v, q)}{\|v\|_1}.
\end{eqnarray*}
D.h. $\beta = \alpha c$, also $E(Q) = V^\bot$.


\begin{Folgerung}
    \label{folgerung:5.12}
    Zu $l(v) = (f, v)_0$ existiert eine eindeutige Lösung
    $(u, p)\in X \times Q$ mit
    \begin{eqnarray*}
        a(u, v) + b(v, p) &=& l(v) \qquad \text{für alle } v\in X \\
        b(u, q) &=& 0 \qquad \quad \text{für alle } q\in Q.
    \end{eqnarray*}
\end{Folgerung}


\subsection{Das Mini-Element}



Sei $\Triangulation$ eine affine Triangulierung in Dreiecke. Mit Hilfe der 
Baryzentrischen Koordinaten definiere auf $K$ eine
\emph{kubische Bubble-Funktion}
\begin{eqnarray*}
    b_K(x) = \lambda_0(x) \lambda_1(x) \lambda_2(x) \in \Polynom_3(\R^2),
\end{eqnarray*}
wobei
$\lambda_0 \circ \varphi_K(\hat x) = 1 - \hat x_1 - \hat x_2$ und $\lambda_i
\circ \varphi_K(\hat x) = \hat x_i \ (i = 1, 2)$
, d.h. $b_K(x) = 0$ auf $\partial K$.
Als zugehörige Finite-Element-Räume wählen wir für das
Geschwindigkeitsfeld den Raum der stückweise linearen Elemente
\begin{eqnarray*}
        X_h
    &=& \{v\in X: v|_K\in \Polynom_1(\R^2)^2 + B_K\}
\end{eqnarray*}
der durch den Raum der Bubblefunktionen erweitert wird 
\begin{eqnarray*}
        B_K
    &=& \Span\{b_K e^1, b_K e^2\}.
\end{eqnarray*}
Für die Approximation des Druckes nehmen wir ebenfalls den Raum der
stückweise linearen Elemente
\begin{eqnarray*}
        Q_h
    &=& \{q_h\in L_2(\Omega) / \R: q|_K\in \Polynom_1\}.
\end{eqnarray*}


\begin{Bemerkung}
    Freiheitsgrade in $B_K$ können eliminiert werden.
\end{Bemerkung}


\begin{Lemma}
    \label{lem:5.13}
    Für $v_h\in X_h$ gilt die \emph{inverse Abschätzung}
    \begin{eqnarray*}
        \|v_h\|_1 \le C h^{-1} \|v_h\|_0.
    \end{eqnarray*}
\end{Lemma}


\begin{proof}
    Da $\dim \hat V < \infty$, liefert die Normäquivalenz
    \begin{eqnarray*}
        \|v\|_{1, K} \le C \|v\|_{0, K}
        \qquad \text{für alle } v\in V
    \end{eqnarray*}
    und es folgt
    \begin{eqnarray*}
              \|\nabla v_h\|_{0, K}^2
        &\le& C h^{-2} |J_K| \, \|\hat \nabla \hat v_h\|_{0, \hat K}^2 \\
        &\le& \tilde C h^{-2} |J_K| \, \|\hat v_h\|_{0, \hat K}^2 \\
        &\le& \hat C h^{-2} \|v_h\|_{0, K}^2
    \end{eqnarray*}
    die inverse Abschätzung.
\end{proof}


\begin{Satz}
    \label{satz:5.14}
    Sei $\Omega$ konvex. Dann ist das Mini-Element stabil.
\end{Satz}


\begin{proof}
    Sei $G_h: X \to X_h^0 = (H_0^1(\Omega) \cap \FEMLinear)^d, \ u \mapsto
    u_h$ eine Galerkin-Projektion mit
    \begin{eqnarray*}
        a(G_h u, v_h) = a(u, v_h) \qquad \forall v_h\in X_h^0.
    \end{eqnarray*}
    Da $\Omega$ konvex ist existiert die duale Lösung $w\in X$ mit
    \begin{eqnarray*}
        a(\Phi, w) = (v - G_h v, \Phi)_0
        \qquad \forall \Phi\in X^0.
    \end{eqnarray*}
    Insbesondere ist $w\in H^2(\Omega, \R^d)$ mit $\|w\|_2 \le C
    \|v - G_h v\|_0$.
    Dann gilt
    \begin{eqnarray*}
            \|u - G_h u\|_0^2
        &=& a(w, u - G_h u) \\
        &=& a(w - G_h w, u - G_h u) \\
        &=& a(w - G_h w, u) \\
        &\le& C_a \|w - G_h w\|_1 \|u\|_1 \\
        &\stackrel{\eqref{satz:3.17}}{\le}&
              C h \|w\|_2 \|u\|_1 \\
        &\le& \tilde C h \|G_h u - u\|_0 \|u\|_1
    \end{eqnarray*}
    und wir erhalten
    \begin{eqnarray*}
        \|u - G_h u\|_0 \le C h \|u\|_1.
    \end{eqnarray*}
    Nun definiere $P_h: L_2(\Omega, \R^d) \to X_h, \
    v \mapsto \sum_{K\in \Triangulation} P_K(v) \circ b_K$ mit
    $P_K(v) = \frac{\int_K v \,dx}{\int_K b_K \,dx}$.
    Definiere $F_h v = G_h v + P_K(v - G_h v)$.
    Da $\nabla q_h|_K\in \R^2$ für $q_h\in Q_h$ (Konstant) gilt
    \begin{eqnarray*}
            b(v - F_h v, q_h)
        &=& \int_\Omega \div(v - F_h v) q_h \,dx \\
        &=& \int_{\partial\Omega} \underbrace{(v - F_h v)}_{= 0} \cdot \nu q_h
            \,dx - \int_\Omega (v - F_h v) \cdot \nabla q_h \,dx \\
        &=& \sum_{K\in \Triangulation} \int_K ((v - G_h v) \cdot \nabla q_h -
            P_K(v - G_h v) \cdot \nabla q_h) \,dx \\
        &=& 0.
    \end{eqnarray*}
    Dadurch erhalten wir
    \begin{eqnarray*}
              \|F_h v\|_1
        &\le& \|G_h v\|_1 + \|P_K(v - G_h v)\|_1 \\
        &\stackrel{\eqref{lem:5.13}}{\le}&
              \|v\|_1 + C h^{-1} \|P_K(v - G_h v)\|_0 \\
        &\le& \|v\|_1 + C h^{-1} \|v - G_h v\|_0 \\
        &\le& \|v\|_1 + \hat C h^{-1} \|v\|_1 h \\
        &\le& \tilde C \|v\|_1,
    \end{eqnarray*}
    dass das Mini-Element stabil ist.
\end{proof}


Sei $W = X \times Q, \ X = H_0^1(\Omega, \R^d), \ Q = L_2(\Omega) / \R$ mit
$\|(v, q)\|_W = \sqrt{\|v\|_1 + \|q\|_0}$.

Zu $f\in L_2(\Omega, \R^3)$ existiert eine eindeutige Lösung $(u, p)\in W$:
\begin{eqnarray*}
    a(u, v) + b(v, p) &=& (f, v)_0
    \qquad \text{für alle } v\in X \\
    b(u, q) &=& 0
    \qquad \qquad \text{für alle } q\in Q.
\end{eqnarray*}
Zu $W_h = X_h \times Q_h \subset W$ existiert eine eindeutige Lösung
$(u_h, p_h)\in W_h$ von
\begin{eqnarray*}
    a(u_h, v_h) + b(v_h, p_h) &=& (f, v_h)_0
    \qquad \text{für alle } v_h\in X_h \\
    b(u_h, q_h) &=& 0
    \qquad \ \qquad \ \text{für alle } q_h\in Q_h.
\end{eqnarray*}
Damit wird ein Lösungsoperator $T_h: W \to W_h, \ (u, p) \mapsto (u_h, p_h)$
definiert:

Zu $(u, p)\in W$ sei $(u_h, p_h)\in W_h$ eindeutig bestimmt durch
\begin{eqnarray*}
    a(u_h, w_h) + b(w_h, p_h) &=& a(u, w_h) + b(w_h, p)
    \qquad \text{für alle } w_h\in W_h \\
    b(u_h, r_h) &=& b(u, r_h)
    \qquad \ \qquad \ \qquad \ \text{für alle } r_h\in W_h.
\end{eqnarray*}


\begin{Lemma}
    \label{lem:5.15}
    Unter der Voraussetzung \eqref{def:5.7} gilt
    \begin{enumerate}[a)]
      \item
        $T_h(v_h, q_h) = (v_h, q_h)
        \qquad \forall (v_h, q_h)\in W_h$
      \item
        $\|T_h\|_{L(W, W)} = \sup_{\|(v, q)\|_W = 1} \|T_h(v, q)\|_W \le C$
    \end{enumerate}
\end{Lemma}


\begin{proof}
    \begin{enumerate}[a)]
      \item
        Einfach einsetzen.
      \item
        Zeige: Es existiert $C > 0$ mit $\|(v_h, q_h)\|_W \le C$ für
        $(v_h, q_h) = T_h(v, q)$.

        Wähle dazu $(v, q)\in W$ mit $\|(v, q)\|_W = 1$.
        Es gilt
        \begin{eqnarray*}
                \alpha_0 \|v_h\|_1^2
            &=& a(v_h, v_h)
             =  a(v, v_h) + b(v_h, q) - b(v_h, q_h) \\
            &\le& C_a \|v\|_1 \|v_h\|_1 + \|v_h\|_1 \|q\|_0 + \|v\|_1 \|q_h\|_0
                  \\
            &\le& (C_a + 1) \|v_h\|_1 + \|q_h\|_0 \\
            &\le& \frac{(C_a + 1)^2}{2 \alpha_0} + \frac{\alpha_0}{2}
                  \|v_h\|_1^2 + \|q_h\|_0,
        \end{eqnarray*}
        da $x y \le \frac{s}{2} x^2 + \frac{1}{2 s} y^2$
        (Young'sche Ungleichung) mit $s = \alpha_0, \ x = C_a +1, \
        y = \|v_h\|_1$.
        Au\ss{}erdem ist
        \begin{eqnarray*}
                  \|q_h\|_0
            &\stackrel{\eqref{satz:5.10}}{\le}&
                  \frac{1}{\beta} \sup_{\|w\|_1 = 1} b(w, q_h) \\
            &\stackrel{\eqref{def:5.7}}{\le}&
                  \frac{1}{\beta \beta_0} \sup_{\|w_h\|_1 = 1} b(w_h, q_h) \\
            &\le& \frac{1}{\beta \beta_0} \sup_{\|w_h\|_1 = 1}(a(v, w_h) +
                  b(w_h, q) - a(v_h, w_h)) \\
            &\le& \frac{1}{\beta \beta_0} (C_a + 1 + C_a \|v_h\|_1) \\
            &\le& \frac{1}{\beta \beta_0} \left(C_a + 1 +
                  \frac{C_a^2}{2 \alpha_0} +
                  \frac{\alpha_0}{2} \|v_h\|_1^2\right).
        \end{eqnarray*}
        Zusammen ergibt sich
        \begin{eqnarray*}
            \left(1 + \frac{1}{\beta \beta_0}\right) \|v_h\|_1^2 \le \overline C,
        \end{eqnarray*}
        also
        \begin{eqnarray*}
            \|v_h\|_1 \le C_1
            \qquad \text{und} \qquad
            \|q_h\|_0 \le C_2 + \left(\frac{\alpha_0}{2 \beta \beta_0}\right)
            C_1.
        \end{eqnarray*}
        Schlie\ss{}lich erhalten wir
        \begin{eqnarray*}
            \|T_h(v, q)\|_W \le C
        \end{eqnarray*}
        die Beschränktheit des Lösungsoperators.
    \end{enumerate}
\end{proof}


\begin{Satz}
    \label{satz:5.16}
    Für stabile Stokes-Diskretisierungen gilt
    \begin{eqnarray*}
            \|(u, p) - (u_h, p_h)\|_W
        \le C \inf_{(v_h, q_h)\in W_h} \|(u, p) - (v_h, q_h)\|_W.
    \end{eqnarray*}
\end{Satz}


\begin{proof}
    Aus der Galerkin-Eigenschaft $(id - Th)(v_h, q_h) = 0$ folgt
    \begin{eqnarray*}
            \|(u, p) - (u_h, p_h)\|_W
        &=& \|(id - T_h)(u, p)\|_W \\
        &=& \|id - T_h((u, p) - (v_h, q_h))\|_W \\
        &\le& \|id - T_h\|_{L(W, W)} \|(u, p) - (v_h, q_h)\|_W \\
        &\stackrel{\eqref{lem:5.15}}{\le}&
              (1 + C ) \|(u, p) - (v_h, q_h)\|_W
    \end{eqnarray*}
    die Fehlerabschätzung.
\end{proof}


Folgerung:
Wenn $u\in H^2(\Omega, \R^d), \ p\in H^1(\Omega)$, dann gilt
\begin{eqnarray*}
          \|(u, p) - (u_h, p_h)\|_W
    &\le& C \|(u - \Pi_h^\text{Lsg.} u, p - \Pi_h^\text{Cl.} p)\|_W \\
    &\le& C h (\|u\|_2 + \|p\|_1).
\end{eqnarray*}


\begin{Bemerkung}
    Nach \eqref{lem:5.15} a) gilt $T_h \circ T_h = T_h$ ist Idempotent, d.h.
    $T_h \circ T_h = T_h \neq 0, id$, das wiederum $\|id - T_h\|_{L(W,W)} =
    \|T_h\|_{L(W,W)}$ impliziert.
\end{Bemerkung}


\subsection{Taylor Hood Element}



Sei $\Triangulation$ eine reguläre Triangulierung in Tetraeder (d.h.
$\frac{h_K}{\rho_K} \le C_\mathcal{T}$), 
$X_h = (H_0^1(\Omega) \cap \mathcal{S}_h^2)^3$
der Raum der quadratischen Elemente und
$Q_h = \FEMLinear / \R$ der Raum der linearen Elemente.


\begin{Lemma}
    \label{lem:5.18}
    Für die Clement-Interpolation $\Pi_h: X \to X_h$ und $v\in X$ gilt:
    \begin{enumerate}[a)]
      \item
        $\|v - \Pi_h v\|_0 \le C \|v\|_{1,h}$
        mit
        $\|v\|_{1,h} = \left(\sum_{K\in \Triangulation} h_K^2 \|\nabla v\|_{0,K}
        ^2\right)^\frac{1}{2}$
      \item
        $\|\Pi_h v\|_1 \le C \|v\|_1$
    \end{enumerate}
\end{Lemma}


\begin{proof}
    \begin{enumerate}[a)]
      \item
        Es gilt
        \begin{eqnarray*}
                \|v - \Pi_h v\|_0^2
            &=& \sum_{K\in \Triangulation} \|v - \Pi_h v\|_{0,K}^2 \\
            &\stackrel{\eqref{satz:4.16}}{\le}&
                  C \sum_{K\in \Triangulation} h_K^2 \|\nabla v\|_{0,\omega_K}^2
                  \\
            &\le& C \sum_{K\in \Triangulation} \sum_{K^\prime \subset \omega_K}
                  h_K^2 \|\nabla v\|_{0,K^\prime}^2 \\
            &\le& C C_\mathcal{T} C_\# \sum_{K\in \Triangulation} h_K^2
                  \|\nabla v\|_{0, K}^2,
        \end{eqnarray*}
        denn für $K^\prime \subset \omega_K$ gilt $h_{K^\prime} \le
        C_\mathcal{T} h_K$ mit $C_\mathcal{T} \le \frac{h_K}{\rho_K}$ und
        $\#\{K^\prime: K^\prime \subset \omega_K\} \le C_\#$ abhängig von
        $\frac{h_K}{\rho_K}$ mit $C_\# \le 8 + 12 C_\mathcal{T} +
        6 C_\mathcal{T}^2$.
      \item
        Es gilt
        \begin{eqnarray*}
                  \|\nabla v\|_{0,K}
            &\le& C_\mathcal{T} |F_K^{-1}| J_K^\frac{1}{2}
                  \|\hat\nabla \hat v\|_{0,\hat K} \\
            &\le& C |F_K^{-1}| J_K^\frac{1}{2} \|\hat v\|_{0,\hat K} \\
            &\le& C h_K^{-1} \|v\|_{0,K}.
        \end{eqnarray*}
        Dann ist
        \begin{eqnarray*}
                \|\nabla (v - \Pi_h v)\|_0^2
            &=& \sum_{K\in \Triangulation} \|\nabla (v - \Pi_h v) \|_{0,K}^2 \\
            &\le& C \sum_{K\in \Triangulation} h_K^{-2} \|v - \Pi_h v\|_{0,K}^2
                  \\
            &\stackrel{\eqref{satz:4.16}}{\le}&
                  C^\prime \sum_{K\in \Triangulation}
                  \|\nabla v\|_{0,\omega_K}^2 \\
            &\le& C^\prime C_\# \sum_{K\in \Triangulation} \|\nabla v\|_{0,K}^2\\
            &\le& \tilde C \|\nabla v\|_0^2
        \end{eqnarray*}
        und es folgt
        \begin{eqnarray*}
                  \|\Pi_h v\|_1
            &\le& \|v\|_1 + \|v - \Pi_h v\|_1 \\
            &\le& \|v\|_1 + C_P \|\nabla (v - \Pi_h v)\|_0 \\
            &\le& \|v\|_1 + \tilde C C_P \|\nabla v\|_0 \\
            &\le& \hat C \|v\|_1,
        \end{eqnarray*}
        was zu zeigen war.
    \end{enumerate}
\end{proof}


\begin{Lemma}
    \label{lem:5.19}
    Es existiert $\beta_1 > 0$ mit $\sup_{\|w_h\|_1 = 1} b(w_h, q_h) \ge \beta_1
    \|q_h\|_{1,h}$ für alle $q_h\in Q_h$.
\end{Lemma}


\begin{proof}
    Sei $K = \conv\{z^0, z^1, z^2, z^3\}\in \Triangulation$ ein Tetraeder und
    $q_h|_K\in \Polynom_1(\R^3)$.
    Die Freiheitsgrade von $v_h|_K$ sind die Ecken $z^m$ und die
    Knotenmittelpunkte $z^{mk} = \frac{1}{2}(z^m + z^k)$.
    Setze $v_h|_K$ durch
    \begin{eqnarray*}
            v_h(z^j)
        &=& 0
            \qquad \text{und} \\
            v_h(z^{mk})
        &=& - (\nabla q_h|_K \cdot (z^i - z^j)) (z^i - z^j)\in \R^3.
    \end{eqnarray*}
    Aus $|z^m - z^k| \le h_K$ und $\nabla q_h|_K\in \R^3$ folgt
    \begin{eqnarray*}
            |v_h(z^{mk})|
        \le h_K^2 |\nabla q_h|_K|.
    \end{eqnarray*}
    Dann ist
    \begin{eqnarray*}
            \|\nabla v\|_{0,K}^2
        &=& \left\|\sum_{m < k} v_h(z^{mk}) \nabla \Phi_{ij}\right\|_{0,K}^2 \\
        &\le& (h_K^2 |\nabla q_h|_K|)^2 \sum_{m < k}
              \|\nabla \Phi^{mk}\|_{0,K}^2 \\
        &\le& h_K^4 |q_h|_K|^2 \sum_{m < k} h_K^{-2} \|1\|_{0,K}^2 \\
        &\le& 6 h_K^2 \|\nabla q_h\|_{0,K}^2.
    \end{eqnarray*}
    Die Poinc\'are Ungleichung liefert uns
    \begin{eqnarray*}
        \|v_h\|_1 \le C_P \|\nabla v_h\|_1 \le C \|q\|_{1,h}.
    \end{eqnarray*}
    Für quadratische Polynome $p\in \Polynom_2(\R^2)$ gilt die Quadraturformel
    \begin{eqnarray*}
          \int_K p(x) \,dx
        = \meas_3(K) \left(\frac{1}{5} \sum_{m < k} p(z^{mk}) - \frac{1}{20}
          \sum_m p(z^m)\right).
    \end{eqnarray*}
    Damit ergibt sich
    \begin{eqnarray*}
            b(v_h, q_h)
        &=& \int_\Omega \nabla \cdot v_h q_h \,dx \\
        &=& - \int_\Omega v_h \cdot \nabla q_h \,dx \\
        &=& - \sum_{K\in \Triangulation} \meas_3(K) \frac{1}{5} \sum_{m < k}
            v_h(z^{mk}) \cdot \nabla q_h|_K \\
        &=& \sum_{K\in \Triangulation} \frac{\meas_3(K)}{5} \sum_{m < k}
            |\nabla q_h|_K \cdot (z^m - z^k)|^2 \\
        &\ge& C_\mathcal{T} \sum_{K\in \Triangulation} \frac{\|1\|_{0,K}^2}{5}
              \sum_{m < k} h_K^2 |\nabla q_h|_K|^2 \\
        &\ge& C \|q_h\|_{1,h}^2 \\
        &\ge& \tilde C \|q_h\|_{1,h} \|v_h\|_1,
    \end{eqnarray*}
    was zu zeigen war.
\end{proof}


\begin{Satz}
    \label{satz:5.20}
    Das Taylor-Hood-Element ist inf-sup stabil.
\end{Satz}


\begin{proof}
    Die Cl\'ement-Interpolation $\Pi_h: X \to X_h$ erfüllt
    \begin{eqnarray*}
            \|v - \Pi_h v\|_0
        \le \overline C h \|v\|_1
    \end{eqnarray*}
    und die inverse Ungleichung liefert uns
    \begin{eqnarray*}
              \|\Pi_h v\|_1
        &\le& \|v\|_1 + \|\Pi_h v - v\|_1 \\
        &\le& \|v\|_1 + \tilde C h^{-1} \|\Pi_h v - v\|_0 \\
        &\le& (1 + \tilde C h^{-1} \overline C h) \|v\|_1 \\
        &\le& C \|v\|_1.
    \end{eqnarray*}
    Dann ist
    \begin{eqnarray*}
            \sup_{v_h \neq 0} \frac{\int_\Omega q_h \nabla \cdot v_h \,dx}
            {\|v_h\|_1}
        &=& \sup_{v_h \neq 0} \frac{\int_\Omega q_h \nabla \cdot (\Pi_h v) \,dx}
            {\|\Pi_h v\|_1} \\
        &\ge& \frac{1}{C} \sup_{v \neq 0} \frac{b(q_h, \Pi_h v)}{\|v\|_1} \\
        &\ge& \frac{1}{C} \sup_{v \neq 0} \left(\frac{b(q_h, v)}{\|v\|_1} +
              \frac{b(q_h, \Pi_h v - v)}{\|v\|_1}\right) \\
        &\ge& \frac{1}{C} \left(\beta \|q_h\|_0 - \sum_{K\in \Triangulation}
              h_K^2 \|\nabla q_h\|_{0,K} h_K^{-2} \|\Pi_h v - v\|_{0,K}
              \frac{1}{\|v\|_1}\right) \\
        &\ge& \frac{1}{C} \left(\beta \|q_h\|_0 - \tilde C \|q_h\|_{1,h}
              \frac{\|v\|_1}{\|v\|_1}\right) \\
        &\ge& C_1 \|q_h\|_0 - C_2 \|q_h\|_{1,h}
    \end{eqnarray*}
    und es gilt
    \begin{eqnarray*}
            \sup_{\|v_h\|_1 = 1} b(v_h, q_h)
        \ge C_1 \|q_h\|_0 - C_2 \|q_h\|_{1,h}.
    \end{eqnarray*}
    Mit $\sup_{\|v_h\|_1 = 1} b(v_h, q_h) \ge \beta_1 \|q_h\|_{1,h}$ folgt
    \begin{eqnarray*}
        (\beta_1 + C_2) \sup_{\|v_h\|_1 = 1} b(v_h, q_h)
        \ge \beta_1 C_1 \|q_h\|_0.
    \end{eqnarray*}
    Die inf-sup Konstante ergibt sich also zu $\beta_0 = \frac{\beta_1 C_1}
    {\beta_1 + C_2}$.
\end{proof}


\begin{Folgerung}
    \label{folgerung:5.21}
    Wenn $u\in H^3(\Omega, \R^3), \ p\in H^2(\Omega)$, dann gilt
    \begin{eqnarray*}
        \|u - u_h\|_1 + \|p - p_h\|_0 \le C h^2 (\|u\|_3 + \|p\|_2).
    \end{eqnarray*}
\end{Folgerung}
