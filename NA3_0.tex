\pagestyle{empty}

\includegraphics[width=4.2cm]{kit-logo}    

\vspace{79.385mm}


\vspace{9.385mm}

\vspace{2mm}

{\LARGE\bf Finite Elemente Methoden}

\vspace{9.385mm}

{\bf Skript zur Vorlesung im Wintersemester 2014/2015}

\vspace{9.385mm}

\href{http://www.mathematik.uni-karlsruhe.de/ianm3/~wieners}
{{\large\bf Christian Wieners }}\\[2mm]
{\bf Institut für Angewandte und Numerische Mathematik, KIT}

\vspace{9.385mm}

{Version vom \today}

\clearpage
%\renewcommand{\rmdefault}{cm}

\tableofcontents

\clearpage

Diese Vorlesung ist eine Einführung in die 
numerische Verfahren zur Lösung von partiellen Differentialgleichungen.
Hier beschränken wir uns im Wesentlichen auf elliptische 
Randwertaufgaben in zwei Raumdimensionen, das Modellproblem und die 
Stokes-Gleichungen. Neben der Konstruktion der Verfahren wird im
Wesentlichen die Konvergenzanalyse betrachtet. Das Skript soll auch als
Grundlage für vertiefende Vorlesungen zur numerischen Behandlung 
von weiteren partiellen Differentialgleichungen dienen.

Der gesamte Stoff wurde aus Standard-Lehrbüchern zusammengestellt;
die wesentlichen Quellen sind unten angegeben.

Ich danke Herrn Ramin
Shirazi-Nejad für die mühsame \LaTeX-Arbeit.

\section*{Literatur}
%\markboth{Literatur}{Literatur}
%\addcontentsline{toc}{section}{Literatur}

Knabner/Angermann: Numerik partieller Differentialgleichungen, Springer 2000

Braess: Finite Elemente, Springer 2013

Ciarlet: The finite element method for elliptic problems, North-Holland

Deuflhard/Bornemann: Numerische Mathematik III, de Gruyter 2002

Brenner/Scott: The Mathematical Theory of Finite Element Methods, Springer 2008

Ern/Guermond: Theory and Practice of Finite Elements, Springer 2004

Hanke ????

\clearpage

\pagestyle{plain}

\setcounter{page}{1}
