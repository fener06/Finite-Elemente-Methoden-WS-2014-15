\section{Einführung}

\textbf{Ziel:} Analyse von Diskretisierungs- und Lösungsmethoden für PDEs partial differential equations.

\subsection{PDE-Modelle}

\subsection{Diffusion (als Prototyp = Modellproblem)}
$\Omega \in \R^D$
Partielle Differentialgleichungen (PDE) sind Gleichungen, die partielle
Ableitungen einer unbekannten Funktion $u$ in mehreren Variablen enthalten.
Im Folgenden nehmen wir an, dass alle Grö\ss en hinreichend glatt sind.


\begin{Bezeichnung}
    \begin{align*}
        &u = u(x,t)     && \text{Zustandsgrö\ss e} \\
        &x = \left(\begin{smallmatrix}
                x_1 \\
                \vdots \\
                x_d
              \end{smallmatrix}\right)\in\Omega\subset\R^d
                        && \text{Ortsvariable} \\
        &t\in [0,T]	    && \text{Zeitvariable} \\
        &\partial_t = \frac{\partial}{\partial t}
                        && \text{partielle Ableitung nach der Zeit}
    \end{align*}
\end{Bezeichnung}


Sei $S(u)$ eine (Volumen-)Dichte oder Konzentration.
Für jedes Kontroll-Volumen $V\subset\Omega$ beschreibt
\begin{eqnarray*}
     \partial_t \int_{V} S(u(x, t))\,dx
\end{eqnarray*}
die zeitliche änderung von $S(u)$ in $V$.


Die änderung entsteht durch:


\begin{enumerate}[1)]
  \item
      Quellen und Senken $Q = Q(x, t, u(x,t))$: \\
      Falls $Q > 0$, dann wird $S(u)$ erzeugt. \\
      Falls $Q < 0$, dann wird $S(u)$ verringert.
  \item
      Fluss $J(x,t)$ durch den Rand von $\partial V$: \\
      $J = \left(\begin{smallmatrix}
              J_1 \\ \vdots \\ J_d
	      \end{smallmatrix}\right)$,
      $J_i$ Fluss-Dichte in $x_i$-Richtung. \\
      $J\nu = \sum_{i=0}^d \nu_i J_i$ Fluss über den Rand. \\
      (dabei ist $\nu\colon \partial V \to \R^d$ mit $|\nu| = 1$ die
      äu\ss{}ere Einheitsnormale)
\end{enumerate}


Die (Massen-) Erhaltung ergibt, falls das Volumen $V$ nicht von der Zeit
abhängt
\begin{eqnarray*}
      \int_V \partial_t S(u(x,t)) \dx
    = - \int_{\partial V} J(x,t) \nu \da + \int_V Q(x,t,u(x,t))) \dx.
\end{eqnarray*}
Sie sagt aus das weder Masse erzeugt noch erschaffen werden kann. Die
zeitlichen änderungen im Volumen $V$ sind Folgen des (Masse-) Transports
über den Rand $\partial V$.


\textbf{Satz von Gau\ss}

  Sei $V\subset\R^d$ beschränkt mit stückweise glattem Rand. Sei $F\in
  C^1(V,\R^d)$ ein Vektorfeld. Dann gilt
  \begin{eqnarray*}
      \int_V \operatorname{div} F(x) \dx = \int_{\partial V} F(x) \nu(x) \da,
  \end{eqnarray*}
  mit der Divergenz $\operatorname{div} F = \nabla \cdot F = \sum_{i=0}^d
  \partial_i F_i, \
  \nabla = \left(\begin{smallmatrix}
                \partial_1 \\ \vdots \\ \partial_d
           \end{smallmatrix}\right) \text{ und}
  \ \partial_i = \frac{\partial}{\partial x_i}$.


\begin{Anwendung}
    Es gilt also für alle $V\subset \Omega$
    \begin{eqnarray*}
        \int_V \partial_tS(u(x,t)) + \operatorname{div} J(x,t)
        - Q(x,t,u(x,t)))\,dx = 0.
    \end{eqnarray*}
    Da $V\subset\Omega$ beliebig folgt daraus die
    \emph{Erhaltungsgleichung} in differentieller Form
    \begin{eqnarray*}
        \partial_tS(u(x,t)) + \operatorname{div} J(x,t) - Q(x,t,u(x,t)) = 0,
        \qquad \forall x\in \Omega, \ t\in [0,T].
    \end{eqnarray*}
\end{Anwendung}


Die Erhaltungsgleichung muss durch (phenomenologische)
\emph{konstitutive Gesetze} geschlossen werden, die eine Beziehung zwischen den
Zustandsgrö\ss en $u$ und den Fluss $J$ postulieren.


\begin{Beispiel}
    $\Omega$ sei mit einer Flüssigkeit gefüllt und $S(u) = u$ die
    Konzentration einer gelösten Substanz. \\
    \emph{Szenario I} \\
    Flüssigkeit ruht, die Konzentration strebt ins Gleichgewicht von hoher zu
    niedriger Konzentration:
    \begin{eqnarray*}
        J^{(1)} = -K\nabla u, \qquad K > 0 \qquad \left[\frac{m^2}{s}\right]
        \qquad \text{(Materialparameter)}\\
    \end{eqnarray*}
    und wir erhalten die \emph{Diffusionsgleichung}
    \begin{eqnarray*}
        \partial_t u - \nabla \cdot (K \nabla u) = Q.
    \end{eqnarray*}
    \emph{Szenario II} \\
    Flüssigkeit bewegt sich mit der Geschwindigkeit $c\colon \Omega \to \R^d$:
    \begin{eqnarray*}
        J^{(2)} = cu
    \end{eqnarray*}
    sodass die \emph{Transportgleichung} gegeben ist durch
    \begin{eqnarray*}
        \partial_t u + \nabla \cdot (cu) = 0.
    \end{eqnarray*}
    Zusammen ergibt sich die \emph{Konvektions-Diffusions-Gleichung}
    \begin{eqnarray*}
        \partial_t u - \nabla \cdot (K \nabla u  - cu) = Q.
    \end{eqnarray*}
\end{Beispiel}


\begin{Entdimensionalisierung}
    Fixiere Refernzgrö\ss en $x_\text{ref}, \ u_\text{ref}$ und $t_\text{ref}$.
    Reskaliere dazu
    \begin{eqnarray*}
        x\mapsto \frac{x_i}{x_\text{ref}}, &
        u\mapsto \frac{u_i}{u_\text{ref}}, &
        t\mapsto \frac{t_i}{t_\text{ref}}.
    \end{eqnarray*}
    Somit erhält man keine Einheiten. Die typische Grö\ss enordnung ist 1.
\end{Entdimensionalisierung}



\subsection{Randbedingungen}


Betrachte $\partial_t S(x,u(x,t)) + \nabla \cdot (C(x,u(x,t)) - K(x,\nabla
u(x,t))) = Q(x,t,u(x,t))$ im Raum-Zeit-Zylinder $(x,t)\in \Omega\times (0,T)
\subset \R^d\times \R$.

Im Wesentlichen unterscheiden wir drei Randbedingungen, wozu wir den Rand in
drei disjunkte Teile unterteilen:
$\partial\Omega = \Gamma_1\cup \Gamma_2\cup \Gamma_3\colon$

\begin{enumerate}[1)]
    \item
	Fluss Randbedingungen auf $\Gamma_1$ (Neumann RB)
	\begin{eqnarray*}
	    -(C(u) - K(\nabla u)) \cdot \nu = g_1
	\end{eqnarray*}
    \item
	Gemischte Randbedingungen auf $\Gamma_2$ (Robin RB)
	\begin{eqnarray*}
	    -(C(u) - K(\nabla u)) \cdot \nu + \alpha u = g_2
	\end{eqnarray*}
     \item
	Dirichlet Randbedingungen auf $\Gamma_3$ (wesentliche RB)
	\begin{eqnarray*}
	    u = g_3
	\end{eqnarray*}
\end{enumerate}


Es liegen homogene Randbedingungen vor falls $g_i = 0$.


\begin{Spezialfälle}
    Wir erhalten:
    \begin{enumerate}[1)]
	\item
	    den Sturm Liouville Operator für $d=1$ im Raum.
	\item
	    eine stationäre Gleichung, falls keine Zeitableitung vorliegt.
	\item
	    die lineare Gleichung $Q(x,u) = f(x) - \underbrace{q(x)u(x)}
	    _{\text{Reaktionsterm}}$
	    mit $C(u) = cu, \ K(\nabla u) = K\nabla u$.
	\item
	    die Poisson-Gleichung $-\Delta u = f$ für $c\equiv 0, \
	    K\equiv 1, \ q\equiv 0$.
	\item
	    die Laplace-Gleichung $-\Delta u = 0$ für $f\equiv 0$.
    \end{enumerate}
\end{Spezialfälle}



\subsection{Modell-Problem (die Poisson-Gleichung)}


\begin{Definition}
\label{def:1.1}
    Sei $\Omega\subset \R$ beschränkt mit stückweise glattem Rand. Sei
    $f\in C(\Omega), \ g\in C(\partial\Omega)$. Dann hei\ss t $u\in C^2(\Omega)
    \cap C(\overline\Omega)$ mit
    \begin{eqnarray*}
          -\Delta u(x)
        &=& f(x), \quad x\in \Omega \qquad \ \ (\text{in } \Omega) \\
            u(x)
        &=& g(x), \quad x\in \partial\Omega \qquad (\text{auf } \partial\Omega)
    \end{eqnarray*}
    eine (klassische) Lösung der Poisson-Gleichung (mit Dirichlet
    Randbedingungen).
\end{Definition}


\begin{Beispiel}
    Mit Hilfe von Fundamentallösungen kann eine Lösung der Poisson-
    Gleichung konstruiert werden. Hierbei ist $K(x,y) = \Psi(|x-y|)$ die
    Fundamentallösung, wobei
    \begin{eqnarray*}
        \Psi(r) = \begin{cases}
                      \frac{r^{2-d}}{(2-d)\omega_d}, 	& d>2 \\
                      \frac{\log{r}}{2\pi}, 		& d=2
                  \end{cases},
    \end{eqnarray*}
    und $\omega_d$ das Volumen der Einheitssphäre ist. Sie erfüllt
    $\Delta_x K(x,y) = 0$ für $x\neq y$. Falls $u\in C^2(\overline\Omega)$
    ist, dann ist eine Lösung der Poisson-Gleichung gegeben durch
    \begin{eqnarray*}
          u(y)
        = \int_{\partial \Omega} \big(\partial_\nu K(x,y) g(x) - K(x,y) h(x) \big) \da_x
          -\int_\Omega K(x,y) f(x) \dx
    \end{eqnarray*}
    mit der äu\ss{}eren Einheitsnormalen $\nu\colon \partial\Omega \to
    \R^d, \ |\nu| = 1, \ h(x) = \partial_\nu u(x)$, wobei die Normalenableitung
    $\partial_\nu = \nu\cdot\nabla = \sum_{i=0}^d \nu_i \partial_i$ und
    $g(x) = u(x), \ f(x) = -\Delta u(x)$.

    $\textbf{Problem:}$ Wenn $f$ und $g$ vorgegeben sind, muss $h$ bestimmt
    werden!
\end{Beispiel}


\begin{Bezeichnung}
    Eine Funktion $u \in C^2(\Omega)$ heißt harmonisch, falls $\Delta u = 0$ in $\Omega$ gilt.
\end{Bezeichnung}


\begin{Bemerkung}
    Wenn $\Delta u \le 0$, dann ist $u(y) \le \frac{1}{R^d \omega_d}
    \int_{|x-y|=R} u(x) \da$.
\end{Bemerkung}


\begin{Maximumsprinzip}
\label{max:1.2}
    Sei $u\in C^2(\Omega)\cap C(\overline\Omega)$ mit $\Delta u \le 0$.
    Dann gilt
    \begin{eqnarray*}
        \sup_{x\in \Omega} u(x) = \sup_{x\in \partial\Omega} u(x).
    \end{eqnarray*}
\end{Maximumsprinzip}


\begin{proof}
    Ohne Beweis!
\end{proof}


\begin{Folgerung}
    \label{fol:1.3}
    Das Poisson-Problem \eqref{def:1.1} hat höchstens eine Lösung.
\end{Folgerung}


\begin{proof}
    Seien $u$ und $v$ Lösungen von
    \begin{eqnarray*}
        -\Delta u(x) &=& f(x), \
        -\Delta  v(x) = f(x)
        \qquad \text{ in } \Omega \\
        u(x) &=& g(x),
        \ \ \quad \ v(x) = g(x)
        \qquad \text{ auf } \partial\Omega. 
    \end{eqnarray*}
    Dann  löst $w(x) = u(x) - v(x)$
    \begin{eqnarray*}
        -\Delta w(x) &=& 0 \qquad \text{ in } \Omega \\
        w(x) &=& 0 \qquad \text{ auf } \partial\Omega.
    \end{eqnarray*}
    Mit \eqref{max:1.2} folgt
    \begin{eqnarray*}
        \sup_{x\in \Omega} w(x) = \sup_{x\in \partial\Omega} w(x) = 0
        \quad \text{und } \inf_{x\in \Omega} w(x) = 0.
    \end{eqnarray*}
    Also $w(x) \equiv 0$.
\end{proof}


\begin{Definition}
    \label{def:1.4}
    $\Omega\subset \R^d$ ist ein Lipschitz-Gebiet, wenn
    \begin{enumerate}[a)]
	\item
	    $\Omega$ ist offen in $\R^d$,
	\item
	    $\Omega$ ist zusammenhängend,
	\item
	    $\partial\Omega$ ist eine Lipschitz-Mannigfaltigkeit, d.h. für alle
	    $y\in \partial\Omega$ existiert eine lokale Karte $\Psi\in C^{0,1}
        (V, \R^d)$ mit $V\subset \R^{d-1}$ offen und $y\in \Psi(V)
        \subset \partial\Omega$.
    \end{enumerate}
\end{Definition}


\begin{Beispiel}
    Das Einheitsquadrat
    $\Omega = (0,1)^2$ ist ein Lipschitz-Gebiet. Offensichtlich ist $\Omega$
    offen und zusammenhängend. Für die Lipschitz-Mannigfaltigkeit $\partial \Omega$ ist
    beispielsweise für
    $\ y =  \left(\begin{smallmatrix}
		0 \\ 0
	    \end{smallmatrix}\right)$
    die Abbildung
    $\Psi(t) = \left(\begin{smallmatrix}
		  t + |t| \\ -t + |t|
              \end{smallmatrix}\right)$
    für $t \in V = (-\frac{1}{2},\frac{1}{2})$ eine lokale Karte mit $y = \Psi(0)$.
\end{Beispiel}


\begin{Satz}
    \label{satz:1.5}
    In einem Lipschitz-Gebiet hat das Poisson-Problem \eqref{def:1.1} mit
    Dirichlet Randbedingungen eine eindeutige Lösung.
\end{Satz}


\begin{proof}
    Ohne Beweis!
\end{proof}
